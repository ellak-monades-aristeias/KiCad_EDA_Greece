\documentclass[a4paper]{article}

\usepackage{fontspec}
%για επιλογή γραμματοσειρών (μεταξύ άλλων)
\usepackage{polyglossia}
%για συλλαβισμό λέξεων (μεταξύ άλλων)
\usepackage{geometry}
%για μορφή σελίδας a4 (μεταξύ άλλων)
\usepackage{longtable}
\usepackage{tabu}
%για προχωρημένους πίνακες

\setdefaultlanguage{greek}
\setotherlanguage{english}

\setmainfont{DejaVu Sans}

\title{Tutorial - Οδηγός Εκμάθησης KiCad}
\author{Μιχάλης Μισιρλής}
\date{Οκτώβριος 2015}

\begin{document}

\maketitle

\section{Εισαγωγή}
Αυτός ο οδηγός εκμάθησης έχει ως στόχο την εξοικοίωση του Έλληνα χρήστη του KiCad μέσα από την ανάπτυξη ενός απλού αλλά πλήρους κυκλώματος (ηλεκτρονική συσκευή) ξεκινώντας από την βασική ιδέα και το σκοπό της συσκευής και καταλήγοντας στην παραγωγή αρχείων κατασκευής ώστε να μπορεί να παραχθεί η συσκευή από οποιοδήποτε εργοστάσιο ή εταιρεία παραγωγής πλακετών.

\subsection{Προαπαιτούμενα}
Αυτός ο οδηγός έχει βασιστεί και απαιτεί το πρόγραμμα KiCad έκδοση 4.0.0 [TBD] με την Ελληνική μετάφραση [TBD]. Επίσης ο οδηγός θεωρεί ότι ο χρήστης είναι εξοικοιωμένος με τις βασικές έννοιες ηλεκτρονικού σχεδιασμού και τις βασικές έννοιες χρήσης ηλεκτρονικών υπολογιστών.

\subsection{Λεξικό όρων}
\begin{longtabu}{ c c }
    Ελληνικός όρος	&	Αγγλικός όρος	\\
    3Δ προβολή	&	3d viewer	\\
    αγκύρωση	&	anchor	\\
    ακροδέκτης	&	pin	\\
    ακτίνα	&	spoke	\\
    ακτίνα	&	spoke	\\
    αλφαριθμητικό	&	string	\\
    ανάλυση	&	parse	\\
    αναφορά, ονομασία αναφοράς	&	reference	\\
    αντεστραμμένο	&	inverted	\\
    αντικολλητικής μάσκας	&	solder resist mask	\\
    αντίσταση συγκόλλησης	&	solder resist	\\
    απόδοση	&	render	\\
    απόκλιση	&	skew	\\
    αποτύπωμα	&	footprint	\\
    γομα	&	eraser	\\
    γραμμή γραφικών	&	graphic line	\\
    διάκενο	&	clearance	\\
    διαμπερές	&	through (hole)	\\
    διάτρηση	&	drill	\\
    δίαυλος	&	bus	\\
    δίκτυα μίας έδρας	&	single pad nets	\\
    δρομολόγηση	&	route	\\
    δρόμος	&	pcb trace / track	\\
    έδρα	&	antipad clearance	\\
    έδρα	&	pad	\\
    εηκ	&	erc	\\
    εκς	&	drc	\\
    ελεγκτής ηλεκτρικών κανόνων	&	electrical rules checker	\\
    εξάρτημα	&	component	\\
    έξοδος	&	output	\\
    επάνω πλευρά	&	top side	\\
    επεξεργαστής σχεδίασης σελίδας	&	page layout editor	\\
    επιλογές	&	options / choices / selections	\\
    επίπεδα	&	layers	\\
    επίπεδα eco	&	eco layers	\\
    επίπεδο σχεδίων	&	drawings layer	\\
    επίπεδοα επικόλλησης	&	adhesive layers	\\
    επισήμανση	&	flag	\\
    ζώνες	&	zones	\\
    θαμμένο via	&	buried via	\\
    θύρα	&	port	\\
    θύρα ισχύος	&	power port	\\
    κανόνες σχεδιασμού	&	design rule	\\
    κατοπτρισμός, καθρέφτισμα	&	mirror	\\
    κάτω πλευρά	&	bottom side	\\
    κενό	&	gap	\\
    κοίλες γωνίες	&	fillet	\\
    κώδικας g	&	g code	\\
    λειτουργία γεμίσματος	&	filled mode	\\
    λειτουργία περιγράμματος	&	sketch mode	\\
    λειτουργία περιγράμματος	&	outline mode	\\
    λίστα δικτύων	&	netlist	\\
    λίστα υλικών	&	bill of materials	\\
    λοξότμηση	&	chamfer	\\
    μάσκα συγκόλλησης	&	solder mask	\\
    μέρη	&	unit	\\
    μεταξοτυπία	&	silk screen	\\
    Μετατόπιση \& σπρώξιμο	&	Push \& shove	\\
    μετατόπιση,θέση	&	offset	\\
    μηχανές ανύψωσης και τοποθέτησης	&	“pick and place”	\\
    μικρο-ρύθμιση, ρύθμιση	&	tune	\\
    μπροστινή όψη	&	front view	\\
    ονοματοδοσία	&	annotate	\\
    όρια πλακέτας	&	pcb/board edge	\\
    παλαιότερου τύπου	&	legacy	\\
    παραπομπή πλαισίου	&	frame reference	\\
    πάστα συγκόλλησης	&	solder paste	\\
    περιγράμμα πλακέτας	&	board outline	\\
    πίνακας	&	matrix	\\
    πίσω όψη	&	back view	\\
    πλαίσιο	&	box (για ui)	\\
    πλοηγός	&	navigator	\\
    πρόγραμμα προβολής pdf	&	pdf viewer	\\
    προκαθορισμένου	&	default	\\
    πρόσθετα	&	plugin	\\
    προτιμώμενο	&	favourite	\\
    ρυθμίσεις	&	settings	\\
    ρυθμίσεις	&	configurations	\\
    σκιαγράφηση	&	hatch	\\
    στελέχη	&	stubs	\\
    στοίβα έδρας	&	padstack	\\
    σύγκρουση, ασυμβατότητα	&	conflict	\\
    συντομεύσεις	&	hotkeys	\\
    συνώνυμο	&	alias	\\
    σύρμα	&	wire	\\
    σύρσιμο	&	drag	\\
    συστοιχία	&	array	\\
    σχεδίαση	&	design	\\
    σχεδίασης σελίδας	&	page layout	\\
    σχέδιο	&	drawing	\\
    σχεδιογραφώ	&	plot	\\
    σχηματικό διάγραμμα, σχηματικό	&	schematic	\\
    τόξο	&	arc	\\
    τρύπα	&	hole	\\
    φάκελος	&	folder / directory	\\
    φωλιά συνδέσεων	&	ratsnest	\\
    χαλκός	&	copper pour	\\
    χώρος αποτυπώματος	&	courtyard	\\
    χώρος σχεδίασης	&	canvas	\\
    ψευδώνυμο	&	nickname	\\
    excellon	&	excellon	\\
    excellon	&	excellon	\\
    gerber	&	gerber	\\
    NPTH	&	NPTH	\\
    PTH	&	PTH	\\
    SMD	&	SMD	\\
    VIA	&	VIA	\\
\end{longtabu}

\subsection{Περιγραφή συσκευής}
Η συσκευή ονομάζεται USB2UART και είναι ένας μετατροπέας USB σε 3.3V UART, βασισμένος στο ολοκληρωμένο CP2104 από την εταιρεία Silicon Labs. υποστηρίζεται από τα περισσότερα σύγχρονα λειτουργικά συστήματα ως εικονική σειριακή θύρα. Συνδέεται με έναν standard type B connector από τη μία πλευρά και ένα four-pin header από την άλλη. Περιλαμβάνει power status LED και RX/TX LED.

\subsection{Βασικές Έννοιες}
Το KiCad είναι μία ολοκληρωμένη σουίτα εφαρμογών σχεδίασης ηλεκτρονικών κυκλωμάτων EDA (Electronic Design Automation). Με το KiCad είναι εφικτή η σχεδίαση σχηματικών και τυπωμένων ηλεκτρονικών κυκλωμάτων.

Τα σχηματικά κυκλώματα  σχεδιάζονται στο KiCad με την εφαρμογή EEschema. Αποτελούνται από εξαρτήματα, για παράδειγμα ένας πυκνωτής ή ένα ολοκληρωμένο, συνδέσεις μεταξύ τους, και άλλα στοιχεία. Τα σχηματικά κυκλώματα είναι οργανωμένα σε σχηματικά φύλλα. Κάθε σχηματικό φύλλο αποτελεί και ένα αρχείο στον υπολογιστή του χρήστη.
Τα εξαρτήματα των σχηματικών μπορεί να ανήκουν σε μία βιβλιοθήκη του KiCad ή μπορεί να τα έχει σχεδιάσει από το μηδέν ο χρήστης χρησιμοποιώντας το σχετικό εργαλείο του KiCad. Κάθε εξάρτημα αποτελεί και ένα αρχείο στον υπολογιστή του χρήστη.

Τα τυπωμένα κυκλώματα σχεδιάζονται στο KiCad με την εφαρμογή PCBnew. Αποτελούνται από πλακέτες που περιέχουν αποτυπώματα εξαρτημάτων, συνδέσεις μεταξύ τους μέσω δρόμων, και από τρύπες, via, κ.α.
Κάθε πλακέτα αποτελεί και ένα αρχείο στον υπολογιστή του χρήστη. 
Τα αποτυπώματα εξαρτημάτων μπορεί να ανήκουν σε μία βιβλιοθήκη του KiCad ή μπορεί να τα έχει σχεδιάσει από το μηδέν ο χρήστης χρησιμοποιώντας το σχετικό εργαλείο του KiCad. Κάθε αποτύπωμα αποτελεί και ένα αρχείο στον υπολογιστή του χρήστη. 

Τα αποτυπώματα μπορεί να έχουν τρισδιάστατη απεικόνιση. Οι τρισδιάστατες απεικονίσεις μπορεί να ανήκουν σε μία βιβλιοθήκη του KiCad ή μπορεί να τα έχει σχεδιάσει από το μηδέν ο χρήστης χρησιμοποιώντας κάποιο εργαλείο τρισδιάστατης σχεδίασης. Κάθε τρισδιάστατη απεικόνιση αποτελεί και ένα αρχείο στον υπολογιστή του χρήστη. 

Το σύνολο του κυκλώματος (σχηματικό, πλακέτα) οργανώνεται από το KiCad σε έργα. Κάθε έργο αποτελεί και ένα αρχείο στον υπολογιστή του χρήστη. 

\section{Δημιουργία Κυκλώματος}

\subsection{Δημιουργία έργου}
Εκτελούμε το πρόγραμμα KiCad και από το μενού στο πάνω μέρος της οθόνης επιλέγουμε Αρχείο -> Νέο Έργο -> Νέο Έργο. Δίνουμε στο νέο έργο (και το σχετικό αρχείο) ένα όνομα, έστω usb2uart. Προτείνετε να δημιουργηθεί το αρχείο του έργου σε έναν φάκελο του υπολογιστή σας ο οποίος θα είναι αφιερωμένος στο συγκεκριμένο έργο.

\subsection{Δημιουργία σχηματικού}
Από το μενού στο πάνω μέρος της οθόνης επιλέγουμε Εργαλεία -> Εκτέλεση EEschema. Εάν εμφανιστεί κάποια προειδοποίηση, αποδεχτείτε την (TBD). Θα δημιουργηθούν δύο αρχεία στον υπολογιστή σας, τα usb2uart.sch και usb2uart-cache.lib. Εκτελέστε ξανά την εφαρμογή (TBD).

Θα εμφανιστεί στην οθόνη σας κεντρική σελίδα της εφαρμογής EEschema. Αυτή αποτελείται από ένα φύλλο σχηματικού κυκλώματος στο κέντρο της οθόνης, από τη γραμμή του μενού στο πάνω μέρος της οθόνης, μία μπάρα βασικών λειτουργιών επίσης στο πάνω μέρος, μία μπάρα στο αριστερό μέρος με γενικές λειτουργίες και μία μπάρα στο δεξί μέρος με συγκεκριμένες λειτουργίες.

Από το μενού επιλέξτε Αρχείο -> Ρυθμίσεις Σελίδας. Στο παράθυρο που εμφανίζεται, ορίστε τις ρυθμίσεις της σελίδας σας, όπως ημερομηνία έκδοσης, τίτλος. Αυτές οι ρυθμίσεις δεν έχουν κάποια λειτουργική σημασία για το κύκλωμα.

Από τη δεξιά μπάρα, επιλέξτε το εργαλείο τοποθέτησης γραμμής γραφικών. Σχεδιάστε στο φύλλο μία κατακόρυφη γραμμή που χωρίζει το φύλλο σε δύο τμήματα. Το δεξί τμήμα πρέπει να είναι περίπου το ένα τρίτο του συνολικού φύλλου. Αυτό το τμήμα θα το χρησιμοποιήσουμε για να γράφουμε και να σχεδιάζουμε βοηθητικές πληροφορίες, και να εναποθέτουμε προχωρινά τα εξαρτήματα πριν τα τοποθετήσουμε στην οριστική τους θέση στο κύκλωμα.

Αφαιρέσετε όλες τις βιβλιοθήκες του KiCad εκτός από τις βιβλιοθήκες power, device και conn.

\end{document}
