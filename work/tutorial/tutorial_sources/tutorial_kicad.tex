\documentclass[a4paper]{article}

\usepackage{fontspec}
%για επιλογή γραμματοσειρών (μεταξύ άλλων)
\usepackage{polyglossia}
%για συλλαβισμό λέξεων (μεταξύ άλλων)
\usepackage{geometry}
%για μορφή σελίδας a4 (μεταξύ άλλων)
\usepackage{longtable}
%για προχωρημένους πίνακες
\usepackage{tabu}
%για προχωρημένους πίνακες
\usepackage{hyperref}
\usepackage{comment}

\setdefaultlanguage{greek}
\setotherlanguage{english}

\setmainfont{DejaVu Sans}

\title{Tutorial - Οδηγός Εκμάθησης KiCad}
\author{Μιχάλης Μισιρλής}
\date{Οκτώβριος 2015}

\begin{document}

\maketitle

\pagebreak
\tableofcontents
\pagebreak

\section{Εισαγωγή}
Αυτός ο οδηγός εκμάθησης έχει ως στόχο την εξοικείωση του Έλληνα χρήστη του \textenglish{KiCad} μέσα από την ανάπτυξη ενός απλού αλλά πλήρους κυκλώματος (ηλεκτρονική συσκευή) ξεκινώντας από την βασική ιδέα και το σκοπό της συσκευής και καταλήγοντας στην παραγωγή αρχείων κατασκευής ώστε να μπορεί να παραχθεί η συσκευή από οποιοδήποτε εργοστάσιο ή εταιρεία παραγωγής πλακετών.

\subsection{Προαπαιτούμενα}
\label{sec:prereq}
Αυτός ο οδηγός έχει βασιστεί και απαιτεί το πρόγραμμα \textenglish{KiCad} έκδοση 4.0.0 με την Ελληνική μετάφραση. Επίσης ο οδηγός θεωρεί ότι ο χρήστης είναι εξοικειωμένος με τις βασικές έννοιες ηλεκτρονικού σχεδιασμού και μπορεί να σχεδιάσει πλήρως ένα κύκλωμα στο χαρτί ή σε κάποιο άλλο πρόγραμμα. Αυτός ο οδηγός δεν έχει ως στόχο τη διδασκαλία ηλεκτρονικού σχεδιασμού, αλλά τη διδασκαλία του πως γίνεται ο ηλεκτρονικός σχεδιασμός με το \textenglish{KiCad}.
Τέλος ο οδηγός θεωρεί ότι ο χρήστης είναι εξοικειωμένος τις βασικές έννοιες χρήσης ηλεκτρονικών υπολογιστών, όπως επιλογή μενού, έξοδος από ένα παράθυρο, επιλογή πολλαπλών στοιχείων κα.
%TBD έκδοση kicad
%TBD έκδοση ελλ μετάφρασης

Το tutorial αυτό πρέπει να συνοδεύεται από αρχεία τύπου zip (kicad\_tut01, κοκ) τα οποία περιέχουν τα σχετικά αρχεία του \textenglish{KiCad} για κάθε βήμα του tutorial.

\subsection{Λεξικό όρων}
\begin{longtabu}{ l l }
    Ελληνικός όρος	&	Αγγλικός όρος	\\
    3Δ προβολή	&	3d viewer	\\
    αγκύρωση	&	anchor	\\
    ακροδέκτης	&	pin	\\
    ακτίνα	&	spoke	\\
    ακτίνα	&	spoke	\\
    αλφαριθμητικό	&	string	\\
    ανάλυση	&	parse	\\
    αναφορά, ονομασία αναφοράς	&	reference	\\
    αντεστραμμένο	&	inverted	\\
    αντικολλητικής μάσκας	&	solder resist mask	\\
    αντίσταση συγκόλλησης	&	solder resist	\\
    απόδοση	&	render	\\
    απόκλιση	&	skew	\\
    αποτύπωμα	&	footprint	\\
    γομα	&	eraser	\\
    γραμμή γραφικών	&	graphic line	\\
    διάκενο	&	clearance	\\
    διαμπερές	&	through (hole)	\\
    διάτρηση	&	drill	\\
    δίαυλος	&	bus	\\
    δίκτυα μίας έδρας	&	single pad nets	\\
    δρομολόγηση	&	route	\\
    δρόμος	&	pcb trace / track	\\
    έδρα	&	antipad clearance	\\
    έδρα	&	pad	\\
    εηκ	&	erc	\\
    εκς	&	drc	\\
    ελεγκτής ηλεκτρικών κανόνων	&	electrical rules checker	\\
    εξάρτημα	&	component	\\
    έξοδος	&	output	\\
    επάνω πλευρά	&	top side	\\
    επεξεργαστής σχεδίασης σελίδας	&	page layout editor	\\
    επιλογές	&	options / choices / selections	\\
    επίπεδα	&	layers	\\
    επίπεδα eco	&	eco layers	\\
    επίπεδο σχεδίων	&	drawings layer	\\
    επίπεδοα επικόλλησης	&	adhesive layers	\\
    επισήμανση	&	flag	\\
    ζώνες	&	zones	\\
    θαμμένο via	&	buried via	\\
    θύρα	&	port	\\
    θύρα ισχύος	&	power port	\\
    κανόνες σχεδιασμού	&	design rule	\\
    κατοπτρισμός, καθρέφτισμα	&	mirror	\\
    κάτω πλευρά	&	bottom side	\\
    κενό	&	gap	\\
    κοίλες γωνίες	&	fillet	\\
    κώδικας g	&	g code	\\
    λειτουργία γεμίσματος	&	fil\textenglish{LED} mode	\\
    λειτουργία περιγράμματος	&	sketch mode	\\
    λειτουργία περιγράμματος	&	outline mode	\\
    λίστα δικτύων	&	netlist	\\
    λίστα υλικών	&	bill of materials	\\
    λοξότμηση	&	chamfer	\\
    μάσκα συγκόλλησης	&	solder mask	\\
    μέρη	&	unit	\\
    μεταξοτυπία	&	silk screen	\\
    Μετατόπιση \& σπρώξιμο	&	Push \& shove	\\
    μετατόπιση,θέση	&	offset	\\
    μηχανές ανύψωσης και τοποθέτησης	&	“pick and place”	\\
    μικρο-ρύθμιση, ρύθμιση	&	tune	\\
    μπροστινή όψη	&	front view	\\
    ονοματοδοσία	&	annotate	\\
    όρια πλακέτας	&	pcb/board edge	\\
    παλαιότερου τύπου	&	legacy	\\
    παραπομπή πλαισίου	&	frame reference	\\
    πάστα συγκόλλησης	&	solder paste	\\
    περιγράμμα πλακέτας	&	board outline	\\
    πίνακας	&	matrix	\\
    πίσω όψη	&	back view	\\
    πλαίσιο	&	box (για ui)	\\
    πλοηγός	&	navigator	\\
    πρόγραμμα προβολής pdf	&	pdf viewer	\\
    προκαθορισμένου	&	default	\\
    πρόσθετα	&	plugin	\\
    προτιμώμενο	&	favourite	\\
    ρυθμίσεις	&	settings	\\
    ρυθμίσεις	&	configurations	\\
    σκιαγράφηση	&	hatch	\\
    στελέχη	&	stubs	\\
    στοίβα έδρας	&	padstack	\\
    σύγκρουση, ασυμβατότητα	&	conflict	\\
    συντομεύσεις	&	hotkeys	\\
    συνώνυμο	&	alias	\\
    σύρμα	&	wire	\\
    σύρσιμο	&	drag	\\
    συστοιχία	&	array	\\
    σχεδίαση	&	design	\\
    σχεδίασης σελίδας	&	page layout	\\
    σχέδιο	&	drawing	\\
    σχεδιογραφώ	&	plot	\\
    σχηματικό διάγραμμα, σχηματικό	&	schematic	\\
    τόξο	&	arc	\\
    τρύπα	&	hole	\\
    φάκελος	&	folder / directory	\\
    φωλιά συνδέσεων	&	ratsnest	\\
    χαλκός	&	copper pour	\\
    χώρος αποτυπώματος	&	courtyard	\\
    χώρος σχεδίασης	&	canvas	\\
    ψευδώνυμο	&	nickname	\\
    excellon	&	excellon	\\
    excellon	&	excellon	\\
    gerber	&	gerber	\\
    NPTH	&	NPTH	\\
    PTH	&	PTH	\\
    SMD	&	SMD	\\
    VIA	&	VIA	\\
\end{longtabu}

\subsection{Περιγραφή συσκευής}
Η συσκευή ονομάζεται USB2UART και είναι ένας μετατροπέας USB σε 3.3V UART, βασισμένος στο ολοκληρωμένο \textenglish{CP2104} από την εταιρεία Silicon Labs. υποστηρίζεται από τα περισσότερα σύγχρονα λειτουργικά συστήματα ως εικονική σειριακή θύρα. Συνδέεται με έναν standard type B connector από τη μία πλευρά και ένα four-pin header από την άλλη. Περιλαμβάνει power status \textenglish{LED} και RX/TX \textenglish{LED}.

Το τελικό σχηματικό κύκλωμα της συσκευή και το τελικό σχέδιο της πλακέτας φαίνονται στα σχήματα \ref{fig:eesch-schem-final} και \ref{fig:pcbn-layout-final}.

\begin{figure}
  \begin{center}
    \includegraphics[width=.9\textwidth]{img/eesch-schem-final.png}
    \caption{Τελικό σχηματικό κύκλωμα της συσκευής}
    \label{fig:eesch-schem-final}
  \end{center}
\end{figure}

\begin{figure}
  \begin{center}
    \includegraphics[width=.9\textwidth]{img/pcbn-layout-final.png}
    \caption{Τελικό σχέδιο της πλακέτας της συσκευής}
    \label{fig:pcbn-layout-final}
  \end{center}
\end{figure}

\subsection{Βασικές Έννοιες KiCad}
Το \textenglish{KiCad} είναι μία ολοκληρωμένη σουίτα εφαρμογών σχεδίασης ηλεκτρονικών κυκλωμάτων \textenglish{EDA (Electronic Design Automation)}. Με το \textenglish{KiCad} είναι εφικτή η σχεδίαση σχηματικών και τυπωμένων ηλεκτρονικών κυκλωμάτων, χρησιμοποιώντας διαφορετικές εφαρμογές για κάθε δουλειά.

Τα σχηματικά κυκλώματα σχεδιάζονται στο \textenglish{KiCad} με την εφαρμογή \textenglish{EEschema}. Αποτελούνται από εξαρτήματα (για παράδειγμα ένας πυκνωτής ή ένα ολοκληρωμένο) συνδέσεις μεταξύ τους, και άλλα στοιχεία. Τα σχηματικά κυκλώματα είναι οργανωμένα σε σχηματικά φύλλα. Κάθε σχηματικό φύλλο αποτελεί και ένα αρχείο στον υπολογιστή του χρήστη.

Τα εξαρτήματα των σχηματικών μπορεί να ανήκουν σε μία βιβλιοθήκη του \textenglish{KiCad} ή μπορεί να τα έχει σχεδιάσει από το μηδέν ο χρήστης χρησιμοποιώντας το σχετικό εργαλείο του \textenglish{KiCad}. Κάθε εξάρτημα αποτελεί και ένα αρχείο στον υπολογιστή του χρήστη.

Τα τυπωμένα κυκλώματα σχεδιάζονται στο \textenglish{KiCad} με την εφαρμογή PCBnew. Αποτελούνται από πλακέτες που περιέχουν αποτυπώματα εξαρτημάτων, συνδέσεις μεταξύ τους μέσω δρόμων, και από τρύπες, via, κ.α.
Κάθε πλακέτα αποτελεί και ένα αρχείο στον υπολογιστή του χρήστη. 

Τα αποτυπώματα εξαρτημάτων μπορεί να ανήκουν σε μία βιβλιοθήκη του \textenglish{KiCad} ή μπορεί να τα έχει σχεδιάσει από το μηδέν ο χρήστης χρησιμοποιώντας το σχετικό εργαλείο του \textenglish{KiCad}. Κάθε αποτύπωμα αποτελεί και ένα αρχείο στον υπολογιστή του χρήστη. 

Τα αποτυπώματα μπορεί να έχουν τρισδιάστατη απεικόνιση. Οι τρισδιάστατες απεικονίσεις μπορεί να ανήκουν σε μία βιβλιοθήκη του \textenglish{KiCad} ή μπορεί να τα έχει σχεδιάσει από το μηδέν ο χρήστης χρησιμοποιώντας κάποιο εργαλείο τρισδιάστατης σχεδίασης. Κάθε τρισδιάστατη απεικόνιση αποτελεί και ένα αρχείο στον υπολογιστή του χρήστη. 

Το σύνολο του κυκλώματος (σχηματικό, πλακέτα) οργανώνεται από το \textenglish{KiCad} σε έργα. Κάθε έργο αποτελεί και ένα αρχείο στον υπολογιστή του χρήστη. 

Όλα τα παραπάνω αρχεία (εξαρτήματα, αποτυπώματα, κυκλώματα, έργα, κοκ) καλό είναι να τα διατηρούμε στον ίδιο φάκελο αρχείων του υπολογιστή μας, για λόγους οργάνωσης.

\section{Δημιουργία έργου}
Εκτελέστε το πρόγραμμα \textenglish{KiCad}. Δείτε το σχετικό κεφάλαιο~\ref{sec:prereq} για την έκδοση του \textenglish{KiCad} που πρέπει να εκτελέσετε.

Θα εμφανιστεί η κεντρική οθόνη του προγράμματος \textenglish{KiCad} \ref{kicad-main-window}. 

Αυτή αποτελείται από τα παρακάτω
\begin{itemize}
    \item γραμμή μενού και μπάρα βασικών λειτουργιών στο πάνω μέρος της οθόνης
    \item μπάρα για εκκίνηση βοηθητικών εφαρμογων δεξί μέρος της οθόνης
    \item λίστα με τα αρχεία έργου στο αριστερό μέρος της οθόνης
\end{itemize}

Από το μενού στο πάνω μέρος της οθόνης επιλέγουμε Αρχείο $\rightarrow$ Νέο Έργο $\rightarrow$ Νέο Έργο. Δίνουμε στο νέο έργο (και το σχετικό αρχείο) ένα όνομα, προτείνεται το usb2uart. Προτείνεται να δημιουργηθεί το αρχείο του έργου σε έναν φάκελο του υπολογιστή σας ο οποίος θα είναι αφιερωμένος στο συγκεκριμένο έργο και θα περιέχει όλα τα αρχεία του έργου, και μόνο αυτά.

\begin{figure}
  \begin{center}
    \includegraphics[width=.9\textwidth]{img/kicad-main-window}
    \caption{Κεντρική οθόνη του προγράμματος KiCad}
    \label{fig:kicad-main-window}
  \end{center}
\end{figure}


\section{Σχεδίαση Σχηματικού Κυκλώματος}

\subsection{Δημιουργία σχηματικού}
Από το μενού στο πάνω μέρος της κεντρικής οθόνης του \textenglish{KiCad} επιλέγουμε Εργαλεία $\rightarrow$ Εκτέλεση \textenglish{EEschema}, για να εκτελεστεί η εφαρμογή \textenglish{EEschema} με την οποία σχεδιάζουμε το σχηματικό κύκλωμα. 

Θα δημιουργηθούν δύο αρχεία στον υπολογιστή σας, τα usb2uart.sch και usb2uart-cache.lib. 
%Εάν εμφανιστεί κάποια προειδοποίηση, αποδεχτείτε την (TBD). 
%Εκτελέστε ξανά την εφαρμογή ώστε να μην εμφανιστεί κάποια προειδοποίηση (TBD).

Θα εμφανιστεί στην οθόνη σας η κεντρική οθόνη της εφαρμογής \textenglish{EEschema} \ref{fig:eesch-main-window}. Αυτή αποτελείται από τα παρακάτω
\begin{itemize}
    \item ένα φύλλο σχηματικού κυκλώματος στο κέντρο της οθόνης
    \item γραμμή μενού στο πάνω μέρος της οθόνης
    \item μπάρα βασικών λειτουργιών στο πάνω μέρος της οθόνης
    \item μπάρα με γενικές λειτουργίες στο αριστερό μέρος της οθόνης
    \item μπάρα με συγκεκριμένες λειτουργίες σχεδίασης στο δεξί μέρος της οθόνης
\end{itemize}
Αξίζει να σημειωθεί πως όπως και στις περισσότερες εφαρμογές, έτσι και στο \textenglish{KiCad} και όλες τις εφαρμογές του, οι περισσότερες λειτουργίες μπορούν να γίνουν με δύο τρόπους: είτε επιλέγοντας κάτι στο μενού είτε κάνοντας κλικ στο σχετικό κουμπί μίας μπάρας.

\begin{figure}
  \begin{center}
    \includegraphics[width=.9\textwidth]{img/eesch-main-window.png}
    \caption{Κεντρική οθόνη της εφαρμογής \textenglish{EEschema}}
    \label{fig:eesch-main-window}
  \end{center}
\end{figure}

Το \textenglish{EEschema} μπορεί να σχεδιάσει κυκλώματα που αποτελούνται από πολλά σχηματικά φύλλα. Σε αυτό το tutorial το κύκλωμά μας αποτελείται από ένα και μοναδικό φύλλο.

Από το μενού στο πάνω μέρος της οθόνης του \textenglish{EEschema} επιλέξτε Αρχείο $\rightarrow$ Ρυθμίσεις Σελίδας. Στο παράθυρο που εμφανίζεται \ref{fig:eesch-dial-pagesett}, ορίστε τις ρυθμίσεις της σελίδας σας, όπως μέγεθος χαρτιού, ημερομηνία έκδοσης, τίτλος κυκλώματος, και πατήστε ΟΚ. Αυτές οι ρυθμίσεις δεν έχουν κάποια ηλεκτρική ή λειτουργική σημασία για το κύκλωμα, είναι όμως χρήσιμες πληροφορίες για την οργάνωση των κυκλωμάτων σας.

\begin{figure}
  \begin{center}
    \includegraphics[width=.9\textwidth]{img/eesch-dial-pagesett.png}
    \caption{Ρυθμίσεις Σελίδας \textenglish{EEschema}}
    \label{fig:eesch-dial-pagesett}
  \end{center}
\end{figure}

Στη συνέχεια θα σχεδιάσουμε κάποιες βοηθητικές γραμμές στο σχηματικό μας.

Από το μενού στο πάνω μέρος της οθόνης του \textenglish{EEschema} επιλέξτε "Τοποθέτηση" $\rightarrow$ "Γραφικό Πολυγραμμής".

Με αυτό το εργαλείο μπορείτε να σχεδιάσετε γραμμές στο φύλλο σας. Αυτές οι γραμμές γραφικών δεν έχουν κάποια ηλεκτρική ή λειτουργική σημασία για το κύκλωμα, είναι απλή "ζωγραφική", είναι όμως χρήσιμες για την οργάνωση των κυκλωμάτων σας.

Σχεδιάστε στο φύλλο μία κατακόρυφη γραμμή που χωρίζει το φύλλο σε δύο τμήματα \ref{fig:eesch-main-lines}. Το δεξί τμήμα πρέπει να είναι περίπου το ένα τρίτο του συνολικού φύλλου. Αυτό το δεξί τμήμα θα το χρησιμοποιήσουμε ως χώρο για να γράφουμε  βοηθητικές πληροφορίες (όπως η τρέχουσα έκδοση του κυκλώματος), και για να εναποθέτουμε προχωρινά τα εξαρτήματα πριν τα τοποθετήσουμε στην οριστική τους θέση στο αριστερό τμήμα της σελίδας.

\begin{figure}
  \begin{center}
    \includegraphics[width=.9\textwidth]{img/eesch-main-lines.png}
    \caption{Σχηματικό φύλλο χωρισμένο στα δύο}
    \label{fig:eesch-main-lines}
  \end{center}
\end{figure}

Η εφαρμογή \textenglish{EEschema} περιλαμβάνει πολλά εξαρτήματα, τα οποία μπορείτε να χρησιμοποιήσετε στα κυκλώματά σας. Τα εξαρτήματα αυτά είναι οργανωμένα σε βιβλιοθήκες. Το \textenglish{EEschema} περιλαμβάνει αρχικά περίπου 30 βιβλιοθήκες οι οποίες περιλαμβάνονται εξαρχής σε κάθε νέο έργο που δημιουργείται. Για λόγους απλότητας του έργου μας, εμείς θα αφαιρέσουμε τις βιβλιοθήκες εξαρτημάτων που δεν έχουν εξαρτήματα χρήσιμα για το συγκεκριμένο κύκλωμα που σχεδιάζουμε.

Από το μενού στο πάνω μέρος της οθόνης του \textenglish{EEschema} επιλέξτε "Προτιμήσεις" $\rightarrow$ "Βιβλιοθήκες Εξαρτημάτων". Θα εμφανιστεί ένα παράθυρο με τις βιβλιοθήκες του έργου. Σε αυτό το παράθυρο επιλέξτε όλες τις βιβλιοθήκες (μία προς μία ή όλες μαζί) εκτός από τις βιβλιοθήκες power, device και conn και κάντε κλικ στο "Αφαίρεση". Θα πρέπει στο παράθυρο να φαίνονται μόνο οι τρεις βιβλιοθήκες power, device και conn \ref{fig:eesch-dial-libr}. 

Κάντε κλικ στο "ΟΚ". Θα επανέλθετε στην κεντρική οθόνη της εφαρμογής \textenglish{EEschema}.

Από το μενού στο πάνω μέρος της οθόνης του \textenglish{EEschema} επιλέξτε "Αρχείο" $\rightarrow$ "Αποθήκευση Σχηματικού Έργου".

\begin{figure}
  \begin{center}
    \includegraphics[width=.9\textwidth]{img/eesch-dial-libr.png}
    \caption{Παράθυρο με τις βιβλιοθήκες του έργου}
    \label{fig:eesch-dial-libr}
  \end{center}
\end{figure}

%Σε αυτό το σημείο αντιστοιχούν τα αρχεία από το αρχείο kicad\_tut00.zip.

\subsection{Προσθήκη εξαρτημάτων}
Εφόσον έχουμε δημιουργήσει και ρυθμίσει το φύλλο μας, πρέπει να προσθέσουμε τα εξαρτήματα που θα αποτελέσουν το κύκλωμά μας. Πρέπει να βρισκόμαστε στην κεντρική οθόνη της εφαρμογής \textenglish{EEschema}.

Από το μενού στο πάνω μέρος της οθόνης του \textenglish{EEschema} επιλέξτε "Τοποθέτηση" $\rightarrow$ "Εξάρτημα" και κάντε κλικ στο δεξί τμήμα του φύλλου σας. Θα εμφανιστεί ένα παράθυρο επιλογής εξαρτήματος \ref{fig:eesch-dial-addcomp}.

\begin{figure}
  \begin{center}
    \includegraphics[width=.9\textwidth]{img/eesch-dial-addcomp.png}
    \caption{Παράθυρο επιλογής εξαρτήματος}
    \label{fig:eesch-dial-addcomp}
  \end{center}
\end{figure}

Στο παράθυρο επιλογής εξαρτήματος επιλέξτε το εξάρτημα με όνομα "C" (ένας πυκνωτής) από τη βιβλιοθήκη "device", πατήστε ΟΚ (θα επανέλθετε το φύλλο σχηματικού) και κάντε κλικ στο δεξί τμήμα του φύλλου σας ώστε να τοποθετηθεί στο φύλλο το εξάρτημα που επιλέξατε. 

Αφού τοποθετήσετε το εξάρτημα με όνομα "C" στο φύλλο, ακολουθήστε την ίδια διαδικασία ώστε να τοποθετήσετε στο φύλλο σας (στο δεξί τμήμα) όλα τα παρακάτω εξαρτήματα. 

\begin{itemize}
    \item R, από τη βιβλιοθήκη device
    \item \textenglish{LED}, από τη βιβλιοθήκη device
    \item VCC, από τη βιβλιοθήκη power
    \item +3.3V, από τη βιβλιοθήκη power
    \item GND, από τη βιβλιοθήκη power
    \item CONN\_01X04, από τη βιβλιοθήκη conn
\end{itemize}

Με την ολοκλήρωση αυτών των τοποθετήσεων, έχουμε στο φύλλο μας 7 εξαρτήματα, τοποθετημένα όλα τακτοποιημένα στο δεξί τμήμα του φύλλου \ref{fig:eesch-circ-placedccomp}.

\begin{figure}
  \begin{center}
    \includegraphics[width=.9\textwidth]{img/eesch-circ-placedccomp.png}
    \caption{Φύλλο σχηματικού με τοποθετημένα 7 εξαρτήματα}
    \label{fig:eesch-circ-placedccomp}
  \end{center}
\end{figure}

Για λόγους ευκολίας ανάγνωση του κυκλώματος, θα χρειαστεί να αλλάξουμε το πρόθεμα της ονομασίας αναφοράς του CONN\_01X04 από P σε J. Για να το κάνουμε αυτό πρέπει να κάνουμε δεξί κλικ πάνω στο εξάρτημα CONN\_01X04, και από το μενού που εμφανίζεται να επιλέξουμε "Επεξεργασία Εξαρτήματος" $\rightarrow$ "Επεξεργασία". Αυτό θα μας εμφανίσει το παράθυρο "Ιδιότητες Εξαρτήματος" \ref{fig:eeasch-dial-compprop}, όπου πρέπει να ορίσουμε την Τιμή Πεδίου της Ονομασίας Αναφοράς να είναι J? και όχι P?.

\begin{figure}
  \begin{center}
    \includegraphics[width=.9\textwidth]{img/eeasch-dial-compprop.png}
    \caption{Παράθυρο 'Ιδιότητες Εξαρτήματος'}
    \label{fig:eeasch-dial-compprop}
  \end{center}
\end{figure}


Στη συνέχεια θα χρειαστεί να τοποθετήσουμε στο κύκλωμά μας και άλλα εξαρτήματα C, R και CONN\_01X04. Για να το κάνουμε αυτό, αντί να κάνουμε πάλι τοποθέτηση και να τα επιλέγουμε από τις βιβλιοθήκες, μπορούμε να κάνουμε αντιγραφή των εξαρτημάτων που ήδη έχουμε στο φύλλο μας. 

Για να αντιγράψουμε (δηλαδή να φτιάξουμε ακόμα ένα αντίγραφο) ενός εξαρτήματος πρέπει να κάνουμε δεξί κλικ πάνω στο εξάρτημα, από το μενού που εμφανίζεται να επιλέξουμε "Αντιγραφή", και να κάνουμε κλικ πάνω στο φύλλο εκεί όπου θέλουμε να φιταχτεί το αντίγραφο του εξαρτήματος. 

Με αυτό τον τρόπο πρέπει να φτιάξουμε τα παρακάτω εξαρτήματα.

\begin{itemize}
    \item 3 αντίγραφα του C, ώστε να έχουμε συνολικά 4 C στο φύλλο
    \item 2 αντίγραφα του R, ώστε να έχουμε συνολικά 3 R στο φύλλο
    \item 1 αντίγραφο του CONN\_01X04, ώστε να έχουμε συνολικά 2 CONN\_01X04 στο φύλλο
\end{itemize}

Με την ολοκλήρωση αυτών των τοποθετήσεων έχουμε στο φύλλο μας 13 εξαρτήματα \ref{fig:eesch-circ-placedccompcop}, τοποθετημένα όλα τακτοποιημένα στο δεξί τμήμα του φύλλου.

\begin{figure}
  \begin{center}
    \includegraphics[width=.9\textwidth]{img/eesch-circ-placedccompcop.png}
    \caption{Φύλλο σχηματικού με τοποθετημένα 13 εξαρτήματα}
    \label{fig:eesch-circ-placedccompcop}
  \end{center}
\end{figure}

%Σε αυτό το σημείο αντιστοιχούν τα αρχεία από το αρχείο kicad\_tut01.zip.

\subsection{Δημιουργία εξαρτημάτων}
Το \textenglish{EEschema} μας δίνει τη δυνατότητα να επεξεργαστούμε ένα υπάρχον εξάρτημα ή και να δημιουργήσουμε ένα εξάρτημα από το μηδέν.

Εμείς σε αυτή τη φάση θέλουμε να εντάξουμε στο σχηματικό κύκλωμα το ολοκληρωμένο \textenglish{CP2104}. Το \textenglish{KiCad} όμως δεν έχει στις βιβλιοθήκες του εξάρτημα που να αντιστοιχεί στο ολοκληρωμένο \textenglish{CP2104}, οπότε θα το δημιουργήσουμε.

Από το μενού στο πάνω μέρος της οθόνης του \textenglish{EEschema} επιλέξτε Εργαλεία $\rightarrow$ Επεξεργαστής Βιβλιοθήκης.

Θα εμφανιστεί στην οθόνη σας η κεντρική οθόνη του Επεξεργαστή Βιβλιοθήκης Εξαρτημάτων \ref{fig:libed-main-window}. Αυτή αποτελείται από τα παρακάτω
\begin{itemize}
    \item έναν κενό χώρο στο κέντρο της οθόνης
    \item γραμμή μενού στο πάνω μέρος της οθόνης
    \item μπάρα βασικών λειτουργιών στο πάνω μέρος της οθόνης
    \item μπάρα με γενικές λειτουργίες στο αριστερό μέρος της οθόνης
    \item μπάρα με συγκεκριμένες λειτουργίες σχεδίασης στο δεξί μέρος της οθόνης
\end{itemize}

Στον κενό χώρο στο κέντρο της οθόνης θα σχεδιάσουμε το σώμα του εξαρτήματος, θα προσθέσουμε ακροδέκτες, θα γράψουμε το όνομά του, κλπ. Όλα αυτά τα στοιχεία αφού τα προσθέσουμε στο σώμα του εξαρτήματος, μπορούμε να τα επιλέγουμε με δεξί κλικ του ποντικιού και να τα επεξεργαζόμαστε- κυρίως να τα μετακινούμε και να τα περιστρέφουμε.

\begin{figure}
  \begin{center}
    \includegraphics[width=.9\textwidth]{img/libed-main-window.png}
    \caption{Kεντρική σελίδα - Επεξεργαστής Βιβλιοθήκης Εξαρτημάτων}
    \label{fig:libed-main-window}
  \end{center}
\end{figure}

Αρχικά θα δημιουργήσουμε ένα νέο εξάρτημα.

Κάντε κλικ στο εικονίδιο "Δημιουργία νέου εξαρτήματος" στην πάνω μπάρα %\includegraphics{img/graphic.png}
. Θα εμφανιστεί το παράθυρο "Ιδιότητες Εξαρτήματος" \ref{fig:eesch-dial-compprop}. Στο παράθυρο αυτό, στο "Όνομα εξαρτήματος" γράψτε το όνομα \textenglish{CP2104}, αφήστε όλες τις υπόλοιπες επιλογές στις προκαθορισμένες ρυθμίσεις, και επιλέξτε ΟΚ.

\begin{figure}
  \begin{center}
    \includegraphics[width=.9\textwidth]{img/eesch-dial-compprop.png}
    \caption{Παράθυρο "Ιδιότητες Εξαρτήματος"}
    \label{fig:eesch-dial-compprop}
  \end{center}
\end{figure}

Πλέον έχουμε δημιουργήσει ένα νέο εξάρτημα, χωρίς κανέναν ακροδέκτη. 

Πρέπει να δημιουργήσουμε μία νέα βιβλιοθήκη, να το αποθηκεύσουμε σε αυτήν, και να εντάξουμε τη βιβλιοθήκη στο έργο μας.

Από την μπάρα στο πάνω μέρος της οθόνης επιλέξτε "Αποθήκευση τρέχοντος εξαρτήματος σε νέα βιβλιοθήκη" %\includegraphics{img/graphic.png}.

Αυτό θα εμφανίσει ένα παράθυρο όπου θα πρέπει να δώσετε το όνομα της νέας βιβλιοθήκης που θέλετε να δημιουργήσετε ώστε να μπει σε αυτή το νέο εξάρτημα. Το όνομα της βιβλιοθήκης θα είναι και το όνομα του αρχείου στον υπολογιστή σας, το οποίο θα περιέχει τη βιβλιοθήκη. 

Δώστε στη βιβλιοθήκη το όνομα \textenglish{usb2uart.lib} και πατήστε Save/Αποθήκευση ώστε το αρχείο της βιβλιοθήκης να αποθηκευτεί στον υπολογιστή σας. 

Θα εμφανιστεί ένα μήνυμα που θα λέει ότι η βιβλιοθήκη πρέπει να δηλωθεί στο \textenglish{EEschema} για να χρησιμοποιηθεί. Πατήστε ΟΚ.

Κλείστε τον Επεξεργαστή Βιβλιοθήκης Εξαρτημάτων, εκτελέστε το \textenglish{EEschema} (αν δεν τρέχει ήδη) και από το μενού στο πάνω μέρος της οθόνης του \textenglish{EEschema} επιλέξτε "Προτιμήσεις" $\rightarrow$ "Βιβλιοθήκες Εξαρτημάτων". Θα εμφανιστεί ένα παράθυρο με τις βιβλιοθήκες του έργου. 

Σε αυτό το παράθυρο πατήστε "Προσθήκη", βρείτε στο σύστημα αρχείων του υπολογιστή σας το αρχείο της βιβλιοθήκης που δημιουργήσατε προηγουμένως (\textenglish{usb2uart.lib}), και πατήστε Open. Η βιβλιοθήκη usb2uart θα πρέπει να έχει προστεθεί στις βιβλιοθήκες του έργου σας \ref{fig:libed-dial-libs}. Πατήστε ΟΚ στο παράθυρο με τις βιβλιοθήκες του έργου.

\begin{figure}
  \begin{center}
    \includegraphics[width=.9\textwidth]{img/libed-dial-libs.png}
    \caption{Παράθυρο με τις βιβλιοθήκες του έργου, με την usb2uart}
    \label{fig:libed-dial-libs}
  \end{center}
\end{figure}

Από το μενού στο πάνω μέρος της οθόνης του \textenglish{EEschema} επιλέξτε "Αρχείο" $\rightarrow$ "Αποθήκευση Σχηματικού Έργου".

Από το μενού στο πάνω μέρος της οθόνης του \textenglish{EEschema} επιλέξτε Εργαλεία $\rightarrow$ Επεξεργαστής Βιβλιοθήκης για να συνεχίσετε να δουλεύετε στον Επεξεργαστή Βιβλιοθήκης.

Κάντε κλικ στο εικονίδιο "Επιλογή βιβλιοθήκης εργασίας" στην πάνω μπάρα %\includegraphics{img/graphic.png}
, στο παράθυρο που εμφανίζεται επιλέξτε τη βιβλιοθήκη usb2uart, και πατήστε ΟΚ. Πρέπει να έχετε επιστρέψει στην κεντρική οθόνη του Επεξεργαστή Βιβλιοθήκης.

\begin{figure}
  \begin{center}
    %\includegraphics[width=.9\textwidth]{img/kicad-main.png}
    %\caption{Παράθυρο επιλογής βιβλιοθήκης εργασίας}
    \label{fig:kicad-main}
  \end{center}
\end{figure}

Τώρα πρέπει, στην κεντρική οθόνη του Επεξεργαστή Βιβλιοθήκης, να φορτώσουμε το εξάρτημα το οποίο θέλουμε να επεξεργαστούμε. 

Κάντε κλικ στο εικονίδιο "Φόρτωση εξαρτήματος για επεξεργασία από την τρέχουσα βιβλιοθήκη" %\includegraphics{img/graphic.png}
. Στο παράθυρο επιλογής εξαρτήματος που θα εμφανιστεί επιλέξτε το εξάρτημα με όνομα "\textenglish{CP2104}" από τη βιβλιοθήκη "usb2uart", και πατήστε ΟΚ. Θα επανέλθετε στην κεντρική οθόνη του Επεξεργαστή Βιβλιοθήκης και θα έχει φορτωθεί στην οθόνη το εξάρτημα "\textenglish{CP2104}".

Στη συνέχεια θα ορίσουμε κάποιες από τις γενικές ιδιότητες του εξαρτήματος. 

Κάντε κλικ στο εικονίδιο "Επεξεργασία ιδιοτήτων εξαρτήματος" %\includegraphics{img/graphic.png} 
για να εμφανιστεί το παράθυρο Ιδιοτήτων για το εξάρτημα. 

\begin{figure}
  \begin{center}
    %\includegraphics[width=.9\textwidth]{img/kicad-main.png}
    %\caption{Παράθυρο Ιδιοτήτων για το εξάρτημα}
    \label{fig:kicad-main}
  \end{center}
\end{figure}

Σε αυτό το παράθυρο συμπληρώστε τα παρακάτω πεδία και μετά πατήστε ΟΚ για να επιστρέψετε στην κεντρική οθόνη του Επεξεργαστή Βιβλιοθήκης.

\begin{itemize}
    \item Περιγραφή: \textenglish{CP2104} Μετατροπέας USB-σε-UART
    \item Φίλτρο αποτυπώματος: QFN
    %\item docfile
\end{itemize}

Στην κεντρική οθόνη του Επεξεργαστή Βιβλιοθήκης πρέπει να σχεδιάσουμε αρχικά το σώμα του εξαρτήματος. 

Από το μενού στο πάνω μέρος της οθόνης επιλέξετε Τοποθέτηση $\rightarrow$ Ορθογώνιο και σχεδιάστε στην οθόνη ένα κατακόρυφο ορθογώνιο όπως φαίνεται στη σχετική εικόνα. Για να σχεδιάσετε, κάντε κλικ εκεί που θέλετε να είναι η πάνω αριστερή γωνία και μετά ένα ακόμα κλικ εκεί που θέλετε να είναι η κάτω δεξιά γωνία του σχεδίου.

Στο εξάρτημα πρέπει να εμφανίζονται το όνομά του (\textenglish{CP2104}) και η ονομασία αναφοράς (U?). Κάντε δεξί κλικ επάνω στο όνομα, επιλέξτε μετακίνηση, και τοποθετήστε το όνομα κάτω από το σώμα του εξαρτήματος. Επίσης κάντε δεξί κλικ επάνω στην ονομασία αναφοράς, επιλέξτε μετακίνηση, και τοποθετήστε την πάνω από το σώμα του εξαρτήματος. 

\begin{figure}
  \begin{center}
    %\includegraphics[width=.9\textwidth]{img/kicad-main.png}
    %\caption{Ορθογώνιο σώμα εξαρτήματος, με τοποθετημένα όνομα και αναφορά}
    \label{fig:kicad-main}
  \end{center}
\end{figure}

Να σημειωθεί ότι έως τώρα έχουμε ορίσει μόνο βοηθητικά στοιχεία του εξαρτήματος: το όνομά του, τι φίλτρο αποτυπώματος θα έχει, πως θα εμφανίζεται στο σχηματικό, κα. Τίποτα από αυτά δεν έχει ηλεκτρική/λειτουργική σημασία για το εξάρτημα.

Τώρα πρέπει να προσθέσουμε τους ακροδέκτες που θα αποτελούν το εξάρτημά μας. Το πόσους και τι είδους ακροδέκτες πρέπει να προσθέσουμε θα το γνωρίζουμε από το εγχειρίδιο/φύλλο δεδομένων (data sheet) του ολοκληρωμένου. Αυτό μπορείτε να το βρείτε στην ιστοσελίδα του κατασκευαστή του ολοκληρωμένου. 

Κατά τη συγγραφή αυτού του οδηγού εκμάθησης το εγχειρίδιο του \textenglish{CP2104} (έκδοση/Rev. 1.1) βρισκόταν στον παρακάτω σύνδεσμο:

\href{https://www.silabs.com/Support\%20Documents/TechnicalDocs/cp2104.pdf}{https://www.silabs.com/Support\%20Documents/TechnicalDocs/cp2104.pdf}. 

Αν δεν είναι διαθέσιμο σε αυτό τον σύνδεσμο επισκεφθείτε τη σελίδα του κατασκευαστή και αναζητήστε εκεί το εγχειρίδιο του \textenglish{CP2104}: 

\href{https://www.silabs.com/}{https://www.silabs.com/}

Με βάση το εγχειρίδιο του \textenglish{CP2104}, πρέπει να ορίσουμε 25 ακροδέκτες, όπως φαίνεται και στη σχετική εικόνα. Εκτός από τους 24 τυπικούς ακροδέκτες θα έχουμε ως ακροδέκτη και την θερμική έδρα GND του \textenglish{CP2104}. Οπότε σύνολικά θα έχουμε 24 ακροδέκτες για το \textenglish{CP2104}.

\begin{figure}
  \begin{center}
    %\includegraphics[width=.9\textwidth]{img/kicad-main.png}
    %\caption{Ακροδέκτες του \textenglish{CP2104}}
    \label{fig:kicad-main}
  \end{center}
\end{figure}

Σε αυτό το tutorial θα ορίσουμε 2 ακροδέκτες μόνο, και οι υπόλοιποι πρέπει να προστεθούν από εσάς, κατά τον ίδιο τρόπο όπως και οι 2 πρώτοι.

Αρχικά ας προσθέσουμε τον ακροδέκτη Vio, ο οποίος σύμφωνα με το εγχειρίδιο έχει το όνομα Vio, του έχει αποδοθεί ο αριθμός 5, και η λειτουργία του είναι είσοδος ισχύος τροφοδοσίας.

Από το μενού στο πάνω μέρος της οθόνης επιλέξετε Τοποθέτηση $\rightarrow$ Ακροδέκτης και κάντε κλικ στην κεντρική οθόνη για να εμφανιστεί το παράθυρο Ιδιότητες Ακροδέκτη.

\begin{figure}
  \begin{center}
    %\includegraphics[width=.9\textwidth]{img/kicad-main.png}
    %\caption{Παράθυρο Ιδιότητες Ακροδέκτη}
    \label{fig:kicad-main}
  \end{center}
\end{figure}

Σε αυτό το παράθυρο πρέπει να ορίσουμε τις παρακάτω σημαντικές ιδιότητες για τον ακροδέκτη, και μετά να το τοποθετήσουμε στο σώμα του εξαρτήματος.

\begin{itemize}
    \item Όνομα ακροδέκτη: Vio
    \item Αριθμός ακροδέκτη: 5
    \item Προσανατολισμός: Αριστερά
    \item Ηλεκτρικός Τύπος: Είσοδος ισχύος
\end{itemize}

Δεν έχει ηλεκτρική σημασία σε ποιο σημείο του σώματος θα τοποθετήσουμε τον ακροδέκτη (πάνω αριστερά, στη μέση κλπ) ή τι προσανατολισμό θα του δώσουμε. Αυτά είναι δευτερεύοντα χαρακτηριστικά, τα οποία μας βοηθούν στην απεικόνιση του σχηματικού. Συνήθως τους ακροδέκτες που έχουν συναφή λειτουργία τους σχεδιάζουμε κοντά τον έναν στον άλλο.

Το όνομα, ο αριθμός, και ο τύπος του ακροδέκτη έχουν ηλεκτρική σημασία καθώς λαμβάνονται υπόψιν κατά τον Έλεγχο Ηλεκτρικών Κανόνων που θα κάνουμε με το \textenglish{KiCad} αργότερα.

\begin{figure}
  \begin{center}
    %\includegraphics[width=.9\textwidth]{img/kicad-main.png}
    %\caption{Σώμα εξαρτήματος, με τοποθετημένο έναν ακροδέκτη}
    \label{fig:kicad-main}
  \end{center}
\end{figure}

Στη συνέχεια προσθέσετε τον ακροδέκτη D+, ο οποίος σύμφωνα με το εγχειρίδιο έχει το όνομα D+, του έχει αποδοθεί ο αριθμός 3, και η λειτουργία του είναι είσοδος/έξοδος δεδομένων.

Όπως και πριν, από το μενού στο πάνω μέρος της οθόνης επιλέξετε Τοποθέτηση $\rightarrow$ Ακροδέκτης και κάντε κλικ στην κεντρική οθόνη για να εμφανιστεί το παράθυρο Ιδιότητες Ακροδέκτη.

Σε αυτό το παράθυρο πρέπει να ορίσουμε τις παρακάτω σημαντικές ιδιότητες για τον ακροδέκτη, και μετά να τον τοποθετήσουμε στο σώμα του εξαρτήματος.

\begin{itemize}
    \item Όνομα ακροδέκτη: D+
    \item Αριθμός ακροδέκτη: 3
    \item Προσανατολισμός: Δεξιά
    \item Ηλεκτρικός Τύπος: Αμφίδρομο
\end{itemize}

\begin{figure}
  \begin{center}
    %\includegraphics[width=.9\textwidth]{img/kicad-main.png}
    %\caption{Σώμα εξαρτήματος, με τοποθετημένους δύο ακροδέκτες}
    \label{fig:kicad-main}
  \end{center}
\end{figure}

Συνεχίζοντας κατά τον ίδιο τρόπο, τοποθετήστε και τους υπόλοιπους 23 ακροδέκτες του εξαρτήματος, ώστε να καταλήξετε τελικά στο εξάρτημα όπως αυτό φαίνεται στη σχετική εικόνα.

\begin{figure}
  \begin{center}
    %\includegraphics[width=.9\textwidth]{img/kicad-main.png}
    %\caption{Σώμα εξαρτήματος, με τοποθετημένους όλους τους ακροδέκτες}
    \label{fig:kicad-main}
  \end{center}
\end{figure}

Στο εξάρτημα, εκτός από το σχεδιασμό του περιγράμματος και την τοποθέτηση των ακροδεκτών, μπορούμε να τοποθετήσουμε και άλλες οπτικές πληροφορίες, όπως κείμενο και επιπλέον γραμμές γραφικών που να προσφέρουν βοηθητικές πληροφορίες.

Για να προσθέσουμε κείμενο, επιλέγουμε από τη δεξιά μπάρα το το εικονίδιο "Προσθήκη κειμένου στο σώμα εξαρτήματος" %\includegraphics{img/graphic.png}
, κάνουμε κλικ στο σώμα του εξαρτήματος, και στο παράθυρο "Ιδιότητες Κειμένου Βιβλιοθήκης" που εμφανίζεται γράφουμε ό,τι κείμενο θέλουμε, επιλέγουμε αν θα εμφανίζεται κάθετα, τι στοίχιση θα έχει, κοκ.

\begin{figure}
  \begin{center}
    %\includegraphics[width=.9\textwidth]{img/kicad-main.png}
    %\caption{Παράθυρο "Ιδιότητες Κειμένου Βιβλιοθήκης"}
    \label{fig:kicad-main}
  \end{center}
\end{figure}

Για παράδειγμα, στο \textenglish{CP2104} μπορούμε να βάλουμε κατακόρυφα δίπλα στους ακροδέκτες D+ και D- τη λέξη USB, το οποίο θα μας βοηθάει να θυμόμαστε ότι αυτή είναι η λειτουργία τους.

Επίσης, μπορούμε να βάλουμε κατακόρυφα δίπλα στους ακροδέκτες RTX, TXD, DTR τη λέξη UARTOUT, το οποίο θα μας βοηθάει να θυμόμαστε ότι αυτή είναι η λειτουργία τους.

Ομοίως και για τους ακροδέκτες DCD, RI, CTS, RXD, DSR τη λέξη UARTΙΝ.

Αφού έχουμε ορίσει όλους τους ακροδέκτες, καλό θα είναι να αποθηκεύσουμε το εξάρτημα στη βιβλιοθήκη. Από το μενού στο πάνω μέρος της οθόνης επιλέξτε Αρχείο $\rightarrow$ Αποθήκευση της τρέχουσας βιβλιοθήκης. Θα εμφανιστούν δύο παράθυρα που θα ζητούν την επιβεβαίωση αυτή της αποθήκευσης. Επιλέξτε Yes/Ναι και στα δύο.

Πλέον έχουμε ολοκληρώσει τη δημιουργία του εξαρτήματος.

%Σε αυτό το σημείο αντιστοιχούν τα αρχεία από το αρχείο kicad\_tut02.zip.

\subsection{Επεξεργασία εξαρτημάτων με πολλά μέρη}

Σε αυτό το κεφάλαιο θα φορτώσουμε ένα εξάρτημα (ένα \textenglish{LED}) από μία εσωτερική βιβλιοθήκη του \textenglish{KiCad}, θα το επεξεργαστούμε κάνοντάς το να αποτελείται από πολλά μέρη, και θα το αποθηκεύσουμε (αλλαγμένο) σε μία δική μας βιβλιοθήκη.

Όταν ένα εξάρτημα αποτελείται από πολλά "μέρη" στο \textenglish{KiCad}, αυτό σημαίνει ότι το εξάρτημα είναι ένα φυσικό αντικείμενο (πχ ένα ολοκληρωμένο) το οποίο όμως αποτελείται εσωτερικά από πολλά ίδια μέρη. Στην περίπτωσή μας, εμείς θέλουμε ένα εξάρτημα το οποίο αποτελείται από δύο \textenglish{LED}.

Δεν υπάρχει εξάρτημα το οποίο αποτελείται από δύο \textenglish{LED} στις γνωστές βιβλιοθήκες του \textenglish{KiCad}. Θα πρέπει να το σχεδιάσουμε εμείς.

Θα μπορούσαμε να δημιουργήσουμε από το μηδέν ένα νέο τέτοιο εξάρτημα, στον Επεξεργαστή Βιβλιοθήκης Εξαρτημάτων. 

Είναι όμως καλύτερα να χρησιμοποιήσουμε το υπάρχον εξάρτημα \textenglish{LED}, και να σχεδιάσουμε ένα νέο εξάρτημα το οποίο θα αποτελείται από δύο εξαρτήματα \textenglish{LED}.

\subsubsection{Άνοιγμα του εξαρτήματος}
Θεωρούμε ότι είμαστε στην αρχική οθόνη του \textenglish{KiCad}. 

Από το μενού στο πάνω μέρος της οθόνης επιλέξτε Εργαλεία $\rightarrow$ Εκτέλεση Επεξεργαστή Βιβλιοθήκης, ώστε να φορτωθεί η κεντρική οθόνη του Επεξεργαστή Βιβλιοθήκης Εξαρτημάτων.

\begin{figure}
  \begin{center}
    %\includegraphics[width=.9\textwidth]{img/kicad-main.png}
    %\caption{Κεντρική οθόνη του Επεξεργαστή Βιβλιοθήκης Εξαρτημάτων}
    \label{fig:kicad-main}
  \end{center}
\end{figure}

Από την πάνω μπάρα, επιλέξτε το κουμπί "Επιλογή βιβλιοθήκης εργασίας" και στο παράθυρο που θα εμφανιστεί κάντε κλικ πάνω στη βιβλιοθήκη device και πατήστε ΟΚ για να φορτωθεί η βιβλιοθήκη και να επιστρέψετε στην κεντρική οθόνη του Επεξεργαστή Βιβλιοθήκης Εξαρτημάτων.

Από την κεντρική οθόνη του Επεξεργαστή Βιβλιοθήκης Εξαρτημάτων κάντε κλικ στο εικονίδιο "Φόρτωση εξαρτήματος για επεξεργασία από την τρέχουσα βιβλιοθήκη" %\includegraphics{img/graphic.png}
. Το παράθυρο που θα εμφανιστεί εμφανίζει όλα τα εξαρτήματα της βιβλιοθήκης \textenglish{LED}. Βρείτε στη λίστα το εξάρτημα \textenglish{LED}, επιλέξτε το, και πατήστε ΟΚ για να φορτωθεί στην κεντρική οθόνη του Επεξεργαστή Βιβλιοθήκης Εξαρτημάτων.

\begin{figure}
  \begin{center}
    %\includegraphics[width=.9\textwidth]{img/kicad-main.png}
    %\caption{Κεντρική οθόνη του Επεξεργαστή Βιβλιοθήκης Εξαρτημάτων με φορτωμένο το εξάρτημα \textenglish{LED}}
    \label{fig:kicad-main}
  \end{center}
\end{figure}

Τώρα, αφού έχουμε φορτώσει το εξάρτημα, θα φτιάξουμε ένα αντίγραφό του, και το αντίγραφο θα το αλλάξουμε ώστε να καλύπτει τις ανάγκες μας.

\subsubsection{Δημιουργία αντίγραφου εξαρτήματος}

Από την κεντρική οθόνη του Επεξεργαστή Βιβλιοθήκης Εξαρτημάτων κάντε κλικ στο εικονίδιο "Δημιουργία νέου εξαρτήματος από το τρέχον" %\includegraphics{img/graphic.png}
. Θα εμφανιστεί το παράθυρο "Όνομα Εξαρτήματος". Ονομάστε το νέο εξάρτημα b\textenglish{LED} και πατήστε ΟΚ.

\begin{figure}
  \begin{center}
    %\includegraphics[width=.9\textwidth]{img/kicad-main.png}
    %\caption{Παράθυρο "Όνομα Εξαρτήματος"}
    \label{fig:kicad-main}
  \end{center}
\end{figure}

Πλέον έχουμε δημιουργήσει ένα νέο εξάρτημα, το οποίο καλό είναι να αποθηκεύσουμε. Αλλά βρισκόμαστε στη βιβλιοθήκη device, η οποία επειδή είναι εσωτερική βιβλιοθήκη του \textenglish{KiCad} δεν μπορούμε να γράψουμε σε αυτήν. Πρέπει να αποθηκεύσουμε το νέο εξάρτημα σε άλλη βιβλιοθήκη.

Από την πάνω μπάρα, επιλέξτε το κουμπί "Επιλογή βιβλιοθήκης εργασίας" %\includegraphics{img/graphic.png}
 και στο παράθυρο που θα εμφανιστεί κάντε κλικ πάνω στη βιβλιοθήκη uart και πατήστε ΟΚ για να φορτωθεί η βιβλιοθήκη και να επιστρέψετε στην κεντρική οθόνη του Επεξεργαστή Βιβλιοθήκης Εξαρτημάτων.

Από το μενού στο πάνω μέρος της οθόνης επιλέξτε Αρχείο $\rightarrow$ Αποθήκευση της τρέχουσας βιβλιοθήκης. Θα εμφανιστούν δύο παράθυρα που θα ζητούν την επιβεβαίωση αυτή της αποθήκευσης. Επιλέξτε Yes/Ναι και στα δύο.

Πλέον έχουμε αποθηκεύσει το εξάρτημα που φτιάξαμε (αντιγράφοντας ένα υπάρχον) στη δική μας βιβλιοθήκη.

\subsubsection{Αλλαγή ιδιοτήτων εξαρτήματος}
Κάντε κλικ στο εικονίδιο "Επεξεργασία ιδιοτήτων εξαρτήματος" %\includegraphics{img/graphic.png} 
για να εμφανιστεί το παράθυρο Ιδιοτήτων για το εξάρτημα. 

\begin{figure}
  \begin{center}
    %\includegraphics[width=.9\textwidth]{img/kicad-main.png}
    %\caption{Παράθυρο Ιδιοτήτων Εξαρτήματος}
    \label{fig:kicad-main}
  \end{center}
\end{figure}

Αλλάξτε τον αριθμό των μερών από ένα σε 2, και πατήστε ΟΚ για να επιστρέψετε στην κεντρική οθόνη του Επεξεργαστή Βιβλιοθήκης Εξαρτημάτων.

Πλέον το εξάρτημά μας αποτελείται από δύο μέρη. Ποιο μέρος του εξαρτήματος (μέρος Α ή μέρος Β) επεξεργάζεστε ανά πάσα στιγμή φαίνεται στο πάνω δεξιά μέρος της κεντρικής οθόνης του Επεξεργαστή Βιβλιοθήκης Εξαρτημάτων.

\begin{figure}
  \begin{center}
    %\includegraphics[width=.9\textwidth]{img/kicad-main.png}
    %\caption{Ένδειξη μέρους εξαρτήματος}
    \label{fig:kicad-main}
  \end{center}
\end{figure}

Αφού πλέον το εξάρτημα b\textenglish{LED} αποτελείται από δύο μέρη, πρέπει να αριθμήσουμε τους ακροδέκτες του μέρους Α ως 1 και 3 και τους ακροδέκτες του μέρους Β ως 2 και 4.

Για να το κάνουμε αυτό, επιλέγουμε στην "Ένδειξη μέρους εξαρτήματος" το "Μέρος A", και φροντίζουμε οι δύο ακροδέκτες να έχουν αριθμό 1 (άνοδος) και 3 (κάθοδος). Μπορούμε να αλλάξουμε τον αριθμό ενός ακροδέκτη κάνοντας δεξί κλικ επάνω στον ακροδέκτη και επιλέγοντας "Επεξεργασία Ακροδέκτη".

Παρομοίως, επιλέγουμε στην "Ένδειξη μέρους εξαρτήματος" το "Μέρος B", και φροντίζουμε οι δύο ακροδέκτες να έχουν αριθμό 2 (άνοδος) και 4 (κάθοδος). Μπορούμε να αλλάξουμε τον αριθμό ενός ακροδέκτη κάνοντας δεξί κλικ επάνω στον ακροδέκτη και επιλέγοντας "Επεξεργασία Ακροδέκτη".

\begin{figure}
  \begin{center}
    %\includegraphics[width=.9\textwidth]{img/kicad-main.png}
    %\caption{Εξάρτημα b\textenglish{LED} - Μέρος Α}
    \label{fig:kicad-main}
  \end{center}
\end{figure}

\begin{figure}
  \begin{center}
    %\includegraphics[width=.9\textwidth]{img/kicad-main.png}
    %\caption{Εξάρτημα b\textenglish{LED} - Μέρος Β}
    \label{fig:kicad-main}
  \end{center}
\end{figure}

Αφού έχετε ορίσει τους αριθμούς των ακροδεκτών, δεν μένει παρά να αποθηκεύσουμε το εξάρτημα.

Από το μενού στο πάνω μέρος της οθόνης επιλέξτε Αρχείο $\rightarrow$ Αποθήκευση της τρέχουσας βιβλιοθήκης. Θα εμφανιστούν δύο παράθυρα που θα ζητούν την επιβεβαίωση αυτή της αποθήκευσης. Επιλέξτε Yes/Ναι και στα δύο.

Kλείστε τον Επεξεργαστή Βιβλιοθήκης Εξαρτημάτων ώστε να επιστρέψετε στην αρχική οθόνη του \textenglish{KiCad}.

%Σε αυτό το σημείο αντιστοιχούν τα αρχεία από το αρχείο kicad\_tut03.zip.

\subsection{Σχεδίαση κυκλώματος}
Εφόσον έχουμε ετοιμάσει όλα τα εξαρτήματα που θα χρειαστούμε, πρέπει να σχεδιάσουμε το κύκλωμα. 

Βεβαιωθείτε ότι είστε στην αρχική οθόνη του \textenglish{KiCad} και έχετε φορτώσει το έργο usb2uart.

Από το μενού στο πάνω μέρος της οθόνης επιλέγουμε Εργαλεία $\rightarrow$ Εκτέλεση \textenglish{EEschema}, για να εκτελεστεί η εφαρμογή \textenglish{EEschema} με την οποία σχεδιάζουμε το σχηματικό κύκλωμα. 

\subsubsection{Προσθήκη εξαρτημάτων}

Πρέπει να βρίσκεστε στην αρχική οθόνη της εφαρμογής \textenglish{EEschema}, και να είναι φορτωμένο το φύλλο του σχηματικού κυκλώματος του έργου usb2uart.

\begin{figure}
  \begin{center}
    %\includegraphics[width=.9\textwidth]{img/kicad-main.png}
    %\caption{Αρχική οθόνη της εφαρμογής \textenglish{EEschema}, με φορτωμένο το σχηματικό κύκλωμα του έργου usb2uart}
    \label{fig:kicad-main}
  \end{center}
\end{figure}

Αρχικά πρέπει να προσθέσουμε στο κύκλωμα τα εξαρτήματα \textenglish{CP2104} και b\textenglish{LED}. 

Όπως κάναμε και σε προηγούμενα κεφάλαια, επιλέξτε από το μενού Τοποθέτηση $\rightarrow$ Εξάρτημα, επιλέξτε το \textenglish{CP2104} από τη βιβλιοθήκη usb2uart, και τοποθετήστε το στο κύκλωμα. Στη συνέχεια κάντε το ίδιο για να τοποθετήσετε το εξάρτημα b\textenglish{LED} (και τα δύο μέρη), από την ίδια βιβλιοθήκη. Το εξάρτημα b\textenglish{LED} αποτελείται από δύο μέρη, οπότε πρέπει να προσθέσετε πρώτα το Μέρος Α και μετά το μέρος Β.

Να σημειωθεί ότι αντί να προσθέσουμε δύο φορές το b\textenglish{LED} (μία το Μέρος Α, και μία το Μέρος Β), θα μπορούσαμε να το προσθέσουμε μία φορά, μετά να κάναμε αντιγραφή του (δεξί κλικ στο εξάρτημα $\rightarrow$ Αντιγραφή Εξαρτήματος), και μετά να αλλάζαμε στις ιδιότητες του αντίγραφου το Μέρος. Το αποτέλεσμα θα ήταν το ίδιο.

Πλέον πρέπει να έχουμε φορτώσει στο φύλλο όλα τα εξαρτήματα που χρειαζόμαστε για το κύκλωμά μας.

\begin{figure}
  \begin{center}
    %\includegraphics[width=.9\textwidth]{img/kicad-main.png}
    %\caption{Φύλλο σχηματικού κυκλώματος έργου usb2uart με όλα τα εξαρτήματα που χρειαζόμαστε}
    \label{fig:kicad-main}
  \end{center}
\end{figure}

%Σε αυτό το σημείο αντιστοιχούν τα αρχεία από το αρχείο kicad\_tut04.zip.

\subsubsection{Σχεδίαση κυκλώματος}

Πλέον αφού έχουμε όλα τα απαραίτητα εξαρτήματα στο φύλλο μας, πρέπει να τα τοποθετήσουμε στο φύλλο με τρόπο που να μας βοηθάει στην κατανόηση του κυκλώματος, και (κυρίως) να κάνουμε όλες τις συνδέσεις μεταξύ τους.

Κάποιες βασικές λειτουργίες για τη σχεδίαση του κυκλώματος είναι οι παρακάτω.

\paragraph{Μετακίνηση εξαρτήματος}
Μετατοπίζουμε το εξάρτημα όπου θέλουμε πάνω στο φύλλο του κυκλώματος. Η θέση αυτή δεν έχει κάποια ηλεκτρική σημασία, βοηθάει όμως στην καλύτερη οργάνωση του κυκλώματος. Για παράδειγμα το μοναδικό ολοκληρωμένο ενός κυκλώματος καλό θα ήταν να είναι στο μέσο του φύλλου ώστε να υπάρχει χώρος να τοποθετηθούν και άλλα εξαρτήματα γύρω του.

Για να μετακινήσουμε ένα εξάρτημα κάνουμε κλικ επάνω του, από το μενού που εμφανίζεται επιλέγουμε "Μετακίνηση Εξαρτήματος", και μετά το τοποθετούμε όπου θέλουμε στο φύλλο. 

Εναλλακτικά μπορούμε να πάμε τον δείκτη του ποντικιού επάνω από το εξάρτημα (χωρίς να κάνουμε κλικ) και να πατήσουμε το πλήκτρο m.

\paragraph{Περιστροφή εξαρτήματος}
Περιστρέφουμε το εξάρτημα πάνω στο φύλλο του κυκλώματος. Η θέση περιστροφής δεν έχει κάποια ηλεκτρική σημασία, βοηθάει όμως στην καλύτερη οργάνωση του κυκλώματος. Για παράδειγμα μία αντίσταση μπορεί να προτιμάμε να εμφανίζεται οριζόντια για να διαβάζεται πιο εύκολα το όνομά της.

Για να περιστρέψουμε ένα εξάρτημα κάνουμε δεξί κλικ επάνω του, και από το μενού που εμφανίζεται επιλέγουμε "Προσανατολισμός Εξαρτήματος" $\rightarrow$ "Περιστροφή Αριστερόστροφα" (ή "Περιστροφή Δεξιόστροφα"). 

Εναλλακτικά μπορούμε να πάμε τον δείκτη του ποντικιού επάνω από το εξάρτημα (χωρίς να κάνουμε κλικ) και να πατήσουμε το πλήκτρο r.

\paragraph{Τοποθέτηση σύρματος}
Είναι από τις πιο σημαντικές λειτουργίες, καθώς με αυτή συνδέουμε αγώγιμα τα εξαρτήματα μεταξύ ενώνοντάς τα με σύρμα.

Για να ενώσουμε αγώγιμα δύο σημεία του κυκλώματος (πχ τα άκρα δύο αντιστάσεων) από το μενού στο πάνω μέρος της οθόνης επιλέγουμε "Τοποθέτηση" $\rightarrow$ "Σύρμα", μετά κάνουμε αριστερό κλικ με το ποντίκι πάνω στο κύκλωμα μία φορά για να ξεκινήσει το σύρμα, μετά κάνουμε κλικ όσες φορές θέλουμε για να ορίσουμε τη διαδρομή, και όταν έχουμε τοποθετήσει όσο σύρμα θέλουμε κάνουμε δεξί κλικ και επιλέγουμε "Τέλος Σύρματος". 

Εναλλακτικά μπορούμε να επιλέξουμε το κουμπί "Τοποθέτηση σύρματος" %\includegraphics{img/graphic.png} 
στην αριστερή μπάρα, και μετά να πατήσουμε το πλήκτρο w για να αρχίσουμε να τοποθετούμε σύρμα και το πλήκτρο k για ορίσουμε το τέλος του σύρματος.

Αξίζει να σημειωθεί ότι όταν συνδέουμε ένα σύρμα με ένα άλλο σύρμα, το \textenglish{KiCad} εμφανίζει ένα σύμβολο κόμβου %\includegraphics{img/graphic.png}
. Όταν θέλουμε να συνδέσουμε δύο σύρματα σε έναν ακροδέκτη, συνήθως συνδέουμε το ένα σύρμα στον ακροδέκτη και το δεύτερο σύρμα το συνδέουμε πάνω πάνω στο πρώτο σύρμα.

Συνηθίζεται τα σύρματα να έχουν μόνο κατακόρυφη και οριζόντια φορά (να μην τοποθετούνται διαγώνια δύρματα), διότι έτσι παράγονται πιο ευανάγνωστα κυκλώματα.

\begin{figure}
  \begin{center}
    %\includegraphics[width=.9\textwidth]{img/kicad-main.png}
    %\caption{Δυο εξαρτήματα (αντιστάσεις) χωρίς σύρμα να ενώνει τα άκρα τους}
    \label{fig:kicad-main}
  \end{center}
\end{figure}

\begin{figure}
  \begin{center}
    %\includegraphics[width=.9\textwidth]{img/kicad-main.png}
    %\caption{Δυο εξαρτήματα (αντιστάσεις) με σύρμα να ενώνει τα άκρα τους}
    \label{fig:kicad-main}
  \end{center}
\end{figure}
   
\begin{figure}
  \begin{center}
    %\includegraphics[width=.9\textwidth]{img/kicad-main.png}
    %\caption{Δύο σύρματα συνδεδεμένα με τον ίδιο ακροδέκτη}
    \label{fig:kicad-main}
  \end{center}
\end{figure}

\paragraph{Τοποθέτηση ετικέτας}
Μία ετικέτα είναι ένα αντικείμενο του κυκλώματος που συμβολίζει έναν κόμβο, ένα σημείο σύνδεσης. Αν σχεδιάσουμε πχ ένα σύρμα από το εξάρτημα Α προς μία ετικέτα Χ και μετά ένα σύρμα από το εξάρτημα Β προς τη ίδια ετικέτα Χ, τότε τα δύο εξαρτήματα θα είναι συνδεδεμένα μεταξύ τους. Η συνδεσμολογία όπου δύο εξαρτήματα είναι συνδεδεμένα μεταξύ τους απευθείας είναι απόλυτα ισοδύναμη με τη συνδεσμολογία όπου δύο εξαρτήματα είναι συνδεδεμένα μεταξύ τους μέσω ετικέτας.

Για να τοποθετήσουμε μία ετικέτα, από το μενού στο πάνω μέρος της οθόνης επιλέγουμε "Τοποθέτηση" $\rightarrow$ "Ετικέτα", μετά κάνουμε κλικ με το ποντίκι πάνω στο κύκλωμα, στο παράθυρο που εμφανίζεται δίνουμε ένα όνομα στην ετικέτα και πατάμε ΟΚ για να τοποθετήσουμε την ετικέτα πάνω στο κύκλωμα. Στη συνέχεια πρέπει να μετακινήσουμε την ετικέτα ώστε το κουτάκι που υπάρχει κάτω αριστερά από το όνομά της να συμπέσει με το κουτάκι που υπάρχει στο τέλος ενός ασύνδετου σύρματος. Εφόσον αυτά τα δύο κουτάκια είναι στο ίδιο σημείο (συμπίπτουν), πλέον το σύρμα έχει αυτή την ετικέτα, και θεωρείται συνδεδεμένο με όλα τα άλλα σύρματα που έχουν ετικέτα με το ίδιο όνομα. Λέμε ότι όλα αυτά τα σύρματα ανήκουν στο ίδιο δίκτυο.

\begin{figure}
  \begin{center}
    %\includegraphics[width=.9\textwidth]{img/kicad-main.png}
    %\caption{Δύο εξαρτήματα, συνδεδεμένα μεταξύ τους απευθείας}
    \label{fig:kicad-main}
  \end{center}
\end{figure}

\begin{figure}
  \begin{center}
    %\includegraphics[width=.9\textwidth]{img/kicad-main.png}
    %\caption{Δύο εξαρτήματα, συνδεδεμένα μεταξύ τους μέσω ετικέτας}
    \label{fig:kicad-main}
  \end{center}
\end{figure}

\paragraph{Διαγραφή αντικειμένου}
Για να διαγράψουμε ένα αντικείμενο από το φύλλο μας (πχ ένα κομμάτι σύρματος που αποφασίσαμε να τοποθετήσουμε διαφορετικά), κάνουμε δεξί κλικ επάνω του και από το μενού που εμφανίζεται επιλέγουμε "Διαγραφή".

Εναλλακτικά μπορούμε να πάμε τον δείκτη του ποντικιού επάνω από το εξάρτημα (χωρίς να κάνουμε κλικ) και να πατήσουμε το πλήκτρο Delete στο πληκτρολόγιο.

Έχοντας κατανοήσει τις παραπάνω λειτουργίες, μπορούμε να σχεδιάσουμε το κύκλωμά μας.

Για ευκολία στη σχεδίαση συνδέσεων μετακινήστε το \textenglish{CP2104} ώστε να είναι στο κέντρο της σελίδας.

Στη συνέχεια, συνδέστε με σύρμα τους ακροδέκτες 7 και 8 του \textenglish{CP2104} με το εξάρτημα VCC. Για να το κάνετε, συνδέσετε με σύρμα τον ακροδέκτη 7 με το εξάρτημα VCC και συ συνέχεια τοποθετήστε ένα σύρμα από τον  ακροδέκτη 7 προς οποιοδήποτε σημείο του προηγούμενου σύρματος.

Στη συνέχεια, συνδέστε με σύρμα τους ακροδέκτες 5,6 και 9 του \textenglish{CP2104} με το 3.3V.

Στη συνέχεια, συνδέστε με σύρμα τους ακροδέκτες 2 και 25 του \textenglish{CP2104} με το GND.

\begin{figure}
  \begin{center}
    %\includegraphics[width=.9\textwidth]{img/kicad-main.png}
    %\caption{Σύνδεση ακροδεκτών 7 και 8 του \textenglish{CP2104} με το εξάρτημα VCC}
    \label{fig:kicad-main}
  \end{center}
\end{figure}

Για να κάνετε τα παραπάνω, ίσως χρειαστεί να μετακινήσετε τα VCC, 3.3V και GND σε πιο βολικές θέσεις στο φύλλο. Για παράδειγμα, το VCC καλό θα ήταν να τοποθετηθεί πάνω αριστερά από το \textenglish{CP2104}.

\begin{figure}
  \begin{center}
    %\includegraphics[width=.9\textwidth]{img/kicad-main.png}
    %\caption{Το κύκλωμα με τις αρχικές συνδέσεις των VCC, 3.3V και GND}
    \label{fig:kicad-main}
  \end{center}
\end{figure}

Στη συνέχεια, αλλάξτε τις τιμές στους τέσσερις πυκνωτές ώστε οι τέσσερις πυκνωτές να έχουν τιμές 1u, 1u, 100n και 470n. Για να αλλάξετε τις τιμές δεξί κλικ πάνω στον πυκνωτή και επιλέξετε "Επεξεργασία Εξαρτήματος" $\rightarrow$ "Τιμή".

Στη συνέχεια, συνδέστε τους πυκνωτές 1u και 100n ως bypass στο 3.3V, δηλαδή ανάμεσα στο 3.3v και στο GND. Για να το κάνετε αυτό και το κύκλωμα να παραμείνει ευανάγνωστο μπορείτε να προσθέσετε επιπλέον εξαρτήματα GND. Όλα τα εξαρτήματα GND που θα προσθέσετε συμβολίζουν το ίδιο σημείο, τον ίδιο κόμβο στο κύκλωμά σας, τον κόμβο της γείωσης. Οπότε, για τον πυκνωτή 1u προσθέστε ένα εξάρτημα GND στο φύλλο (επιλέγοντας στο μενού Τοποθέτηση $\rightarrow$ Εξάρτημα) λίγο πιο δεξιά από το 3.3V, και μετά τοποθετήστε σύρμα που να συνδέει το 3.3V με το 1u και μετά σύρμα που να συνδέει το 1u με το (νέο) GND. Ομοίως θα κάνετε και για το 100n.

Στη συνέχεια, συνδέστε με τον ίδιο τρόπο όπως παραπάνω τον πυκνωτή 1u ως bypass στο VCC, δηλαδή συνδέστε τον ανάμεσα στο VCC και σε ένα GND.

Στη συνέχεια, συνδέστε τον πυκνωτή 470n ανάμεσα στους ακροδέκτες 16 και 2 του \textenglish{CP2104}.

Στη συνέχεια, αλλάξτε τις τιμές στα δύο εξαρτήματα CONN\_01X04 ώστε το ένα να λέγεται "USB", και το άλλο "Header". Τοποθετήστε το USB αριστερά του \textenglish{CP2104} και το Header δεξιά του \textenglish{CP2104}.

\begin{figure}
  \begin{center}
    %\includegraphics[width=.9\textwidth]{img/kicad-main.png}
    %\caption{Το κύκλωμα με τις πλήρεις συνδέσεις των VCC, 3.3V και GND και τα USB, Header τοποθετημένα}
    \label{fig:kicad-main}
  \end{center}
\end{figure}

Στη συνέχεια, θα πρέπει να συνδέσουμε όλους τους ακροδέκτες των USB και Header. Αν συνδέαμε απευθείας μεταξύ τους όλους του ακροδέκτες το κύκλωμα δεν θα ήταν ιδιαίτερα ευανάγνωστο και οργανωμένο διότι θα υπήρχαν πολλά διασταυρούμενα (αλλά μην συνδεδεμένα) σύρματα. Για να λύσουμε αυτό το πρόβλημα, θα χρησιμοποιήσουμε ετικέτες.

Με δεδομένο το εργαλείο της ετικέτας, θα τοποθετήσουμε σύρματα ώστε να συνδέσουμε όπως θέλουμε τους ακροδέκτες των USB και Header.

Τοποθετήστε σύρμα από τον ακροδέκτη 1 του USB έως το VCC.

Τοποθετήστε σύρμα από τον ακροδέκτη 4 του USB έως το GND.

Τοποθετήστε σύρμα από τον ακροδέκτη 1 του HEADER έως το 3.3V.

Τοποθετήστε σύρμα από τον ακροδέκτη 4 του HEADER έως το GND.

Τοποθετήστε σύρμα από τον ακροδέκτη 2 του USB έως το D- του \textenglish{CP2104}, χρησιμοποιώντας ετικέτα με όνομα USBDP. 

Τοποθετήστε σύρμα από τον ακροδέκτη 3 του USB έως το D+ του \textenglish{CP2104}, χρησιμοποιώντας ετικέτα με όνομα USBDM.

Τοποθετήστε σύρμα από τον ακροδέκτη 2 του HEADER έως το RXD του \textenglish{CP2104}, χρησιμοποιώντας ετικέτα με όνομα RXD.

Τοποθετήστε σύρμα από τον ακροδέκτη 3 του HEADER έως το TXD του \textenglish{CP2104}, χρησιμοποιώντας ετικέτα με όνομα TXD.

Πλέον ολοκληρώσαμε τις συνδέσεις των USB και Header.

Τώρα πρέπει να συνδέσουμε τα \textenglish{LED}. 

Μετακινήστε τα \textenglish{LED} (το ένα μονό και το ένα που αποτελείται απόδύο μέρη) και τις τρεις αντιστάσεις στην περιοχή πάνω από το \textenglish{CP2104}.

Στη συνέχεια, αλλάξτε τις τιμές στις τρεις αντιστάσεις του κυκλώματος ώστε να έχουν τις τιμές 220R, 220R και 430R.

Στη συνέχεια, τοποθετήστε σύρμα που να συνδέει το VCC με τον ένα ακροδέκτη της 430R, και τον άλλο ακροδέκτη της 430R με την άνοδο του μονού \textenglish{LED} και μετά την κάθοδο του μονού \textenglish{LED} με το GND.

Στη συνέχεια, τοποθετήστε σύρμα που να συνδέει το 3.3V με τον ένα ακροδέκτη της 220R, και τον άλλο ακροδέκτη της 220R με την άνοδο του b\textenglish{LED} (Μέρος Α) και την κάθοδο του b\textenglish{LED} (Μέρος Α), μέσω ετικέτας \textenglish{LED}RX, με τον ακροδέκτη 13 του \textenglish{CP2104}.

Στη συνέχεια, τοποθετήστε σύρμα που να συνδέει το 3.3V με τον ένα ακροδέκτη της δεύτερης 220R, και τον άλλο ακροδέκτη της δεύτερης 220R με την άνοδο του b\textenglish{LED} (Μέρος Β) και την κάθοδο του b\textenglish{LED} (Μέρος Β), μέσω ετικέτας \textenglish{LED}RX, με τον ακροδέκτη 14 του \textenglish{CP2104}.

Στη συνέχεια, προσθέστε κείμενο δίπλα στα \textenglish{LED} ώστε να είναι σαφές τι χρώμα είναι το κάθε ένα (μονό \textenglish{LED} = μπλε, A = πράσινο, B = πορτοκαλί). Για να προσθέσουμε κείμενο, επιλέγουμε από τη δεξιά μπάρα το το εικονίδιο "Τοπoθέτηση κειμένου" %\includegraphics{img/graphic.png}
, κάνουμε κλικ στο φύλλο, και στο παράθυρο που εμφανίζεται γράφουμε ό,τι κείμενο θέλουμε, επιλέγουμε αν θα εμφανίζεται κάθετα, τι στοίχιση θα έχει, κοκ.

\begin{figure}
  \begin{center}
    %\includegraphics[width=.9\textwidth]{img/kicad-main.png}
    %\caption{Συνδεσμολογία \textenglish{LED}}
    \label{fig:kicad-main}
  \end{center}
\end{figure}

Τέλος, για λόγους καλύτερης οργάνωσης, πρέπει να προσθέσουμε ετικέτες και στα υπόλοιπα δίκτυα του κυκλώματός μας, και συγκεκριμένα στο VPP (ετικέτα VPP), και στα σύρματα ανάμεσα στις αντιστάσεις και τα \textenglish{LED} (ετικέτες \textenglish{LED}ONR, \textenglish{LED}RXR και \textenglish{LED}TXR).

Πλέον έχουμε σχεδιάσει πλήρως το κύκλωμα. Για λόγους καλύτερης οργάνωσης, και ηλεκτρολογικού ελέγχου του κυκλώματος, χρειαζόμαστε ακόμα δύο βήματα.

Καταρχάς πρέπει σε όλους τους ακροδέκτες που δεν έχουν κάτι συνδεδεμένο επάνω τους (όπως πχ οι ακροδέκτες 15 και 17 του \textenglish{CP2104}) να τοποθετήσουμε ένα σύμβολο Μη Σύνδεσης. Αυτό θα βοηθήσει στο να δηλώσουμε στο \textenglish{KiCad} ότι ο συγκεκριμένος ακροδέκτης θέλουμε όντως να μην συνδέεται πουθενά και δεν έχουμε απλά ξεχάσει να τον συνδέσουμε. Για να τοποθετήσουμε σύμβολο Μη Σύνδεσης, επιλέγουμε από το μενού στο πάνω μέρος της οθόνης "Τοποθέτηση" $\rightarrow$ "Σήμανση Μη Σύνδεσης" και κάνουμε κλικ στο σημείο όπου θέλουμε να το τοποθετήσουμε. Πρέπει να τοποθετήσουμε μία Σήμανση Μη Σύνδεσης σε κάθε ακροδέκτη του \textenglish{CP2104} που δεν συνδέεται με κάτι άλλο.

Κατά δεύτερον, πρέπει στα δίκτυα VCC και GND να συνδέσουμε μέσω σύρματος από ένα εξάρτημα PWR\_FLAG (από τη βιβλιοθήκη power). Με αυτό τον τρόπο το \textenglish{KiCad}, όταν κάνει ηλεκτρολογικό έλεγχο, θα καταλάβει ότι τα δύο δίκτυα αυτά είναι δίκτυα ισχύος, και ότι υπάρχει ισχύς στο κύκλωμά μας.

Πλέον έχουμε ολοκληρώσει πλήρως το κύκλωμα στο \textenglish{KiCad}.

\begin{figure}
  \begin{center}
    \includegraphics{img/final-schem.png}
    \caption{Πλήρως σχεδιασμένο κύκλωμα}
    \label{fig:final-schem}
  \end{center}
\end{figure}

\subsubsection{Ολοκλήρωση σχεδιασμού σχηματικού}
Αφού έχουμε ολοκληρώσει τον σχεδιασμό του κυκλώματος, πρέπει να δώσουμε ονομασίες αναφοράς σε όλα τα εξαρτήματα, ώστε οι αντιστάσεις να έχουν ονομασία αναφοράς R1, R2, R3 κοκ. Να σημειωθεί ότι η ονομασία αναφοράς είναι διαφορετική έννοια από την τιμή (πχ 220R). 

Ονομασίες αναφοράς μπορούμε να δώσουμε μόνοι μας, κάνοντας επεξεργασία σε κάθε ένα εξάρτημα ή μπορούμε να πούμε στο \textenglish{KiCad} να δώσει αυτόματα ονομασίες αναφοράς σε όλα τα εξαρτήματα. Για να το το κάνουμε αυτό, επιλέγουμε από το μενού στο πάνω μέρος της οθόνης "Εργαλεία" $\rightarrow$ "Ονοματοδοσία σχηματικού", στο παράθυρο που εμφανίζεται δεν αλλάζουμε καμία ρύθμιση και κάνουμε κλικ στο "Ονοματοδοσία".

\begin{figure}
  \begin{center}
    %\includegraphics[width=.9\textwidth]{img/kicad-main.png}
    %\caption{Παράθυρο Ονοματοδοσίας}
    \label{fig:kicad-main}
  \end{center}
\end{figure}

Όταν ολοκληρωθεί η διαδικασία, όλα τα εξαρτήματα θα έχουν συγκεκριμένες, αριθμημένες ονομασίες αναφοράς (και όχι ερωτηματικά).

Αφού ολοκληρωθεί η ονοματοδοσία, πρέπει να κάνουμε Έλεγχο Ηλεκτρικών Κανόνων στο κύκλωμα, ώστε να αναλύσει το κύκλωμα το \textenglish{KiCad} και να μας ενημερώσει για πιθανά προβλήματα όπως πχ αν έχουμε ξεχάσει να συνδέσουμε ένα δίκτυο ή αν ένας ακροδέκτης εξαρτήματος έχει χαρακτηρισμό "Είσοδος" αλλά είναι συνδεδεμένος με έναν ακροδέκτη που έχει επίσης χαρακτηρισμό "Είσοδος".

Για να κάνουμε Έλεγχο Ηλεκτρικών Κανόνων, επιλέγουμε από το μενού στο πάνω μέρος της οθόνης "Εργαλεία" $\rightarrow$ "Ελεγκτής Ηλεκτρικών Κανόνων" και στο παράθυρο που εμφανίζεται δεν αλλάζουμε καμία ρύθμιση και κάνουμε κλικ στο "Εκτέλεση". 

Αν έχετε χρησιμοποιήσει τα αρχεία του tutorial, ο Ελεγκτής Ηλεκτρικών Κανόνων πρέπει να σας βγάλει μία προειδοποίηση: ότι ο ακροδέκτης 9 (χαρακτηρισμένος ως αμφίδρομος στο εξάρτημα) του \textenglish{CP2104} είναι συνδεδεμένος με τον ακροδέκτη 6 (χαρακτηρισμένος ως έξοδος) του \textenglish{CP2104}. Αυτό φαίνεται στο \textenglish{KiCad} να είναι λάθος διότι ένας ακροδέκτης έξοδος πρέπει να είναι συνδεδεμένος με έναν ακροδέκτη είσοδο και όχι έναν αμφίδρομο. Στη συγκεκριμένη περίπτωση πρόκειται για απλό θέμα χαρακτηρισμού του ακροδέκτη 9 και μπορούμε να αγνοήσουμε την προειδοποίηση. Οπότε απλά κλείστε το παράθυρο του Ελεγκτή.
Εναλλακτικά, μπορούμε να επεξεργαστούμε το εξάρτημα \textenglish{CP2104} με το σχετικό εργαλείο (εξάλλου εμείς το δημιουργήσαμε το εξάρτημα, από το μηδέν) και να αλλάξουμε τον χαρακτηρισμό του ακροδέκτη 9 σε είσοδο.

Τέλος, αφού έχουμε ολοκληρώσει το κύκλωμά μας, καλό είναι να το κρατήσουμε και σε ένα αρχείο pdf, ώστε να είναι εύκολη η αναφορά σε αυτό χωρίς να ανοίγουμε το \textenglish{KiCad}. Για να το κάνουμε αυτό, επιλέγουμε από το μενού στο πάνω μέρος της οθόνης Αρχείο $\rightarrow$ Σχεδιογράφηση $\rightarrow$ Σχεδιογράφηση και στο παράθυρο που εμφανίζεται επιλέγουμε Μορφή PDF, και πατάμε το κουμπί "Σχεδιογράφηση Τρέχουσας Σελίδας". 

\begin{figure}
  \begin{center}
    %\includegraphics[width=.9\textwidth]{img/kicad-main.png}
    %\caption{Παράθυρο Σχεδιογράφησης}
    \label{fig:kicad-main}
  \end{center}
\end{figure}

Πλέον το κύκλωμα υπάρχει και σε μορφή pdf, στο φάκελο όπου έχουμε και όλα τα άλλα αρχεία του έργου μας.

%Σε αυτό το σημείο αντιστοιχούν τα αρχεία από το αρχείο kicad\_tut05.zip.

Αφού έχουμε ολοκληρώσει το σχηματικό κύκλωμα, θα χρειαστεί να κρατήσουμε κάποια από τα δεδομένα του, τα οποία θα μας χρειαστούν κατά τη σχεδίαση της πλακέτας. Αυτά τα δεδομένα είναι η λίστα υλικών και η λίστα δικτύων.

Για να κρατήσουμε λίστα με όλα τα υλικά και εξαρτήματα του κυκλώματος, από το μενού επιλέγουμε Εργαλεία $\rightarrow$ Δημιουργία Λίστας Υλικών, στο παράθυρο που εμφανίζεται επιλέγουμε Δημιουργία, και τέλος επιλέγουμε Κλείσιμο. Πλέον έχει δημιουργηθεί ένα αρχείο usb2uart.xml, στο φάκελο όπου έχουμε και όλα τα άλλα αρχεία του έργου μας. Αυτό το αρχείο είναι η λίστα υλικών (Bill Of Materials - BOM) του κυκλώματός μας.

Για να κρατήσουμε λίστα με όλα τα δίκτυα του κυκλώματος, από το μενού επιλέγουμε Εργαλεία $\rightarrow$ Δημιουργία Αρχείου Λίστας Δικτύων, στο παράθυρο που εμφανίζεται επιλέγουμε Δημιουργία, και τέλος αποθηκεύουμε το αρχείο με όνομα usb2uart.net στο φάκελο που θέλουμε. Συνίσταται να αποθηκευτεί μαζί με όλα τα άλλα αρχεία του έργου.

%* set custom fields (MFG, MFG code) [TAG JUMP to "tutorial05"]
%-= TAG "tutorial05" =-

%Σε αυτό το σημείο αντιστοιχούν τα αρχεία από το αρχείο kicad\_tut06.zip.

%  - open CSV with localc

\section{Δημιουργία Πλακέτας (PCB)}

\subsection{Δημιουργία πλακέτας}
Από την κεντρική οθόνη του \textenglish{KiCad} εκτελέστε την εφαρμογή \textenglish{Pcbnew}, με την οποία θα σχεδιάσουμε την πλακέτα.
%* accept warning

\begin{figure}
  \begin{center}
    %\includegraphics[width=.9\textwidth]{img/kicad-main.png}
    %\caption{Κεντρική οθόνη του \textenglish{Pcbnew}}
    \label{fig:kicad-main}
  \end{center}
\end{figure}

Αφού βρεθείτε στην κεντρική οθόνη του \textenglish{Pcbnew}, επιλέξτε από το μενού Αρχείο $\rightarrow$ Αποθήκευση. Το \textenglish{Pcbnew} θα δημιουργήσει ένα νέο αρχείο στον υπολογιστή σας, το αρχείο usb2uart.kicad\_pcb και θα το αποθηκεύσει μαζί με τα υπόλοιπα αρχεία του έργου σας.

H κεντρική οθόνη του \textenglish{Pcbnew} αποτελείται από τα παρακάτω
\begin{itemize}
    \item ένας κενός χώρος για σχεδίαση στο κέντρο της οθόνης
    \item γραμμή μενού στο πάνω μέρος της οθόνης
    \item μπάρα βασικών λειτουργιών στο πάνω μέρος της οθόνης
    \item μπάρα με γενικές λειτουργίες στο αριστερό μέρος της οθόνης
    \item μπάρες με συγκεκριμένες λειτουργίες σχεδίασης στο δεξί μέρος της οθόνης
    \item λίστα όλα τα επίπεδα της πλακέτας στο δεξί μέρος της οθόνης
\end{itemize}

Αρχικά πρέπει να ορίσουμε τις ιδιότητες της σελίδας μας, από το μενού Αρχείο $\rightarrow$ Ρυθμίσεις σελίδας. Σε αυτό το παράθυρο, ορίστε τις ρυθμίσεις της σελίδας σας, όπως μέγεθος χαρτιού, ημερομηνία έκδοσης, τίτλος κυκλώματος, και πατήστε ΟΚ. Αυτές οι ρυθμίσεις δεν έχουν κάποια ηλεκτρική ή λειτουργική σημασία για το κύκλωμα, είναι όμως χρήσιμες πληροφορίες για την οργάνωση των κυκλωμάτων σας.

\begin{figure}
  \begin{center}
    %\includegraphics[width=.9\textwidth]{img/kicad-main.png}
    %\caption{Ρυθμίσεις Σελίδας \textenglish{Pcbnew}}
    \label{fig:kicad-main}
  \end{center}
\end{figure}

Στη συνέχεια θα πρέπει να ορίσουμε τα επίπεδα από τα οποία θα αποτελείται η πλακέτα μας. Έχουμε πάρει την απόφαση ότι το κύκλωμα που θα σχεδιάσουμε θέλουμε να έχει δύο επίπεδα, και θα τοποθετηθούν εξαρτήματα μόνο στο ένα επίπεδο, το μπροστινό επίπεδο. 

Από το μενού επιλέξτε Κανόνες Σχεδίασης $\rightarrow$ Ρύθμιση Επιπέδων. Στο παράθυρο που θα εμφανιστεί επιλέξτε από πάνω αριστερά την προκαθορισμένη ομαδοποίηση "Δύο επίπεδα, εξαρτήματα μόνο Μπροστά" και φροντίστε να είναι επιλεγμένα μόνο τα παρακάτω επίπεδα
\begin{itemize}
    \item F.Paste, διότι θα έχουμε εξαρτήματα SMD στην μπροστινή πλευρά
    \item F.SilkS, διότι θα έχουμε σχέδια και κείμενα στην μπροστινή πλευρά
    \item F.Mask, διότι θα έχουμε εξαρτήματα στην μπροστινή πλευρά
    \item F.Cu, διότι θα έχουμε αγώγιμους δρόμους στην μπροστινή πλευρά
    \item B.Cu, διότι θα έχουμε αγώγιμους δρόμους στην πίσω πλευρά
    \item B.Mask, διότι θα έχουμε (μόνο διαμπερή) εξαρτήματα στην πίσω πλευρά
    \item Edge.Cuts, για να σχεδιάσουμε τα όρια της πλακέτας
    \item Dwgs.User για να μπορούμε να σχεδιάζουμε σημειώσεις, να γράφουμε μηχανικές διαστάσεις, κα
\end{itemize}

Αφού κάνετε τα παραπάνω, επιλέξτε ΟΚ και θα επανέλθετε στην κεντρική οθόνη του \textenglish{Pcbnew}.

\begin{figure}
  \begin{center}
    %\includegraphics[width=.9\textwidth]{img/kicad-main.png}
    %\caption{Παράθυρο Ρύθμισης Επιπέδων}
    \label{fig:kicad-main}
  \end{center}
\end{figure}

Στο δεξί μέρος της κεντρικής οθόνης του \textenglish{Pcbnew} υπάρχει λίστα με όλα τα επίπεδα της πλακέτας. Κάνοντας κλικ στο μικρό κουτάκι με το χρώμα αριστερά από το όνομα κάθε επιπέδου, μπορούμε να αλλάξουμε εάν θέλουμε το χρώμα που χρησιμοποιεί το \textenglish{KiCad} για να εμφανίσει κάθε επίπεδο. Συνήθως επιλέγουμε να εμφανίζεται κόκκινο το μπροστά επίπεδο (F.Cu) και να εμφανίζεται πράσινο το πίσω επίπεδο (B.Cu).

Επίσης σε αυτή τη λίστα φαίνεται ένα μικρό βέλος δίπλα στο επίπεδο το οποίο είναι τρέχων, το επίπεδο δηλαδή στο οποίο σχεδιάζουμε. Σε κάθε σημείο όπου σχεδιάζουμε ή τοποθετούμε κάται στην πλακέτα μας, πρέπει να φροντίζουμε αυτό να γίνεται στο σωστό επίπεδο.

Αφού ορίσουμε τα επίπεδα, πρέπει να καθορίσουμε τους κανόνες σχεδίασης που θα διέπουν την πλακέτα μας. Από το μενού επιλέξτε Κανόνες Σχεδίασης $\rightarrow$ Κανόνες Σχεδίασης. Στο παράθυρο που θα εμφανιστεί ορίστε το Διάκενο στα 0,2mm το Πλάτος Δρόμου στα 0.25mm και τη Διάμετρο/Διάτρηση Via στα 0,7/0,4mm. Αφού ορίσετε όλες αυτές τις διαστάσεις, πατήστε ΟΚ και θα επανέλθετε στην κεντρική οθόνη του \textenglish{Pcbnew}.

\begin{figure}
  \begin{center}
    %\includegraphics[width=.9\textwidth]{img/kicad-main.png}
    %\caption{Παράθυρο 'Κανόνες Σχεδίασης'}
    \label{fig:kicad-main}
  \end{center}
\end{figure}

%* set pad dimensions
%  - +0.05mm to mask
%  - 0.25mm to minimum mask
%  - -0.02mm to paste

Μία πολυύ χρήσιμη λειτουργία για τη σχεδίαση στο \textenglish{Pcbnew} είναι ο ορισμός και η αλλαγή του πλέγματος το οποίο υπάρχει νοητά πάνω στο κύκλωμά μας και ορίζει τα σημεία όπου μπορεί να τοποθετηθεί ο δείκτης του ποντικιού. Πλέγμα 1mm σημαίνει ότι ο δείκτης του ποντικιού μπορεί να τοποθετηθεί σε διαστήματα του ενός mm αλλά όχι ενδιάμεσα. Το πλέγμα είναι βοηθητική έννοια και δεν επηρεάζει με κάνενα τρόπο το κύκλωμά μας. Για να αλλάξει το πλέγμα, κάνουμε δεξί κλικ σε κενό χώρο της κεντρική οθόνης και επιλέγουμε "Επιλογή Πλέγματος".

%Σε αυτό το σημείο αντιστοιχούν τα αρχεία από το αρχείο kicad\_tut07.zip.

Στη συνέχεια θα χρειαστεί να ορίσουμε τις βιβλιοθήκες αποτυπωμάτων που θα χρησιμοποιήσουμε. 

Από μενού επιλέξτε Προτιμήσεις $\rightarrow$ Διαχειριστής Βιβλιοθηκών Αποτυπωμάτων. Σε αυτό το παράθυρο, επιλέξτε "Προσθήκη με Οδηγό" και προσθέστε από το αποθετήριο https://github.com/KiCad τις παρακάτω βιβλιοθήκες 

\begin{itemize}
    \item \textenglish{LED}s.pretty
    \item Capacitors\_SMD.pretty
    \item Resistors\_SMD.pretty
    \item SMD\_Packages.pretty
    \item Pin\_Headers.pretty
    \item Connect.pretty
\end{itemize}

Οι βιβλιοθήκες πρέπει να προστεθούν μόνο στο τρέχον Έργο και όχι ως Καθολικές.

\begin{figure}
  \begin{center}
    %\includegraphics[width=.9\textwidth]{img/kicad-main.png}
    %\caption{Παράθυρο 'Διαχειριστής Βιβλιοθηκών Αποτυπωμάτων'}
    \label{fig:kicad-main}
  \end{center}
\end{figure}

\begin{figure}
  \begin{center}
    %\includegraphics[width=.9\textwidth]{img/kicad-main.png}
    %\caption{Οδηγός Προσθήκης Βιβλιοθηκών Αποτυπωμάτων'}
    \label{fig:kicad-main}
  \end{center}
\end{figure}

%* add contrib to search path (relative)

Από το μενού επιλέξτε Αρχείο $\rightarrow$ Αποθήκευση για να αποθηκευτούν όλες οι αλλαγές που κάνατε.

%Σε αυτό το σημείο αντιστοιχούν τα αρχεία από το αρχείο kicad\_tut08.zip.

\subsection{Δημιουργία αποτυπώματος}
Η εφαρμογή \textenglish{Pcbnew} χρησιμοποιεί αποτυπώματα για να αναπαριστά τα εξαρτήματα. Το αποτύπωμα ενός εξαρτήματος δείχνει πως ακριβώς θα τοποθετηθεί, ενωθεί και κολληθεί ένα εξάρτημα σε μία πλακέτα. Πληροφορίες για το πως είναι το αποτύπωμα κάθε εξαρτήματος μπορούν να βρεθούν στο εγχειρίδιο (data sheet) κάθε εξαρτήματος. 

Συνήθως το αποτύπωμα κάθε εξαρτήματος δεν είναι τελείως μοναδικό, αλλά ανήκει σε κάποια γενική κατηγορία. Για παράδειγμα το εξάρτημα \textenglish{CP2104} του κυκλώματός μας έχει αποτύπωμα τύπου qfn24.

Κάθε εξάρτημα (πυκνωτής, αντίσταση, ολοκληρωμένο, \textenglish{LED}, κοκ) του κυκλώματός μας πρέπει να έχει ένα αποτύπωμα, το οποίο θα τοποθετήσουμε στην πλακέτα που σχεδιάζουμε.

\begin{figure}
  \begin{center}
    %\includegraphics[width=.9\textwidth]{img/kicad-main.png}
    %\caption{Ένα εξάρτημα και το αποτύπωμά του}
    \label{fig:kicad-main}
  \end{center}
\end{figure}

Το \textenglish{KiCad} έχει βιβλιοθήκες με αποτυπώματα για πολλά τυπικά εξαρτήματα. Αν κάποιο εξάρτημα που θέλουμε να χρησιμοποιήσουμε δεν έχει αποτύπωμα στις βιβλιοθήκες του \textenglish{KiCad}, μπορούμε να σχεδιάσουμε μόνοι μας το αποτύπωμα από το μηδέν, χρησιμοποιώντας το εργαλείο "Επεξεργαστής Αποτυπώματος".

Συγκεκριμένα για το κύκλωμα usb2uart που θέλουμε να σχεδιάσουμε, πρέπει να σχεδιάσουμε με τον Επεξεργαστή Αποτυπώματος το αποτύπωμα για το διπλό \textenglish{LED} που χρησιμοποιούμε (εξάρτημα b\textenglish{LED}, αποτύπωμα b\textenglish{LED}\_0603) και για το ολοκληρωμένο που χρησιμοποιούμε (εξάρτημα \textenglish{CP2104}, αποτύπωμα qfn24), καθώς τα αποτυπώματα αυτά δεν είναι στις βιβλιοθήκες του \textenglish{KiCad}. Εναλλακτικά, συνίσταται να πάρετε τα αποτυπώματα αυτά από τα αρχεία του tutorial, καθώς η δημιουργία αποτυπωμάτων από το μηδέν είναι μία επίπονη διαδικασία που ξεφεύγει από τα πλαίσια του tutorial.

%Σε αυτό το σημείο αντιστοιχούν τα αρχεία από το αρχείο kicad\_tut09.zip.

\begin{comment}
4.2 Δημιουργία αποτυπώματος
---------------------------
* start library editor
* create new module (b\textenglish{LED}\_0603)
* create new library and save current module
* quit lirary editor
* add usb2uart.mod library
* start library editor
* select active library
* load b\textenglish{LED}\_0603
* move labels out of the way
* insert pad
* edit pad
  - type=smd
  - shape=rect
  - sx=0.7, sy=0.4 (mm)
  - posx= -0.65, posy= -0.35
* insert and edit coordinates for 3 more pads
* select 0.2mm grid, add a 1.6x0.8mm box
* replace labels
* view 3d
* edit module properties, assign 3d from walter
* view 3d
  - uncheck render 3d
* make sure of selected library


* create new module (qfn24)
* select 1mm grid, make 1x1mm box 
* select 0.5mm grid, make pin1 mark
* select 0.25mm grid
* insert pad, click at -2, -1.25
* edit pad
  - type=smd
  - shape=rect
  - sx=0.7, sy=0.25 (mm)
  - posx= -2, posy= -1.25
* set silkscreen to transparent
* insert 5 more with 0.5mm pitch
* insert pad 7 at -1.25, 2
* edit pad 7, orientation 90
* and so on, until pad 24
* click at 0,0 insert pad 25
* edit pad 25
  - sx=sy=2.75
  - uncheck F.Paste
* uncheck show pads in fill mode
* insert pad outside/away of footprint
* edit pad
  - type=smd
  - shape=rect
  - posx=posy=-0.65mm
  - sx=sy=1.1mm
  - no copper layer, only F.Paste
* insert 3 more around the X and Y axis
* restore silkscreen and pad view
* edit module properties
  - mask clear 0.06mm
  - paste clear lim $\rightarrow$ zero (e.g. -0.00001)
  - assign 3d from walter
* view 3d
  - uncheck render 3d
* edit module properties
  - rotate 3d on Z by 90
* view 3d
  - uncheck render 3d
* close editor and pcbnew

-= TAG "tutorial09" =-
\end{comment}

\subsection{Αντιστοίχηση αποτυπωμάτων}
Σε αυτό το σημείο θεωρούμε ότι έχετε όλα τα απαραίτητα αποτυπώματα για το κύκλωμα, αποθηκευμένα στις βιβλιοθήκες του έργου.

Αρχικά θα χρειαστεί να ορίσουμε ποια αποτυπώματα αντιστοιχούν σε ποια εξαρτήματα στο κύκλωμά μας. 

Για να το κάνουμε αυτό, εκτελούμε το πρόγραμμα Eeschema, και από το μενού επιλέγουμε Εργαλεία $\rightarrow$ Αντιστοίχηση Αποτυπωμάτων με Εξαρτήματα. Αυτό θα εκτελέσει τη βοηθητική εφαρμογή \textenglish{Cvpcb} η οποία κάνει ακριβώς αυτή τη δουλειά. Αυτό σημαίνει ότι φορτώνει τη Λίστα Δικτύων του κυκλώματος (την οποία παράξαμε στο τέλος της σχεδίασης του σχηματικού), και μας επιτρέπει να ορίσουμε για κάθε εξάρτημα στη λίστα και ένα αρχείο αποτυπώματος.

\begin{figure}
  \begin{center}
    %\includegraphics[width=.9\textwidth]{img/kicad-main.png}
    %\caption{Βοηθητική εφαρμογή \textenglish{Cvpcb}, με αντιστοιχισμένα όλα τα εξαρτήματα σε αποτυπώματα}
    \label{fig:kicad-main}
  \end{center}
\end{figure}

Αφού ολοκληρώσετε την αντιστοίχιση, επιλέξτε από το μενού Αρχείο $\rightarrow$ Αποθήκευση ώστε να αποθηκευτεί η αντιστοίχιση στο αρχείο usb2uart.cmp. Με αυτό τον τρόπο, θα μπορούμε να φορτώσουμε αυτό το αρχείο στο \textenglish{Pcbnew} να γνωρίζει το \textenglish{Pcbnew} ποια είναι τα εξαρτήματα του κυκλώματος. 

Αφού κάνετε τα παραπάνω, κλείστε την εφαρμογή \textenglish{Cvpcb}.

%Σε αυτό το σημείο αντιστοιχούν τα αρχεία από το αρχείο kicad\_tut10.zip.

\subsection{Εισαγωγή αποτυπωμάτων}
Από την κεντρική οθόνη του \textenglish{KiCad} εκτελέστε την εφαρμογή \textenglish{Pcbnew}. 

Από το δεξί μέρος της κεντρικής οθόνης του \textenglish{Pcbnew}, επιλέξτε το επίπεδο F.cu ως τρέχον επίπεδο.

Στη συνέχεια επιλέξτε στο μενού Εργαλεία $\rightarrow$ Λίστα Δικτύων, στο παράθυρο που εμφανίζεται επιλέξτε "Ανάγνωση τρέχουσας λίστας δικτύων" και τέλος επιλέξτε Κλείσιμο.

\subsubsection{Προσωρινή τοποθέτηση των αποτυπωμάτων}
Πλέον στην πλακέτα μας έχουν εισαχθεί τα αποτυπώματα από όλα τα εξαρτήματα του κυκλώματος. 

Τα αποτυπώματα έχουν τοποθετηθεί προσωρινά όλα στο ίδιο σημείο της πλακέτας. Για καλύτερη οργάνωση, μπορείτε να μετακινήσετε ένα προς ένα τα εξαρτήματα ώστε να τα τοποθετήσετε (και πάλι προσωρινά) το ένα δίπλα στο άλλο ώστε να έχετε καλύτερη εικόνα του κυκλώματος. 

Για να μετακινήσετε ένα αποτύπωμα ακολουθείτε την ίδια διαδικασία όπως και για ένα εξάρτημα. Αυτό σημαίνει ότι μπορείτε να κάνετε δεξί κλικ στο αποτύπωμα και να επιλέξετε μετακίνηση ή να πάτε τον δείκτη του ποντικιού πάνω από το αποτύπωμα (χωρίς να κάνετε κλικ) και να πατήσετε το πλήκτρο m ή να ορίσετε τις ακριβείς συντεταγμένες του (Χ,Ψ) στις παράθυρο Ιδιότητες Εξαρτήματος. 

Εναλλακτικά, αντί να μετακινήσετε όλα τα εξαρτήματα ένα προς ένα μόνοι σας, μπορείτε να πείτε στο \textenglish{Pcbnew} να τα διατάξει αυτόματα στην άκρη της σελίδας σας. Για να το κάνετε αυτό, κάντε κλικ στην επάνω μπάρα στο κουμπί "Λειτουργία αποτυπώματος" ώστε το κουμπί να είναι πατημένο και στη συνέχεια κάντε κλικ επάνω σε κενό σημείο του κυκλώματος και επιλέξτε Καθολικό Άπλωμα και Τοποθέτηση $\rightarrow$ Άπλωμα Όλων των Αποτυπωμάτων, επιλέγοντας Yes/Ναι αν η εφαρμογή σας ζητήσει να επιβεβαιώσετε τη λειτουργία. 

Πλέον όλα τα αποτυπώματα είναι τακτοποιημένα στην άκρη της σελίδας, και έτοιμα να τα τοποθετήσετε ένα προς ένα στις τελικές τους θέσεις.

\begin{figure}
  \begin{center}
    %\includegraphics[width=.9\textwidth]{img/kicad-main.png}
    %\caption{Άπλωμα Όλων των Αποτυπωμάτων}
    \label{fig:kicad-main}
  \end{center}
\end{figure}

\subsubsection{Φωλιά Συνδέσεων}
Θα παρατηρήσετε ότι κάθε αποτύπωμα έχει λεπτές λευκές γραμμές που ενώνουν κάθε μία έδρα/ακροδέκτη του με την έδρα/ακροδέκτη που πρέπει, σύμφωνα με το σχηματικό κύκλωμα. Για παράδειγμα υπάρχει μία λεπτή λευκή γραμμή που ενώνει τη μία άκρη του R2 με τη μία άκρη του \textenglish{LED}. 

Το σύνολο αυτών των λευκών γραμμών λέγεται "Φωλιά Συνδέσεων". Ένας από τους ουσιαστικούς στόχους της σχεδίασης της πλακέτας είναι να αντικαταστήσουμε κάθε μία από από τις γραμμές της φωλιάς με μία πραγματική σύνδεση πάνω στην πλακέτα μας, έναν αγώγιμο δρόμο. 

Μπορούμε να επιλέξουμε την εμφάνιση ή όχι της φωλιάς, κάνοντας κλικ στο κουμπί "Απόκρυψη/Εμφάνιση φωλιάς συνδέσεων πλακέτας" στην αριστερή μπάρα εργαλείων.

\begin{figure}
  \begin{center}
    %\includegraphics[width=.9\textwidth]{img/kicad-main.png}
    %\caption{Κεντρική οθόνη του \textenglish{Pcbnew} με όλα τα αποτυπώματα}
    \label{fig:kicad-main}
  \end{center}
\end{figure}

%Σε αυτό το σημείο αντιστοιχούν τα αρχεία από το αρχείο kicad\_tut11.zip.

\subsubsection{Οριστική τοποθέτηση αποτυπωμάτων}

Σε αυτό το σημείο θα μετακινήσουμε τα αποτυπώματα στις τελικές θέσεις τους στην πλακέτα μας. Στο κάτω μέρος της κεντρικής οθόνης του \textenglish{Pcbnew} εμφανίζονται πολλές χρήσιμες πληροφορίες για το κύκλωμα, και για αντικείμενο που έχετε επιλέξει. Βασική πληροφορία είναι οι συντεταγμένες Χ και Ψ της θέσης του δείκτη του ποντικιού.

Μπορούμε να ορίσουμε τις συντεταγμένες όπου θα τοποθετηθεί ένα αποτύπωμα, στις ιδιότητές του κάνοντας δεξί κλικ επάνω του και επιλέγοντας "Αποτύπωμα (στοιχεία αποτυπώματος)" $\rightarrow$ Επεξεργασία Παραμέτρων. Εναλλακτικά, τοποθετείτε τον δείκτη του ποντικιού πάνω από το αποτύπωμα (χωρίς να κάνετε κλικ) και να πατήσετε το πλήκτρο e.

Στο Παράθυρο Ιδιότητες Αποτυπώματος που εμφανίζεται, στο κάτω αριστερά μέρος του, μπορείτε να ορίσετε την ακριβή θέση του αποτυπώματος με βάση συντεταγμένες.

\begin{figure}
  \begin{center}
    %\includegraphics[width=.9\textwidth]{img/kicad-main.png}
    %\caption{Παράθυρο 'Ιδιότητες Αποτυπώματος'}
    \label{fig:kicad-main}
  \end{center}
\end{figure}

%* hide ratsnest
%* make names invisible, toggle hidden text and anchor visibility

Στη συνέχεια ορίστε το πλέγμα στα 1mm.

Μετακινήστε το U1 στις συντεταγμένες (148,105), και κλειδώστε το σε αυτή τη θέση για να μην μετακινηθεί κατά λάθος. Για να κλειδώσετε ένα εξάρτημα σε μία θέση, κάντε δεξί κλικ επάνω και από το παράθυρο που εμφανίζεται επιλέξτε Κλείδωμα Αποτυπώματος. Εναλλακτικά μπορείτε να πάτε τον δείκτη του ποντικιού πάνω από το αποτύπωμα (χωρίς να κάνετε κλικ) και να πατήσετε το πλήκτρο l για να το κλειδώσετε.

Στη συνέχεια μετακινήστε το J1 στις συντεταγμένες (133,105) και κλειδώστε το.

Στη συνέχεια μετακινήστε το J2 στις συντεταγμένες (156,105) και κλειδώστε το.

Στη συνέχεια ορίστε το πλέγμα στα 0,25mm.

Στη συνέχεια μετακινήστε το C1 στις συντεταγμένες (146.50 108.50), και περιστρέψτε το κατά 90 μοίρες.  Για να περιστρέψετε ένα εξάρτημα κατά 90 μοίρες, κάντε δεξί κλικ επάνω και από το παράθυρο που εμφανίζεται επιλέξτε "Αποτύπωμα (στοιχεία αποτυπώματος)" $\rightarrow$ Περιστροφή. Εναλλακτικά μπορείτε να πάτε τον δείκτη του ποντικιού πάνω από το αποτύπωμα (χωρίς να κάνετε κλικ) και να πατήσετε το πλήκτρο r για να το περιστρέψετε.

%TBD φορά περιστροφής

Στη συνέχεια μετακινήστε το C2 στις συντεταγμένες (151.50 104.25), και περιστρέψτε το κατά 270 μοίρες. 

Στη συνέχεια μετακινήστε το C3 στις συντεταγμένες (143.00 106.75).

Στη συνέχεια μετακινήστε το C4 στις συντεταγμένες (143.00 105.50).

Στη συνέχεια μετακινήστε το D1 στις συντεταγμένες (144.50 109.75), και περιστρέψτε το κατά 270 μοίρες. 

Στη συνέχεια μετακινήστε το D2 στις συντεταγμένες (151.50 109.75), και περιστρέψτε το κατά  90 μοίρες. 

Στη συνέχεια μετακινήστε το R1 στις συντεταγμένες (143.00 108.50), και περιστρέψτε το κατά  90 μοίρες. 

Στη συνέχεια μετακινήστε το R2 στις συντεταγμένες (149.75 109.75), και περιστρέψτε το κατά  90 μοίρες. 

Στη συνέχεια μετακινήστε το R3 στις συντεταγμένες (153.25 109.75), και περιστρέψτε το κατά  90 μοίρες. 

Πλέον έχουμε τοποθετήσει τα αποτυπώματα στις τελικές θέσεις τους.
 
%Σε αυτό το σημείο αντιστοιχούν τα αρχεία από το αρχείο kicad\_tut12.zip.

\subsection{Δημιουργία αγώγιμων ζωνών και δρόμων}

\subsubsection{Σχεδίαση ορίων πλακέτας}
Αρχικά ορίστε το πλέγμα στα 1mm.

Τοποθετήστε το δείκτη του ποντικιού στο κέντρο του U1 και πατήστε το πλήκτο space για να οριστεί αυτό το σημείο ως το σημείο έναρξης των σχετικών συντεταγμένων.

Σε αυτό το σημείο πρέπει να ορίσουμε τα όρια της πλακέτας. Για να το κάνουμε αυτό δεν υπάρχει κάποιο ειδικό εργαλείο, αλλά αρκεί να σχεδιάσουμε γραμμές γραφικών στο επίπεδο Edge.Cuts.

Από το δεξί μέρος της κεντρικής οθόνης του \textenglish{Pcbnew}, επιλέξτε το επίπεδο Edge.Cuts ως τρέχον επίπεδο. Επίσης από την μπάρα στο αριστερό μέρος της οθόνης φροντίστε να μην είναι επιλεγμένη η Εμφάνιση πολικών συντεταγμένων. %\includegraphics[width=.9\textwidth]{img/kicad-main.png}

Στο επίπεδο Edge.Cuts σχεδιάστε ένα ορθογώνιο διαστάσεων 32x18mm με σχετικές συντεταγμένες (-19,-9) πάνω αριστερά και (13, 9) κάτω δεξιά. 

Για να το κάνετε αυτό βεβαιωθείτε ότι έχετε ορίσει το κέντρο του U1  ως το σημείο έναρξης των σχετικών συντεταγμένων, επιλέξτε από το μενού στο πάνω μέρος της οθόνης Τοποθέτηση $\rightarrow$ "Γραμμή ή Πολύγωνο", και σχεδιάστε τις τέσσερις γραμμές που θα αποτελούν το ορθογώνιο των ορίων. Δηλαδή σχεδιάστε μία γραμμή από το (-19,-9) έως το (13,-9) μετά μία γραμμή από το (13,-9) έως το (13,9) μετά μία γραμμή από το (13,9) έως το -19,9) και μετά μία γραμμή από το (-19,9) έως το -19,-9). Όλες οι συντεταγμένες είναι σχετικές ως προς το σημείο που επιλέξαμε.

%Σε αυτό το σημείο αντιστοιχούν τα αρχεία από το αρχείο kicad\_tut13.zip.

\subsubsection{Προσθήκη ζώνης}
Αφού έχουμε σχεδιάσει τα όρια της πλακέτας το πρώτο πράγμα που πρέπει να κάνουμε είναι να σχεδιάσουμε μία ζώνη για τη γείωση του κυκλώματος. 

Βεβαιωθείτε ότι έχετε ορίσει το κέντρο του U1  ως το σημείο έναρξης των σχετικών συντεταγμένων. Επίσης από την μπάρα στο αριστερό μέρος της οθόνης φροντίστε επιλεγμένη η Εμφάνιση γεμισμένων περιοχών σε ζώνες. %\includegraphics[width=.9\textwidth]{img/kicad-main.png} 

Η ζώνη θα είναι στο επίπεδο B.CU και θα έχει συντεταγμένες (-18,8) πάνω αριστερά και (12,8) κάτω δεξιά. %TBD συντεταγμένες

Από το μενού στο πάνω μέρος της οθόνης επιλέξτε Τοποθέτηση $\rightarrow$ Ζώνη και κάντε κλικ στις συντεταγμένες (-18,8). Στο παράθυρο που εμφανίζεται ορίστε τις παρακάτω ρυθμίσεις.

%* show ratsnet
%* enable magnetic tracks/pads

Στη συνέχεια ορίστε το πλέγμα στα 1mm.

\begin{itemize}
    \item επιλέξτε το επίπεδο B.Cu
    \item επιλέξτε το δίκτυο GND
    \item διάκενο 0,2
    \item ελάχιστο πλάτος 0,25
    \item εξομάλυνση γωνιών λοξότμηση με απόσταση 0,25
    \item θερμικές εκτονώσεις διάκενο 0,25 και πάχος ακτίνας 0.5
    \item κλίση περιγράμματος μόνο οριζ., καθ. και 45 μοίρες
\end{itemize}

Αφού ορίσετε τα παραπάνω, πατήστε ΟΚ και θα επιστρέψετε στην αρχική οθόνη του \textenglish{Pcbnew} όπου θα πρέπει να ορίσετε την κάτω δεξιά συντεταγμένη της ζώνης. Αφού την ορίσετε, κάντε δεξί κλικ πάνω στο περίγραμμα της ζώνης και επιλέξτε Ζώνη $\rightarrow$ Γέμισμα ζώνης.

\begin{figure}
  \begin{center}
    %\includegraphics[width=.9\textwidth]{img/kicad-main.png}
    %\caption{Παράθυρο 'Ιδιότητες Χάλκινης Ζώνης'}
    \label{fig:kicad-main}
  \end{center}
\end{figure}

\subsubsection{Προσθήκη αγώγιμων δρόμων}

Αρχικά ορίστε το πλέγμα στα 0,25mm, και ορίστε το F.Cu ως τρέχον επίπεδο.

Στη συνέχεια επιλέξτε το εργαλείο "Προσθήκη δρόμων και via" από την αριστερή μπάρα.

Στη συνέχεια προσθέστε via προς τη ζώνη της γείωσης για όλα τα αποτυπώματα που έχουν έδρα που συνδέεται με τη γείωση (4 via για το U1). 

%  - untoggle temporarily magnetic track/pads for vias inside U1

Στη συνέχεια σχεδιάστε όλους του υπόλοιπους αγώγιμους δρόμους, με οδηγό τις συνδέσεις που εμφανίζει η φωλιά συνδέσεων.

Ειδικά για την κεντρική σύνδεση VCC, προσθέστε στους κανόνες σχεδίασης αγώγιμο δρόμο πλάτους 0,5mm και χρησιμοποιήστε τον για τους δρόμους του δικτύου του VCC.

%TBD δλδ βάζω ολόκληρη κλάση;

%Σε αυτό το σημείο αντιστοιχούν τα αρχεία από το αρχείο kicad\_tut14.zip.

\subsubsection{Προσθήκη ζώνης 3,3V}

Αρχικά ορίστε το πλέγμα στα 1mm.

Στη συνέχεια προσθέστε μία ζώνη (όπως προηγουμένος) με συντεταγμένες (-17,7) πάνω αριστερά και (11,7) κάτω δεξιά, και τις παρακάτω ιδιότητες.

\begin{itemize}
    \item επιλέξτε το επίπεδο F.Cu
    \item επιλέξτε το δίκτυο 3.3V
    \item διάκενο 0,2
    \item ελάχιστο πλάτος 0,25
    \item εξομάλυνση γωνιών λοξότμηση με απόσταση 0,25
    \item θερμικές εκτονώσεις διάκενο 0,25 και πάχος ακτίνας 0.5
    \item κλίση περιγράμματος μόνο οριζ., καθ. και 45 μοίρες
\end{itemize}

Αφού ορίσετε τη ζώνη, κάντε δεξί κλικ πάνω στο περίγραμμα αυτής της ζώνης και επιλέξτε Ζώνη $\rightarrow$ Γέμισμα ζώνης.

Πλέον έχουμε τελειώσει ουσιαστικά τον ηλεκτρικό σχεδιασμό της πλακέτας.

%Σε αυτό το σημείο αντιστοιχούν τα αρχεία από το αρχείο kicad\_tut15.zip.

\subsubsection{Αισθητικές διορθώσεις}

Αν θέλουμε, μπορούμε να κάνουμε αισθητικές διορθώσεις και επεμβάσεις την πλακέτα, όπως τα παρακάτω
\begin{itemize}
    \item Ορισμός ομοιόμορφων μεγεθών σε όλα τα αποτυπώματα για τις ενδείξεις αναφοράς (πλάτος 8, ύψος 7, πάχος 12). Αυτό γίνεται από το μενού Επεξεργασία $\rightarrow$ Ορισμός Μεγεθών για Πεδία Αποτυπωμάτων.
    \item Τοποθέτηση σε ομοιόμορφη θέση (πχ πάνω αριστερά από το αποτύπωμα) για όλες τις ενδείξεις αναφοράς αποτυπωμάτων. Αυτό γίνεται μετακινώντας μία προς μία όλες ενδείξεις αναφοράς.
    \item Τοποθέτηση κειμένων γραφικών όπου θέλουμε στην πλακέτα, με το όνομα του κυκλώματος, την ένδειξη copyright, κα. Αυτό γίνεται από το μενού Τοποθέτηση $\rightarrow$ Κείμενο και τοποθέτηση όποιου κειμένου θέλουμε στο επίπεδο F.Silks.
\end{itemize}

\begin{figure}
  \begin{center}
    %\includegraphics[width=.9\textwidth]{img/kicad-main.png}
    %\caption{Παράθυρο 'Ορισμός Μεγέθους Κειμένου', με τα προτεινόμενα μεγέθη}
    \label{fig:kicad-main}
  \end{center}
\end{figure}

Σε αυτό το σημείο έχουμε τελειώσει τον σχεδιασμό, και πρέπει να κάνουμε έναν ηλεκτρικό έλεγχο με το εργαλείο του \textenglish{KiCad} που βρίσκεται στο μενού Εργαλεία $\rightarrow$ ΕΚΣ (Έλεγχος Κανόνων Σχεδίου).

Πλέον έχουμε ολοκληρώσει πλήρως τον σχεδιασμό της πλακέτας, και είμαστε έτοιμοι να δημιουργήσουμε αρχεία παραγωγής τα οποία θα δώσουμε σε ένα εργοστάσιο και θα μας κατασκευάσει την πλακέτα.

\begin{figure}
  \begin{center}
    \includegraphics[width=.9\textwidth]{img/final-pcb.png}
    \caption{Τελικό κύκλωμα πλακέτας}
    \label{fig:final-pcb}
  \end{center}
\end{figure}

%%Σε αυτό το σημείο αντιστοιχούν τα αρχεία από το αρχείο kicad\_tut16.zip.
%Σε αυτό το σημείο αντιστοιχούν τα αρχεία από το αρχείο kicad\_tut17.zip.

\section{Κατασκευή πλακέτας}
Σε αυτό το κεφάλαιο έχουμε έτοιμο το κύκλωμα και θέλουμε δημιουργήσουμε αρχεία παραγωγής τα οποία θα δώσουμε σε ένα εργοστάσιο και θα μας κατασκευάσει την πλακέτα. 

Τα αρχεία παραγωγής είναι τριών ειδών.

\begin{itemize}
    \item αρχεία gerber, που δηλώνουν κυρίως που θα μπει πάστα συγκόλλησης στην πλακέτα (και πολλές ακόμα πληροφορίες)
    \item αρχεία διάτρησης, που δηλώνουν κυρίως που θα γίνουν μηχανικές τρύπες στην πλακέτα
    \item αρχεία τοποθέτησης, που δηλώνουν κυρίως τη θέση των υλικών πάνω στην πλακέτα
\end{itemize}

Αξίζει να σημειωθεί ότι πολλές από τις επιλογές που θα κάνουμε κατά τη δημιουργία όλων των αρχείων ίσως χρειαστεί να τις ορίσετε αφού επικοινωνήσετε με το εργοστάσιο ή την εταιρεία που θα κατασκευάσει την πλακέτα καθώς μπορεί τα μηχανήματα ή οι διαδικασίες τους να απαιτούν κάποια ειδική ρύθμιση για τα αρχεία. Εμείς έχουμε κάνει τις πιο συνηθισμένες ρυθμίσεις και επιλογές.

Πριν φτιάξουμε τα αρχεία θα πρέπει να κάνουμε μία τελευταία επέμβαση (αισθητική) στο κύκλωμα, προσθέτοντας ενδείξεις διαστάσεων στην πλακέτα μας για το συνολικό πλάτος (32mm) και ύψος (18mm) της πλακέτας. Αυτό γίνεται από το μενού Τοποθέτηση $\rightarrow$ Διάσταση.

\begin{figure}
  \begin{center}
    %\includegraphics[width=.9\textwidth]{img/kicad-main.png}
    %\caption{Κύκλωμα σε σημειωμένες τις αποστάσεις}
    \label{fig:kicad-main}
  \end{center}
\end{figure}

\subsection{Αρχεία gerber}
Για να δημιουργήσουμε τα αρχεία gerber της πλακέτας, επιλέγουμε από το μενού Αρχείο $\rightarrow$ Σχεδιογράφηση, και στο παράθυρο που εμφανίζεται ορίζουμε τις παρακάτω επιλογες.

\begin{itemize}
    \item Μορφή σχεδιογράφησης: Gerber
    \item Επιλογή όλων των επιπέδων
    \item Επιλογή του "Σχεδιογράφηση αναφορών αποτυπωμάτων"
    \item Επιλογή του "Αποκλεισμός επιπέδου ορίων pcb από άλλα επίπεδα"
\end{itemize}

\begin{figure}
  \begin{center}
    %\includegraphics[width=.9\textwidth]{img/kicad-main.png}
    %\caption{Παράθυρο 'Σχεδιογράφηση'}
    \label{fig:kicad-main}
  \end{center}
\end{figure}

Αφού ορίσουμε τις επιλογές, κάνουμε κλικ στο Σχεδιογράφηση, και το \textenglish{Pcbnew} παράγει όλα τα απαραίτητα αρχεία gerber για την παραγωγή της πλακέτας.

Με αυτό τον τρόπο θα δημιουργηθούν τα παρακάτω αρχεία και θα τοποθετηθούν μέσα στο φάκελο μαζί με τα άλλα αρχεία του έργου.

\begin{itemize}
    \item usb2uart-F\_Cu.gtl
    \item usb2uart-B\_Cu.gbl
    \item usb2uart-F\_Paste.gtp
    \item usb2uart-F\_SilkS.gto
    \item usb2uart-B\_Mask.gbs
    \item usb2uart-F\_Mask.gts
    \item usb2uart-Dwgs\_User.gbr
    \item usb2uart-Edge\_Cuts.gm1
\end{itemize}

%TBD 4.5 και 4.6 ποια η διαφορά;
\subsection{Αρχεία διάτρησης}
Για να δημιουργήσουμε τα αρχεία διάτρησης της πλακέτας, επιλέγουμε από το μενού Αρχείο $\rightarrow$ Έξοδοι Παραγωγής $\rightarrow$ Αρχείο Διατρήσεων, και στο παράθυρο που εμφανίζεται ορίζουμε Μορφή Αρχείου Χάρτη Διατρήσεων: Gerber, και στην συνέχεια κάνουμε κλικ στο κουμπί Αρχείο Διατρήσεων και μετά κλικ στο κουμπί Αρχείο Χάρτη. Αφού το κάνουμε αυτό, κάνουμε κλικ στο Κλείσιμο.

\begin{figure}
  \begin{center}
    %\includegraphics[width=.9\textwidth]{img/kicad-main.png}
    %\caption{Παράθυρο 'Παραγωγή Αρχείων Διάτρησης'}
    \label{fig:kicad-main}
  \end{center}
\end{figure}

Με αυτό τον τρόπο θα δημιουργηθούν τα αρχεία usb2uart.drl και usb2uart-drl\_map.gbr και θα τοποθετηθούν μέσα στο φάκελο μαζί με τα άλλα αρχεία του έργου.

\subsection{Αρχεία τοποθέτησης}
Για να δημιουργήσουμε τα αρχεία τοποθέτησης της πλακέτας, επιλέγουμε από το μενού Αρχείο $\rightarrow$ Έξοδοι Παραγωγής $\rightarrow$ Αρχείο Θέσης Αποτυπωμάτων, και στο παράθυρο που εμφανίζεται επιλέγουμε ένα αρχείο ανά πλακέτα και την επιλογή "Εξαναγκασμός ιδιότητας INSERT για όλα τα αποτυπώματα SMD. Αφού το κάνουμε αυτό, κάνουμε κλικ στο OK.

Με αυτό τον τρόπο θα δημιουργηθούν τα αρχεία usb2uart-top.pos και usb2uart-bottom.pos και θα τοποθετηθούν μέσα στο φάκελο μαζί με τα άλλα αρχεία του έργου.

\subsection{Επίλογος}
Πλέον έχουμε δημιουργήσει όλα τα απαραίτητα αρχεία που πρέπει να δώσουμε σε ένα εργοστάστιο για να κατασκευάσει την πλακέτα μας.

Τα αρχεία που συνηθίζεται να δινουμε σε ένα εργοστάσιο παραγωγής είναι τα παρακάτω.

\begin{itemize}
    \item αρχεία gerber
    \item αρχεία διάτρησης
    \item αρχεία τοποθέτησης
    \item αρχείο pdf με το σχηματικό μας κύκλωμα (παράχθηκε σε προηγούμενο κεφάλαιο)
    \item λίστα υλικών (παράχθηκε σε προηγούμενο κεφάλαιο)
\end{itemize}

%Σε αυτό το σημείο αντιστοιχούν τα αρχεία από το αρχείο kicad\_tut18.zip.

%* start gerbview and explore gerber+drill files
%  - note mistake in soldermask
  
\end{document}


