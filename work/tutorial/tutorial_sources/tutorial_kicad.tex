\documentclass[a4paper]{article}

\usepackage{fontspec}
%για επιλογή γραμματοσειρών (μεταξύ άλλων)
\usepackage{polyglossia}
%για συλλαβισμό λέξεων (μεταξύ άλλων)
\usepackage{geometry}
%για μορφή σελίδας a4 (μεταξύ άλλων)

\setdefaultlanguage{greek}
\setotherlanguage{english}

\setmainfont{DejaVu Sans}

\title{Tutorial - Οδηγός Εκμάθησης KiCad}
\author{Μιχάλης Μισιρλής}
\date{Οκτώβριος 2015}

\begin{document}

\maketitle

\section{Εισαγωγή}
Αυτός ο οδηγός εκμάθησης έχει ως στόχο την εξοικοίωση του Έλληνα χρήστη του KiCad μέσα από την ανάπτυξη ενός απλού αλλά πλήρους κυκλώματος (ηλεκτρονική συσκευή) ξεκινώντας από την βασική ιδέα και το σκοπό της συσκευής και καταλήγοντας στην παραγωγή αρχείων κατασκευής ώστε να μπορεί να παραχθεί η συσκευή από οποιοδήποτε εργοστάσιο ή εταιρεία παραγωγής πλακετών.

\subsection{Προαπαιτούμενα}
Αυτός ο οδηγός έχει βασιστεί και απαιτεί το πρόγραμμα KiCad έκδοση 4.0.0 [TBD] με την Ελληνική μετάφραση [TBD]. Επίσης ο οδηγός θεωρεί ότι ο χρήστης είναι εξοικοιωμένος με τις βασικές έννοιες ηλεκτρονικού σχεδιασμού και τις βασικές έννοιες χρήσης ηλεκτρονικών υπολογιστών.

\subsection{Λεξικό όρων}

\subsection{Περιγραφή συσκευής}
Η συσκευή ονομάζεται USB2UART και είναι ένας μετατροπέας USB σε 3.3V UART, βασισμένος στο ολοκληρωμένο CP2104 από την εταιρεία Silicon Labs. υποστηρίζεται από τα περισσότερα σύγχρονα λειτουργικά συστήματα ως εικονική σειριακή θύρα. Συνδέεται με έναν standard type B connector από τη μία πλευρά και ένα four-pin header από την άλλη. Περιλαμβάνει power status LED και RX/TX LED.








\end{document}
