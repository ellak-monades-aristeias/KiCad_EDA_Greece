\documentclass[a4paper]{article}

\usepackage{fontspec}
%για επιλογή γραμματοσειρών (μεταξύ άλλων)
\usepackage{polyglossia}
%για συλλαβισμό λέξεων (μεταξύ άλλων)
\usepackage{geometry}
%για μορφή σελίδας a4 (μεταξύ άλλων)
\usepackage{longtable}
%για προχωρημένους πίνακες
\usepackage{tabu}
%για προχωρημένους πίνακες
\usepackage{hyperref}
\usepackage{comment}

\setdefaultlanguage{greek}
\setotherlanguage{english}

\setmainfont{DejaVu Sans}

\title{Tutorial - Οδηγός Εκμάθησης KiCad}
\author{Μιχάλης Μισιρλής}
\date{Οκτώβριος 2015}

\begin{document}

\maketitle

\section{Εισαγωγή}
Αυτός ο οδηγός εκμάθησης έχει ως στόχο την εξοικείωση του Έλληνα χρήστη του KiCad μέσα από την ανάπτυξη ενός απλού αλλά πλήρους κυκλώματος (ηλεκτρονική συσκευή) ξεκινώντας από την βασική ιδέα και το σκοπό της συσκευής και καταλήγοντας στην παραγωγή αρχείων κατασκευής ώστε να μπορεί να παραχθεί η συσκευή από οποιοδήποτε εργοστάσιο ή εταιρεία παραγωγής πλακετών.

\subsection{Προαπαιτούμενα}
\label{sec:prereq}
Αυτός ο οδηγός έχει βασιστεί και απαιτεί το πρόγραμμα KiCad έκδοση 4.0.0 [TBD] με την Ελληνική μετάφραση [TBD]. Επίσης ο οδηγός θεωρεί ότι ο χρήστης είναι εξοικειωμένος με τις βασικές έννοιες ηλεκτρονικού σχεδιασμού και μπορεί να σχεδιάσει πλήρως ένα κύκλωμα στο χαρτί ή σε κάποιο άλλο πρόγραμμα. Αυτός ο οδηγός δεν έχει ως στόχο τη διδασκαλία ηλεκτρονικού σχεδιασμού, αλλά τη διδασκαλία του πως γίνεται ο ηλεκτρονικός σχεδιασμός με το KiCad.
Τέλος ο οδηγός θεωρεί ότι ο χρήστης είναι εξοικειωμένος τις βασικές έννοιες χρήσης ηλεκτρονικών υπολογιστών, όπως επιλογή μενού, έξοδος από ένα παράθυρο, επιλογή πολλαπλών στοιχείων κα

\subsection{Λεξικό όρων}
\begin{longtabu}{ c c }
    Ελληνικός όρος	&	Αγγλικός όρος	\\
    3Δ προβολή	&	3d viewer	\\
    αγκύρωση	&	anchor	\\
    ακροδέκτης	&	pin	\\
    ακτίνα	&	spoke	\\
    ακτίνα	&	spoke	\\
    αλφαριθμητικό	&	string	\\
    ανάλυση	&	parse	\\
    αναφορά, ονομασία αναφοράς	&	reference	\\
    αντεστραμμένο	&	inverted	\\
    αντικολλητικής μάσκας	&	solder resist mask	\\
    αντίσταση συγκόλλησης	&	solder resist	\\
    απόδοση	&	render	\\
    απόκλιση	&	skew	\\
    αποτύπωμα	&	footprint	\\
    γομα	&	eraser	\\
    γραμμή γραφικών	&	graphic line	\\
    διάκενο	&	clearance	\\
    διαμπερές	&	through (hole)	\\
    διάτρηση	&	drill	\\
    δίαυλος	&	bus	\\
    δίκτυα μίας έδρας	&	single pad nets	\\
    δρομολόγηση	&	route	\\
    δρόμος	&	pcb trace / track	\\
    έδρα	&	antipad clearance	\\
    έδρα	&	pad	\\
    εηκ	&	erc	\\
    εκς	&	drc	\\
    ελεγκτής ηλεκτρικών κανόνων	&	electrical rules checker	\\
    εξάρτημα	&	component	\\
    έξοδος	&	output	\\
    επάνω πλευρά	&	top side	\\
    επεξεργαστής σχεδίασης σελίδας	&	page layout editor	\\
    επιλογές	&	options / choices / selections	\\
    επίπεδα	&	layers	\\
    επίπεδα eco	&	eco layers	\\
    επίπεδο σχεδίων	&	drawings layer	\\
    επίπεδοα επικόλλησης	&	adhesive layers	\\
    επισήμανση	&	flag	\\
    ζώνες	&	zones	\\
    θαμμένο via	&	buried via	\\
    θύρα	&	port	\\
    θύρα ισχύος	&	power port	\\
    κανόνες σχεδιασμού	&	design rule	\\
    κατοπτρισμός, καθρέφτισμα	&	mirror	\\
    κάτω πλευρά	&	bottom side	\\
    κενό	&	gap	\\
    κοίλες γωνίες	&	fillet	\\
    κώδικας g	&	g code	\\
    λειτουργία γεμίσματος	&	filled mode	\\
    λειτουργία περιγράμματος	&	sketch mode	\\
    λειτουργία περιγράμματος	&	outline mode	\\
    λίστα δικτύων	&	netlist	\\
    λίστα υλικών	&	bill of materials	\\
    λοξότμηση	&	chamfer	\\
    μάσκα συγκόλλησης	&	solder mask	\\
    μέρη	&	unit	\\
    μεταξοτυπία	&	silk screen	\\
    Μετατόπιση \& σπρώξιμο	&	Push \& shove	\\
    μετατόπιση,θέση	&	offset	\\
    μηχανές ανύψωσης και τοποθέτησης	&	“pick and place”	\\
    μικρο-ρύθμιση, ρύθμιση	&	tune	\\
    μπροστινή όψη	&	front view	\\
    ονοματοδοσία	&	annotate	\\
    όρια πλακέτας	&	pcb/board edge	\\
    παλαιότερου τύπου	&	legacy	\\
    παραπομπή πλαισίου	&	frame reference	\\
    πάστα συγκόλλησης	&	solder paste	\\
    περιγράμμα πλακέτας	&	board outline	\\
    πίνακας	&	matrix	\\
    πίσω όψη	&	back view	\\
    πλαίσιο	&	box (για ui)	\\
    πλοηγός	&	navigator	\\
    πρόγραμμα προβολής pdf	&	pdf viewer	\\
    προκαθορισμένου	&	default	\\
    πρόσθετα	&	plugin	\\
    προτιμώμενο	&	favourite	\\
    ρυθμίσεις	&	settings	\\
    ρυθμίσεις	&	configurations	\\
    σκιαγράφηση	&	hatch	\\
    στελέχη	&	stubs	\\
    στοίβα έδρας	&	padstack	\\
    σύγκρουση, ασυμβατότητα	&	conflict	\\
    συντομεύσεις	&	hotkeys	\\
    συνώνυμο	&	alias	\\
    σύρμα	&	wire	\\
    σύρσιμο	&	drag	\\
    συστοιχία	&	array	\\
    σχεδίαση	&	design	\\
    σχεδίασης σελίδας	&	page layout	\\
    σχέδιο	&	drawing	\\
    σχεδιογραφώ	&	plot	\\
    σχηματικό διάγραμμα, σχηματικό	&	schematic	\\
    τόξο	&	arc	\\
    τρύπα	&	hole	\\
    φάκελος	&	folder / directory	\\
    φωλιά συνδέσεων	&	ratsnest	\\
    χαλκός	&	copper pour	\\
    χώρος αποτυπώματος	&	courtyard	\\
    χώρος σχεδίασης	&	canvas	\\
    ψευδώνυμο	&	nickname	\\
    excellon	&	excellon	\\
    excellon	&	excellon	\\
    gerber	&	gerber	\\
    NPTH	&	NPTH	\\
    PTH	&	PTH	\\
    SMD	&	SMD	\\
    VIA	&	VIA	\\
\end{longtabu}

\subsection{Περιγραφή συσκευής}
Η συσκευή ονομάζεται USB2UART και είναι ένας μετατροπέας USB σε 3.3V UART, βασισμένος στο ολοκληρωμένο CP2104 από την εταιρεία Silicon Labs. υποστηρίζεται από τα περισσότερα σύγχρονα λειτουργικά συστήματα ως εικονική σειριακή θύρα. Συνδέεται με έναν standard type B connector από τη μία πλευρά και ένα four-pin header από την άλλη. Περιλαμβάνει power status LED και RX/TX LED.

Το τελικό σχηματικό κύκλωμα της συσκευής και το τελικό σχέδιο της πλακέτας φαίνονται παρακάτω.

\begin{figure}
  \begin{center}
    %\includegraphics{img/kicad-main.png}
    %\caption{Τελικό σχηματικό κύκλωμα της συσκευής}
    \label{fig:kicad-main}
  \end{center}
\end{figure}

\begin{figure}
  \begin{center}
    %\includegraphics{img/kicad-main.png}
    %\caption{Τελικό σχέδιο της πλακέτας της συσκευής}
    \label{fig:kicad-main}
  \end{center}
\end{figure}

\subsection{Βασικές Έννοιες}
Το KiCad είναι μία ολοκληρωμένη σουίτα εφαρμογών σχεδίασης ηλεκτρονικών κυκλωμάτων EDA (Electronic Design Automation). Με το KiCad είναι εφικτή η σχεδίαση σχηματικών και τυπωμένων ηλεκτρονικών κυκλωμάτων, χρησιμοποιώντας διαφορετικές εφαρμογές για κάθε δουλειά.

Τα σχηματικά κυκλώματα σχεδιάζονται στο KiCad με την εφαρμογή EEschema. Αποτελούνται από εξαρτήματα (για παράδειγμα ένας πυκνωτής ή ένα ολοκληρωμένο) συνδέσεις μεταξύ τους, και άλλα στοιχεία. Τα σχηματικά κυκλώματα είναι οργανωμένα σε σχηματικά φύλλα. Κάθε σχηματικό φύλλο αποτελεί και ένα αρχείο στον υπολογιστή του χρήστη.

Τα εξαρτήματα των σχηματικών μπορεί να ανήκουν σε μία βιβλιοθήκη του KiCad ή μπορεί να τα έχει σχεδιάσει από το μηδέν ο χρήστης χρησιμοποιώντας το σχετικό εργαλείο του KiCad. Κάθε εξάρτημα αποτελεί και ένα αρχείο στον υπολογιστή του χρήστη.

Τα τυπωμένα κυκλώματα σχεδιάζονται στο KiCad με την εφαρμογή PCBnew. Αποτελούνται από πλακέτες που περιέχουν αποτυπώματα εξαρτημάτων, συνδέσεις μεταξύ τους μέσω δρόμων, και από τρύπες, via, κ.α.
Κάθε πλακέτα αποτελεί και ένα αρχείο στον υπολογιστή του χρήστη. 

Τα αποτυπώματα εξαρτημάτων μπορεί να ανήκουν σε μία βιβλιοθήκη του KiCad ή μπορεί να τα έχει σχεδιάσει από το μηδέν ο χρήστης χρησιμοποιώντας το σχετικό εργαλείο του KiCad. Κάθε αποτύπωμα αποτελεί και ένα αρχείο στον υπολογιστή του χρήστη. 

Τα αποτυπώματα μπορεί να έχουν τρισδιάστατη απεικόνιση. Οι τρισδιάστατες απεικονίσεις μπορεί να ανήκουν σε μία βιβλιοθήκη του KiCad ή μπορεί να τα έχει σχεδιάσει από το μηδέν ο χρήστης χρησιμοποιώντας κάποιο εργαλείο τρισδιάστατης σχεδίασης. Κάθε τρισδιάστατη απεικόνιση αποτελεί και ένα αρχείο στον υπολογιστή του χρήστη. 

Το σύνολο του κυκλώματος (σχηματικό, πλακέτα) οργανώνεται από το KiCad σε έργα. Κάθε έργο αποτελεί και ένα αρχείο στον υπολογιστή του χρήστη. 

Όλα τα παραπάνω αρχεία (εξαρτήματα, αποτυπώματα, κυκλώματα, έργα, κοκ) καλό είναι να τα διατηρούμε στον ίδιο φάκελο αρχείων του υπολογιστή μας, για λόγους οργάνωσης.

\section{Δημιουργία Κυκλώματος}

\subsection{Δημιουργία έργου}
Εκτελέστε το πρόγραμμα KiCad. Δείτε το σχετικό κεφάλαιο~\ref{sec:prereq} για την έκδοση του KiCad που πρέπει να εκτελέσετε. Θα εμφανιστεί η κεντρική οθόνη του προγράμματος KiCad. Από το μενού στο πάνω μέρος της οθόνης επιλέγουμε Αρχείο$\rightarrow$Νέο Έργο$\rightarrow$Νέο Έργο. Δίνουμε στο νέο έργο (και το σχετικό αρχείο) ένα όνομα, έστω usb2uart. Προτείνεται να δημιουργηθεί το αρχείο του έργου σε έναν φάκελο του υπολογιστή σας ο οποίος θα είναι αφιερωμένος στο συγκεκριμένο έργο.

\begin{figure}
  \begin{center}
    %\includegraphics{img/kicad-main.png}
    %\caption{Κεντρική οθόνη του προγράμματος KiCad}
    \label{fig:kicad-main}
  \end{center}
\end{figure}

\subsection{Δημιουργία σχηματικού}
Από το μενού στο πάνω μέρος της οθόνης επιλέγουμε Εργαλεία$\rightarrow$Εκτέλεση EEschema, για να εκτελεστεί η εφαρμογή EEschema με την οποία σχεδιάζουμε το σχηματικό κύκλωμα. Εάν εμφανιστεί κάποια προειδοποίηση, αποδεχτείτε την (TBD). Θα δημιουργηθούν δύο αρχεία στον υπολογιστή σας, τα usb2uart.sch και usb2uart-cache.lib. Εκτελέστε ξανά την εφαρμογή ώστε να μην εμφανιστεί κάποια προειδοποίηση (TBD).

Θα εμφανιστεί στην οθόνη σας η κεντρική σελίδα της εφαρμογής EEschema. Αυτή αποτελείται από τα παρακάτω
\begin{itemize}
    \item ένα φύλλο σχηματικού κυκλώματος στο κέντρο της οθόνης
    \item γραμμή μενού στο πάνω μέρος της οθόνης
    \item μπάρα βασικών λειτουργιών στο πάνω μέρος της οθόνης
    \item μπάρα με γενικές λειτουργίες στο αριστερό μέρος της οθόνης
    \item μπάρα με συγκεκριμένες λειτουργίες σχεδίασης στο δεξί μέρος της οθόνης
\end{itemize}
Αξίζει να σημειωθεί πως όπως και στις περισσότερες εφαρμογές, έτσι και στο KiCad και όλες τις εφαρμογές του, οι περισσότερες λειτουργίες μπορούν να γίνουν με δύο τρόπους: είτε επιλέγοντας κάτι στο μενού είτε κάνοντας κλικ στο σχετικό κουμπί μίας μπάρας.

\begin{figure}
  \begin{center}
    %\includegraphics{img/kicad-main.png}
    %\caption{Κεντρική σελίδα της εφαρμογής EEschema}
    \label{fig:kicad-main}
  \end{center}
\end{figure}

Το EEschema μπορεί να σχεδιάσει κυκλώματα που αποτελούνται από πολλά σχηματικά φύλλα. Σε αυτό το tutorial το κύκλωμά μας αποτελείται από ένα και μοναδικό φύλλο.

Από το μενού στο πάνω μέρος της οθόνης του EEschema επιλέξτε Αρχείο$\rightarrow$Ρυθμίσεις Σελίδας. Στο παράθυρο που εμφανίζεται, ορίστε τις ρυθμίσεις της σελίδας σας, όπως μέγεθος χαρτιού, ημερομηνία έκδοσης, τίτλος κυκλώματος, κα και πατήστε ΟΚ. Αυτές οι ρυθμίσεις δεν έχουν κάποια ηλεκτρική ή λειτουργική σημασία για το κύκλωμα, είναι όμως χρήσιμες πληροφορίες για την οργάνωση των κυκλωμάτων σας.

\begin{figure}
  \begin{center}
    %\includegraphics{img/kicad-main.png}
    %\caption{Ρυθμίσεις Σελίδας}
    \label{fig:kicad-main}
  \end{center}
\end{figure}

Από το μενού στο πάνω μέρος της οθόνης του EEschema επιλέξτε "Τοποθέτηση"$\rightarrow$"Γραφικό Πολυγραμμής".

Με αυτό το εργαλείο μπορείτε να σχεδιάσετε γραμμές στο φύλλο σας. Αυτές οι γραμμές γραφικών δεν έχουν κάποια ηλεκτρική ή λειτουργική σημασία για το κύκλωμα, είναι απλή "ζωγραφική", είναι όμως χρήσιμες για την οργάνωση των κυκλωμάτων σας.

Σχεδιάστε στο φύλλο μία κατακόρυφη γραμμή που χωρίζει το φύλλο σε δύο τμήματα. Το δεξί τμήμα πρέπει να είναι περίπου το ένα τρίτο του συνολικού φύλλου. Αυτό το δεξί τμήμα θα το χρησιμοποιήσουμε ως χώρο για να γράφουμε  βοηθητικές πληροφορίες, και για να εναποθέτουμε προχωρινά τα εξαρτήματα πριν τα τοποθετήσουμε στην οριστική τους θέση στο κύκλωμα που θα σχεδιάσουμε στο αριστερό τμήμα της σελίδας.

\begin{figure}
  \begin{center}
    %\includegraphics{img/kicad-main.png}
    %\caption{Σχηματικό φύλλο χωρισμένο στα δύο}
    \label{fig:kicad-main}
  \end{center}
\end{figure}

Η εφαρμογή EEschema περιλαμβάνει πολλά εξαρτήματα, τα οποία μπορείτε να χρησιμοποιήσετε στα κυκλώματά σας. Τα εξαρτήματα αυτά είναι οργανωμένα σε βιβλιοθήκες. Το EEschema περιλαμβάνει αρχικά περίπου 30 βιβλιοθήκες οι οποίες περιλαμβάνονται εξαρχής σε κάθε νέο έργο που δημιουργείται. Για λόγους απλότητας του έργου μας, εμείς θα αφαιρέσουμε τις βιβλιοθήκες εξαρτημάτων που δεν έχουν εξαρτήματα χρήσιμα για το συγκεκριμένο κύκλωμα που σχεδιάζουμε.

Από το μενού στο πάνω μέρος της οθόνης του EEschema επιλέξτε "Προτιμήσεις"$\rightarrow$"Βιβλιοθήκες Εξαρτημάτων". Θα εμφανιστεί ένα παράθυρο με τις βιβλιοθήκες του έργου. Σε αυτό το παράθυρο επιλέξτε όλες τις βιβλιοθήκες (μία προς μία ή όλες μαζί) εκτός από τις βιβλιοθήκες power, device και conn και κάντε κλικ στο "Αφαίρεση". Θα πρέπει στο παράθυρο να φαίνονται μόνο οι τρεις βιβλιοθήκες power, device και conn. Κάντε κλικ στο "ΟΚ". Θα επανέλθετε στην κεντρική σελίδα της εφαρμογής EEschema.

Από το μενού στο πάνω μέρος της οθόνης του EEschema επιλέξτε "Αρχείο"$\rightarrow$"Αποθήκευση Σχηματικού Έργου".

\begin{figure}
  \begin{center}
    %\includegraphics{img/kicad-main.png}
    %\caption{Παράθυρο με τις βιβλιοθήκες του έργου}
    \label{fig:kicad-main}
  \end{center}
\end{figure}

Σε αυτή τη φάση έχουμε δημιουργήσει τα αρχεία kicad\_tut00.zip.

\subsection{Προσθήκη εξαρτημάτων}
Εφόσον έχουμε δημιουργήσει και ρυθμίσει το φύλλο μας, πρέπει να προσθέσουμε τα εξαρτήματα που θα αποτελέσουν το κύκλωμά μας. Πρέπει να βρισκόμαστε στην κεντρική σελίδα της εφαρμογής EEschema.

Από το μενού στο πάνω μέρος της οθόνης του EEschema επιλέξτε "Τοποθέτηση"$\rightarrow$"Εξάρτημα" και κάντε κλικ στο δεξί τμήμα του φύλλου σας. Θα εμφανιστεί ένα παράθυρο επιλογής εξαρτήματος.

\begin{figure}
  \begin{center}
    %\includegraphics{img/kicad-main.png}
    %\caption{Παράθυρο επιλογής εξαρτήματος}
    \label{fig:kicad-main}
  \end{center}
\end{figure}

Στο παράθυρο επιλογής εξαρτήματος επιλέξτε το εξάρτημα με όνομα "C" (ένας πυκνωτής) από τη βιβλιοθήκη "device", πατήστε ΟΚ (θα επανέλθετε το φύλλο σχηματικού) και κάντε κλικ στο δεξί τμήμα του φύλλου σας ώστε να τοποθετηθεί στο φύλλο το εξάρτημα που επιλέξατε. 

Αφού τοποθετήσετε το εξάρτημα με όνομα "C" στο φύλλο, ακολουθήστε την ίδια διαδικασία ώστε να τοποθετήσετε στο φύλλο σας (στο δεξί τμήμα) όλα τα παρακάτω εξαρτήματα. 

\begin{itemize}
    \item R, από τη βιβλιοθήκη device
    \item LED, από τη βιβλιοθήκη device
    \item VCC, από τη βιβλιοθήκη power
    \item +3.3V, από τη βιβλιοθήκη power
    \item GND, από τη βιβλιοθήκη power
    \item CONN\_01X04, από τη βιβλιοθήκη conn
\end{itemize}

Με την ολοκλήρωση αυτών των τοποθετήσεων, έχουμε στο φύλλο μας 7 εξαρτήματα, τοποθετημένα όλα τακτοποιημένα στο δεξί τμήμα του φύλλου.

\begin{figure}
  \begin{center}
    %\includegraphics{img/kicad-main.png}
    %\caption{Φύλλο σχηματικού με τοποθετημένα 7 εξαρτήματα}
    \label{fig:kicad-main}
  \end{center}
\end{figure}

Για λόγους ευκολίας ανάγνωση του κυκλώματος, θα χρειαστεί να αλλάξουμε το πρόθεμα της ονομασίας αναφοράς του CONN\_01X04 από P σε J. Για να το κάνουμε αυτό πρέπει να κάνουμε δεξί κλικ πάνω στο εξάρτημα CONN\_01X04, και από το μενού που εμφανίζεται να επιλέξουμε "Επεξεργασία Εξαρτήματος"$\rightarrow$"Επεξεργασία". Αυτό θα μας εμφανίσει το παράθυρο "Ιδιότητες Εξαρτήματος", όπου πρέπει να ορίσουμε την Τιμή Πεδίου της Ονομασίας Αναφοράς να είναι J? και όχι P?.

\begin{figure}
  \begin{center}
    %\includegraphics{img/kicad-main.png}
    %\caption{Παράθυρο "Ιδιότητες Εξαρτήματος"}
    \label{fig:kicad-main}
  \end{center}
\end{figure}


Στη συνέχεια θα χρειαστεί να τοποθετήσουμε στο κύκλωμά μας και άλλα εξαρτήματα C, R και CONN\_01X04. Για να το κάνουμε αυτό, αντί να κάνουμε πάλι τοποθέτηση και να τα επιλέγουμε από τις βιβλιοθήκες, μπορούμε να κάνουμε αντιγραφή των εξαρτημάτων που ήδη έχουμε στο φύλλο μας. 

Για να αντιγράψουμε (δηλαδή να φτιάξουμε ακόμα ένα αντίγραφο) ενός εξαρτήματος πρέπει να κάνουμε δεξί κλικ πάνω στο εξάρτημα, από το μενού που εμφανίζεται να επιλέξουμε "Αντιγραφή", και να κάνουμε κλικ πάνω στο φύλλο εκεί όπου θέλουμε να φιταχτεί το αντίγραφο του εξαρτήματος. 

\begin{figure}
  \begin{center}
    %\includegraphics{img/kicad-main.png}
    %\caption{Αντιγραφή Εξαρτήματος}
    \label{fig:kicad-main}
  \end{center}
\end{figure}

Με αυτό τον τρόπο πρέπει να φτιάξουμε τα παρακάτω

\begin{itemize}
    \item 3 αντίγραφα του C, ώστε να έχουμε συνολικά 4 C στο φύλλο
    \item 2 αντίγραφα του R, ώστε να έχουμε συνολικά 3 R στο φύλλο
    \item 1 αντίγραφο του CONN\_01X04, ώστε να έχουμε συνολικά 2 CONN\_01X04 στο φύλλο
\end{itemize}

Με την ολοκλήρωση αυτών των τοποθετήσεων, έχουμε στο φύλλο μας 13 εξαρτήματα, τοποθετημένα όλα τακτοποιημένα στο δεξί τμήμα του φύλλου.

\begin{figure}
  \begin{center}
    %\includegraphics{img/kicad-main.png}
    %\caption{Φύλλο σχηματικού με τοποθετημένα 13 εξαρτήματα}
    \label{fig:kicad-main}
  \end{center}
\end{figure}

Σε αυτή τη φάση έχουμε δημιουργήσει τα αρχεία kicad\_tut01.zip.

\subsection{Δημιουργία εξαρτημάτων}
Το EEschema μας δίνει τη δυνατότητα να επεξεργαστούμε ένα υπάρχον εξάρτημα ή και να δημιουργήσουμε ένα εξάρτημα από το μηδέν.

Εμείς σε αυτή τη φάση θέλουμε να εντάξουμε στο σχηματικό κύκλωμα το ολοκληρωμένο CP2104. Το KiCad όμως δεν έχει στις βιβλιοθήκες του εξάρτημα που να αντιστοιχεί στο ολοκληρωμένο CP2104, οπότε θα το δημιουργήσουμε.

Από το μενού στο πάνω μέρος της οθόνης του EEschema επιλέξτε Εργαλεία$\rightarrow$Επεξεργαστής Βιβλιοθήκης.

Θα εμφανιστεί στην οθόνη σας η κεντρική σελίδα του Επεξεργαστή Βιβλιοθήκης Εξαρτημάτων. Αυτή αποτελείται από τα παρακάτω
\begin{itemize}
    \item έναν κενό χώρο στο κέντρο της οθόνης
    \item γραμμή μενού στο πάνω μέρος της οθόνης
    \item μπάρα βασικών λειτουργιών στο πάνω μέρος της οθόνης
    \item μπάρα με γενικές λειτουργίες στο αριστερό μέρος της οθόνης
    \item μπάρα με συγκεκριμένες λειτουργίες σχεδίασης στο δεξί μέρος της οθόνης
\end{itemize}

Στον κενό χώρο στο κέντρο της οθόνης θα σχεδιάσουμε το σώμα του εξαρτήματος, θα προσθέσουμε ακροδέκτες, θα γράψουμε το όνομά του, κλπ. Όλα αυτά τα στοιχεία αφού τα προσθέσουμε στο σώμα του εξαρτήματος, μπορούμε να τα επιλέγουμε με δεξί κλικ του ποντικιού και να τα επεξεργαζόμαστε- κυρίως να τα μετακινούμε και να τα περιστρέφουμε.

\begin{figure}
  \begin{center}
    %\includegraphics{img/kicad-main.png}
    %\caption{Kεντρική σελίδα - Επεξεργαστής Βιβλιοθήκης Εξαρτημάτων}
    \label{fig:kicad-main}
  \end{center}
\end{figure}

Αρχικά θα δημιουργήσουμε ένα νέο εξάρτημα.

Κάντε κλικ στο εικονίδιο "Δημιουργία νέου εξαρτήματος" στην πάνω μπάρα %\includegraphics{img/graphic.png}
. Θα εμφανιστεί το παράθυρο "Ιδιότητες Εξαρτήματος". Στο παράθυρο αυτό, στο "Όνομα εξαρτήματος" γράψτε το όνομα CP2104, αφήστε όλες τις υπόλοιπες επιλογές στις προκαθορισμένες ρυθμίσεις, και επιλέξτε ΟΚ.

\begin{figure}
  \begin{center}
    %\includegraphics{img/kicad-main.png}
    %\caption{Παράθυρο "Ιδιότητες Εξαρτήματος"}
    \label{fig:kicad-main}
  \end{center}
\end{figure}

Πλέον έχουμε δημιουργήσει ένα νέο εξάρτημα, χωρίς κανέναν ακροδέκτη. 

Πρέπει να δημιουργήσουμε μία νέα βιβλιοθήκη, να το αποθηκεύσουμε σε αυτήν, και να εντάξουμε τη βιβλιοθήκη στο έργο μας.

Από την μπάρα στο πάνω μέρος της οθόνης επιλέξτε "Αποθήκευση τρέχοντος εξαρτήματος σε νέα βιβλιοθήκη" %\includegraphics{img/graphic.png}.

Αυτό θα εμφανίσει ένα παράθυρο όπου θα πρέπει να δώσετε το όνομα της νέας βιβλιοθήκης που θέλετε να δημιουργήσετε ώστε να μπει σε αυτή το νέο εξάρτημα. Το όνομα της βιβλιοθήκης θα είναι και το όνομα του αρχείου στον υπολογιστή σας, το οποίο θα περιέχει τη βιβλιοθήκη. Δώστε στη βιβλιοθήκη το όνομα usb2uart.lib και πατήστε Save/Αποθήκευση ώστε το αρχείο της βιβλιοθήκης να αποθηκευτεί στον υπολογιστή σας. 

Θα εμφανιστεί ένα μήνυμα που θα λέει ότι η βιβλιοθήκη πρέπει να δηλωθεί στο EEschema για να χρησιμοποιηθεί. Πατήστε ΟΚ.

Κλείστε τον Επεξεργαστή Βιβλιοθήκης Εξαρτημάτων, εκτελέστε το EEschema (αν δεν τρέχει ήδη) και από το μενού στο πάνω μέρος της οθόνης του EEschema επιλέξτε "Προτιμήσεις"$\rightarrow$"Βιβλιοθήκες Εξαρτημάτων". Θα εμφανιστεί ένα παράθυρο με τις βιβλιοθήκες του έργου. 

Σε αυτό το παράθυρο πατήστε "Προσθήκη", βρείτε στο σύστημα αρχείων του υπολογιστή σας το αρχείο της βιβλιοθήκης που δημιουργήσατε προηγουμένως (usb2uart.lib), και πατήστε Open. Η βιβλιοθήκη usb2uart θα πρέπει να έχει προστεθεί στις βιβλιοθήκες του έργου σας. Πατήστε ΟΚ στο παράθυρο με τις βιβλιοθήκες του έργου.

\begin{figure}
  \begin{center}
    %\includegraphics{img/kicad-main.png}
    %\caption{Παράθυρο με τις βιβλιοθήκες του έργου, με την usb2uart}
    \label{fig:kicad-main}
  \end{center}
\end{figure}

Από το μενού στο πάνω μέρος της οθόνης του EEschema επιλέξτε "Αρχείο"$\rightarrow$"Αποθήκευση Σχηματικού Έργου".

Από το μενού στο πάνω μέρος της οθόνης του EEschema επιλέξτε Εργαλεία$\rightarrow$Επεξεργαστής Βιβλιοθήκης για να συνεχίσετε να δουλεύετε στον Επεξεργαστή Βιβλιοθήκης.

Κάντε κλικ στο εικονίδιο "Επιλογή βιβλιοθήκης εργασίας" στην πάνω μπάρα %\includegraphics{img/graphic.png}
, στο παράθυρο που εμφανίζεται επιλέξτε τη βιβλιοθήκη usb2uart, και πατήστε ΟΚ. Πρέπει να έχετε επιστρέψει στην κεντρική οθόνη του Επεξεργαστή Βιβλιοθήκης.

\begin{figure}
  \begin{center}
    %\includegraphics{img/kicad-main.png}
    %\caption{Παράθυρο επιλογής βιβλιοθήκης εργασίας}
    \label{fig:kicad-main}
  \end{center}
\end{figure}

Τώρα πρέπει, στην κεντρική οθόνη του Επεξεργαστή Βιβλιοθήκης, να φορτώσουμε το εξάρτημα το οποίο θέλουμε να επεξεργαστούμε. 

Κάντε κλικ στο εικονίδιο "Φόρτωση εξαρτήματος για επεξεργασία από την τρέχουσα βιβλιοθήκη" %\includegraphics{img/graphic.png}
. Στο παράθυρο επιλογής εξαρτήματος που θα εμφανιστεί επιλέξτε το εξάρτημα με όνομα "CP2104" από τη βιβλιοθήκη "usb2uart", και πατήστε ΟΚ. Θα επανέλθετε στην κεντρική οθόνη του Επεξεργαστή Βιβλιοθήκης και θα έχει φορτωθεί στην οθόνη το εξάρτημα "CP2104".

Στη συνέχεια θα ορίσουμε κάποιες από τις γενικές ιδιότητες του εξαρτήματος. 

Κάντε κλικ στο εικονίδιο "Επεξεργασία ιδιοτήτων εξαρτήματος" %\includegraphics{img/graphic.png} 
για να εμφανιστεί το παράθυρο Ιδιοτήτων για το εξάρτημα. 

\begin{figure}
  \begin{center}
    %\includegraphics{img/kicad-main.png}
    %\caption{Παράθυρο Ιδιοτήτων για το εξάρτημα}
    \label{fig:kicad-main}
  \end{center}
\end{figure}

Σε αυτό το παράθυρο συμπληρώστε τα παρακάτω πεδία και μετά πατήστε ΟΚ για να επιστρέψετε στην κεντρική οθόνη του Επεξεργαστή Βιβλιοθήκης.

\begin{itemize}
    \item Περιγραφή: CP2104 Μετατροπέας USB-σε-UART
    \item Φίλτρο αποτυπώματος: QFN
    %\item docfile
\end{itemize}

Στην κεντρική οθόνη του Επεξεργαστή Βιβλιοθήκης πρέπει να σχεδιάσουμε αρχικά το σώμα του εξαρτήματος. 

Από το μενού στο πάνω μέρος της οθόνης επιλέξετε Τοποθέτηση$\rightarrow$Ορθογώνιο και σχεδιάστε στην οθόνη ένα κατακόρυφο ορθογώνιο όπως φαίνεται στη σχετική εικόνα. Για να σχεδιάσετε, κάντε κλικ εκεί που θέλετε να είναι η πάνω αριστερή γωνία και μετά ένα ακόμα κλικ εκεί που θέλετε να είναι η κάτω δεξιά γωνία του σχεδίου.

Στο εξάρτημα πρέπει να εμφανίζονται το όνομά του (CP2104) και η ονομασία αναφοράς (U?). Κάντε δεξί κλικ επάνω στο όνομα, επιλέξτε μετακίνηση, και τοποθετήστε το όνομα κάτω από το σώμα του εξαρτήματος. Επίσης κάντε δεξί κλικ επάνω στην ονομασία αναφοράς, επιλέξτε μετακίνηση, και τοποθετήστε την πάνω από το σώμα του εξαρτήματος. 

\begin{figure}
  \begin{center}
    %\includegraphics{img/kicad-main.png}
    %\caption{Ορθογώνιο σώμα εξαρτήματος, με τοποθετημένα όνομα και αναφορά}
    \label{fig:kicad-main}
  \end{center}
\end{figure}

Να σημειωθεί ότι έως τώρα έχουμε ορίσει μόνο βοηθητικά στοιχεία του εξαρτήματος: το όνομά του, τι φίλτρο αποτυπώματος θα έχει, πως θα εμφανίζεται στο σχηματικό, κα. Τίποτα από αυτά δεν έχει ηλεκτρική/λειτουργική σημασία για το εξάρτημα.

Τώρα πρέπει να προσθέσουμε τους ακροδέκτες που θα αποτελούν το εξάρτημά μας. Το πόσους και τι είδους ακροδέκτες πρέπει να προσθέσουμε θα το γνωρίζουμε από το εγχειρίδιο/φύλλο δεδομένων (data sheet) του ολοκληρωμένου. Αυτό μπορείτε να το βρείτε στην ιστοσελίδα του κατασκευαστή του ολοκληρωμένου. 

Κατά τη συγγραφή αυτού του οδηγού εκμάθησης το εγχειρίδιο του CP2104 (έκδοση/Rev. 1.1) βρισκόταν στον σύνδεσμο \href{https://www.silabs.com/Support\%20Documents/TechnicalDocs/cp2104.pdf}{https://www.silabs.com/Support\%20Documents/TechnicalDocs/cp2104.pdf}. Αν δεν είναι διαθέσιμο σε αυτό τον σύνδεσμο επισκεφθείτε τη σελίδα του κατασκευαστή (\href{https://www.silabs.com/}{https://www.silabs.com/}) και αναζητήστε εκεί το εγχειρίδιο του CP2104.

Με βάση το εγχειρίδιο του CP2104, πρέπει να ορίσουμε 25 ακροδέκτες, όπως φαίνεται και στη σχετική εικόνα. Εκτός από τους 24 τυπικούς ακροδέκτες θα έχουμε ως ακροδέκτη και την θερμική έδρα GND του CP2104. Οπότε σύνολικά θα έχουμε 24 ακροδέκτες για το CP2104.

\begin{figure}
  \begin{center}
    %\includegraphics{img/kicad-main.png}
    %\caption{Ακροδέκτες του CP2104}
    \label{fig:kicad-main}
  \end{center}
\end{figure}

Σε αυτό το tutorial θα ορίσουμε 2 ακροδέκτες μόνο, και οι υπόλοιποι πρέπει να προστεθούν από εσάς, κατά τον ίδιο τρόπο όπως και οι 2 πρώτοι.

Αρχικά ας προσθέσουμε τον ακροδέκτη Vio, ο οποίος σύμφωνα με το εγχειρίδιο έχει το όνομα Vio, του έχει αποδοθεί ο αριθμός 5, και η λειτουργία του είναι είσοδος ισχύος τροφοδοσίας.

Από το μενού στο πάνω μέρος της οθόνης επιλέξετε Τοποθέτηση$\rightarrow$Ακροδέκτης και κάντε κλικ στην κεντρική οθόνη για να εμφανιστεί το παράθυρο Ιδιότητες Ακροδέκτη.

\begin{figure}
  \begin{center}
    %\includegraphics{img/kicad-main.png}
    %\caption{Παράθυρο Ιδιότητες Ακροδέκτη}
    \label{fig:kicad-main}
  \end{center}
\end{figure}

Σε αυτό το παράθυρο πρέπει να ορίσουμε τις παρακάτω σημαντικές ιδιότητες για τον ακροδέκτη, και μετά να το τοποθετήσουμε στο σώμα του εξαρτήματος.

\begin{itemize}
    \item Όνομα ακροδέκτη: Vio
    \item Αριθμός ακροδέκτη: 5
    \item Προσανατολισμός: Αριστερά
    \item Ηλεκτρικός Τύπος: Είσοδος ισχύος
\end{itemize}

Δεν έχει ηλεκτρική σημασία σε ποιο σημείο του σώματος θα τοποθετήσουμε τον ακροδέκτη (πάνω αριστερά, στη μέση κλπ) ή τι προσανατολισμό θα του δώσουμε. Αυτά είναι δευτερεύοντα χαρακτηριστικά, τα οποία μας βοηθούν στην απεικόνιση του σχηματικού. Συνήθως τους ακροδέκτες που έχουν συναφή λειτουργία τους σχεδιάζουμε κοντά τον έναν στον άλλο.

Το όνομα, ο αριθμός, και ο τύπος του ακροδέκτη έχουν ηλεκτρική σημασία καθώς λαμβάνονται υπόψιν κατά τον Έλεγχο Ηλεκτρικών Κανόνων που θα κάνουμε με το KiCad αργότερα.

\begin{figure}
  \begin{center}
    %\includegraphics{img/kicad-main.png}
    %\caption{Σώμα εξαρτήματος, με τοποθετημένο έναν ακροδέκτη}
    \label{fig:kicad-main}
  \end{center}
\end{figure}

Στη συνέχεια προσθέσετε τον ακροδέκτη D+, ο οποίος σύμφωνα με το εγχειρίδιο έχει το όνομα D+, του έχει αποδοθεί ο αριθμός 3, και η λειτουργία του είναι είσοδος/έξοδος δεδομένων.

Όπως και πριν, από το μενού στο πάνω μέρος της οθόνης επιλέξετε Τοποθέτηση$\rightarrow$Ακροδέκτης και κάντε κλικ στην κεντρική οθόνη για να εμφανιστεί το παράθυρο Ιδιότητες Ακροδέκτη.

Σε αυτό το παράθυρο πρέπει να ορίσουμε τις παρακάτω σημαντικές ιδιότητες για τον ακροδέκτη, και μετά να τον τοποθετήσουμε στο σώμα του εξαρτήματος.

\begin{itemize}
    \item Όνομα ακροδέκτη: D+
    \item Αριθμός ακροδέκτη: 3
    \item Προσανατολισμός: Δεξιά
    \item Ηλεκτρικός Τύπος: Αμφίδρομο
\end{itemize}

\begin{figure}
  \begin{center}
    %\includegraphics{img/kicad-main.png}
    %\caption{Σώμα εξαρτήματος, με τοποθετημένους δύο ακροδέκτες}
    \label{fig:kicad-main}
  \end{center}
\end{figure}

Συνεχίζοντας κατά τον ίδιο τρόπο, τοποθετήστε και τους υπόλοιπους 23 ακροδέκτες του εξαρτήματος, ώστε να καταλήξετε τελικά στο εξάρτημα όπως αυτό φαίνεται στη σχετική εικόνα.

\begin{figure}
  \begin{center}
    %\includegraphics{img/kicad-main.png}
    %\caption{Σώμα εξαρτήματος, με τοποθετημένους όλους τους ακροδέκτες}
    \label{fig:kicad-main}
  \end{center}
\end{figure}

Στο εξάρτημα, εκτός από το σχεδιασμό του περιγράμματος και την τοποθέτηση των ακροδεκτών, μπορούμε να τοποθετήσουμε και άλλες οπτικές πληροφορίες, όπως κείμενο και επιπλέον γραμμές γραφικών που να προσφέρουν βοηθητικές πληροφορίες.

Για να προσθέσουμε κείμενο, επιλέγουμε από τη δεξιά μπάρα το το εικονίδιο "Προσθήκη κειμένου στο σώμα εξαρτήματος" %\includegraphics{img/graphic.png}
, κάνουμε κλικ στο σώμα του εξαρτήματος, και στο παράθυρο "Ιδιότητες Κειμένου Βιβλιοθήκης" που εμφανίζεται γράφουμε ό,τι κείμενο θέλουμε, επιλέγουμε αν θα εμφανίζεται κάθετα, τι στοίχιση θα έχει, κοκ.

\begin{figure}
  \begin{center}
    %\includegraphics{img/kicad-main.png}
    %\caption{Παράθυρο "Ιδιότητες Κειμένου Βιβλιοθήκης"}
    \label{fig:kicad-main}
  \end{center}
\end{figure}

Για παράδειγμα, στο CP2104 μπορούμε να βάλουμε κατακόρυφα δίπλα στους ακροδέκτες D+ και D- τη λέξη USB, το οποίο θα μας βοηθάει να θυμόμαστε ότι αυτή είναι η λειτουργία τους.

Επίσης, μπορούμε να βάλουμε κατακόρυφα δίπλα στους ακροδέκτες RTX, TXD, DTR τη λέξη UARTOUT, το οποίο θα μας βοηθάει να θυμόμαστε ότι αυτή είναι η λειτουργία τους.

Ομοίως και για τους ακροδέκτες DCD, RI, CTS, RXD, DSR τη λέξη UARTΙΝ.

Αφού έχουμε ορίσει όλους τους ακροδέκτες, καλό θα είναι να αποθηκεύσουμε το εξάρτημα στη βιβλιοθήκη. Από το μενού στο πάνω μέρος της οθόνης επιλέξτε Αρχείο$\rightarrow$Αποθήκευση της τρέχουσας βιβλιοθήκης. Θα εμφανιστούν δύο παράθυρα που θα ζητούν την επιβεβαίωση αυτή της αποθήκευσης. Επιλέξτε Yes/Ναι και στα δύο.

Πλέον έχουμε ολοκληρώσει τη δημιουργία του εξαρτήματος.

Σε αυτή τη φάση έχουμε δημιουργήσει τα αρχεία kicad\_tut02.zip.

\subsection{Επεξεργασία εξαρτημάτων με πολλά μέρη}

Σε αυτό το κεφάλαιο θα φορτώσουμε ένα εξάρτημα (ένα led) από μία εσωτερική βιβλιοθήκη του KiCad, θα το επεξεργαστούμε κάνοντάς το να αποτελείται από πολλά μέρη, και θα το αποθηκεύσουμε (αλλαγμένο) σε μία δική μας βιβλιοθήκη.

Όταν ένα εξάρτημα αποτελείται από πολλά "μέρη" στο KiCad, αυτό σημαίνει ότι το εξάρτημα είναι ένα φυσικό αντικείμενο (πχ ένα ολοκληρωμένο) το οποίο όμως αποτελείται εσωτερικά από πολλά ίδια μέρη. Στην περίπτωσή μας, εμείς θέλουμε ένα εξάρτημα το οποίο αποτελείται από δύο led.

Δεν υπάρχει εξάρτημα το οποίο αποτελείται από δύο led στις γνωστές βιβλιοθήκες του KiCad. Θα πρέπει να το σχεδιάσουμε εμείς.

Θα μπορούσαμε να δημιουργήσουμε από το μηδέν ένα νέο τέτοιο εξάρτημα, στον Επεξεργαστή Βιβλιοθήκης Εξαρτημάτων. 

Είναι όμως καλύτερα να χρησιμοποιήσουμε το υπάρχον εξάρτημα led, και να σχεδιάσουμε ένα νέο εξάρτημα το οποίο θα αποτελείται από δύο εξαρτήματα led.

\subsubsection{Άνοιγμα του εξαρτήματος}
Θεωρούμε ότι είμαστε στην αρχική οθόνη του KiCad. 

Από το μενού στο πάνω μέρος της οθόνης επιλέξτε Εργαλεία$\rightarrow$Εκτέλεση Επεξεργαστή Βιβλιοθήκης, ώστε να φορτωθεί η κεντρική οθόνη του Επεξεργαστή Βιβλιοθήκης Εξαρτημάτων.

\begin{figure}
  \begin{center}
    %\includegraphics{img/kicad-main.png}
    %\caption{Κεντρική οθόνη του Επεξεργαστή Βιβλιοθήκης Εξαρτημάτων}
    \label{fig:kicad-main}
  \end{center}
\end{figure}

Από την πάνω μπάρα, επιλέξτε το κουμπί "Επιλογή βιβλιοθήκης εργασίας" και στο παράθυρο που θα εμφανιστεί κάντε κλικ πάνω στη βιβλιοθήκη device και πατήστε ΟΚ για να φορτωθεί η βιβλιοθήκη και να επιστρέψετε στην κεντρική οθόνη του Επεξεργαστή Βιβλιοθήκης Εξαρτημάτων.

Από την κεντρική οθόνη του Επεξεργαστή Βιβλιοθήκης Εξαρτημάτων κάντε κλικ στο εικονίδιο "Φόρτωση εξαρτήματος για επεξεργασία από την τρέχουσα βιβλιοθήκη" %\includegraphics{img/graphic.png}
. Το παράθυρο που θα εμφανιστεί εμφανίζει όλα τα εξαρτήματα της βιβλιοθήκης LED. Βρείτε στη λίστα το εξάρτημα LED, επιλέξτε το, και πατήστε ΟΚ για να φορτωθεί στην κεντρική οθόνη του Επεξεργαστή Βιβλιοθήκης Εξαρτημάτων.

\begin{figure}
  \begin{center}
    %\includegraphics{img/kicad-main.png}
    %\caption{Κεντρική οθόνη του Επεξεργαστή Βιβλιοθήκης Εξαρτημάτων με φορτωμένο το εξάρτημα LED}
    \label{fig:kicad-main}
  \end{center}
\end{figure}

Τώρα, αφού έχουμε φορτώσει το εξάρτημα, θα φτιάξουμε ένα αντίγραφό του, και το αντίγραφο θα το αλλάξουμε ώστε να καλύπτει τις ανάγκες μας.

\subsubsection{Δημιουργία αντίγραφου εξαρτήματος}

Από την κεντρική οθόνη του Επεξεργαστή Βιβλιοθήκης Εξαρτημάτων κάντε κλικ στο εικονίδιο "Δημιουργία νέου εξαρτήματος από το τρέχον" %\includegraphics{img/graphic.png}
. Θα εμφανιστεί το παράθυρο "Όνομα Εξαρτήματος". Ονομάστε το νέο εξάρτημα bLED και πατήστε ΟΚ.

\begin{figure}
  \begin{center}
    %\includegraphics{img/kicad-main.png}
    %\caption{Παράθυρο "Όνομα Εξαρτήματος"}
    \label{fig:kicad-main}
  \end{center}
\end{figure}

Πλέον έχουμε δημιουργήσει ένα νέο εξάρτημα, το οποίο καλό είναι να αποθηκεύσουμε. Αλλά βρισκόμαστε στη βιβλιοθήκη device, η οποία επειδή είναι εσωτερική βιβλιοθήκη του KiCad δεν μπορούμε να γράψουμε σε αυτήν. Πρέπει να αποθηκεύσουμε το νέο εξάρτημα σε άλλη βιβλιοθήκη.

Από την πάνω μπάρα, επιλέξτε το κουμπί "Επιλογή βιβλιοθήκης εργασίας" %\includegraphics{img/graphic.png}
 και στο παράθυρο που θα εμφανιστεί κάντε κλικ πάνω στη βιβλιοθήκη uart και πατήστε ΟΚ για να φορτωθεί η βιβλιοθήκη και να επιστρέψετε στην κεντρική οθόνη του Επεξεργαστή Βιβλιοθήκης Εξαρτημάτων.

Από το μενού στο πάνω μέρος της οθόνης επιλέξτε Αρχείο$\rightarrow$Αποθήκευση της τρέχουσας βιβλιοθήκης. Θα εμφανιστούν δύο παράθυρα που θα ζητούν την επιβεβαίωση αυτή της αποθήκευσης. Επιλέξτε Yes/Ναι και στα δύο.

Πλέον έχουμε αποθηκεύσει το εξάρτημα που φτιάξαμε (αντιγράφοντας ένα υπάρχον) στη δική μας βιβλιοθήκη.

\subsubsection{Αλλαγή ιδιοτήτων εξαρτήματος}
Κάντε κλικ στο εικονίδιο "Επεξεργασία ιδιοτήτων εξαρτήματος" %\includegraphics{img/graphic.png} 
για να εμφανιστεί το παράθυρο Ιδιοτήτων για το εξάρτημα. 

\begin{figure}
  \begin{center}
    %\includegraphics{img/kicad-main.png}
    %\caption{Παράθυρο Ιδιοτήτων Εξαρτήματος}
    \label{fig:kicad-main}
  \end{center}
\end{figure}

Αλλάξτε τον αριθμό των μερών από ένα σε 2, και πατήστε ΟΚ για να επιστρέψετε στην κεντρική οθόνη του Επεξεργαστή Βιβλιοθήκης Εξαρτημάτων.

Πλέον το εξάρτημά μας αποτελείται από δύο μέρη. Ποιο μέρος του εξαρτήματος (μέρος Α ή μέρος Β) επεξεργάζεστε ανά πάσα στιγμή φαίνεται στο πάνω δεξιά μέρος της κεντρικής οθόνης του Επεξεργαστή Βιβλιοθήκης Εξαρτημάτων.

\begin{figure}
  \begin{center}
    %\includegraphics{img/kicad-main.png}
    %\caption{Ένδειξη μέρους εξαρτήματος}
    \label{fig:kicad-main}
  \end{center}
\end{figure}

Αφού πλέον το εξάρτημα bLED αποτελείται από δύο μέρη, πρέπει να αριθμήσουμε τους ακροδέκτες του μέρους Α ως 1 και 3 και τους ακροδέκτες του μέρους Β ως 2 και 4.

Για να το κάνουμε αυτό, επιλέγουμε στην "Ένδειξη μέρους εξαρτήματος" το "Μέρος A", και φροντίζουμε οι δύο ακροδέκτες να έχουν αριθμό 1 (άνοδος) και 3 (κάθοδος). Μπορούμε να αλλάξουμε τον αριθμό ενός ακροδέκτη κάνοντας δεξί κλικ επάνω στον ακροδέκτη και επιλέγοντας "Επεξεργασία Ακροδέκτη".

Παρομοίως, επιλέγουμε στην "Ένδειξη μέρους εξαρτήματος" το "Μέρος B", και φροντίζουμε οι δύο ακροδέκτες να έχουν αριθμό 2 (άνοδος) και 4 (κάθοδος). Μπορούμε να αλλάξουμε τον αριθμό ενός ακροδέκτη κάνοντας δεξί κλικ επάνω στον ακροδέκτη και επιλέγοντας "Επεξεργασία Ακροδέκτη".

\begin{figure}
  \begin{center}
    %\includegraphics{img/kicad-main.png}
    %\caption{Εξάρτημα bLEB - Μέρος Α}
    \label{fig:kicad-main}
  \end{center}
\end{figure}

\begin{figure}
  \begin{center}
    %\includegraphics{img/kicad-main.png}
    %\caption{Εξάρτημα bLEB - Μέρος Β}
    \label{fig:kicad-main}
  \end{center}
\end{figure}

Αφού έχετε ορίσει τους αριθμούς των ακροδεκτών, δεν μένει παρά να αποθηκεύσουμε το εξάρτημα.

Από το μενού στο πάνω μέρος της οθόνης επιλέξτε Αρχείο$\rightarrow$Αποθήκευση της τρέχουσας βιβλιοθήκης. Θα εμφανιστούν δύο παράθυρα που θα ζητούν την επιβεβαίωση αυτή της αποθήκευσης. Επιλέξτε Yes/Ναι και στα δύο.

Kλείστε τον Επεξεργαστή Βιβλιοθήκης Εξαρτημάτων ώστε να επιστρέψετε στην αρχική οθόνη του KiCad.

Σε αυτή τη φάση έχουμε δημιουργήσει τα αρχεία kicad\_tut03.zip.

\subsection{Σχεδίαση κυκλώματος}
Εφόσον έχουμε ετοιμάσει όλα τα εξαρτήματα που θα χρειαστούμε, πρέπει να σχεδιάσουμε το κύκλωμα. 

Βεβαιωθείτε ότι είστε στην αρχική οθόνη του KiCad και έχετε φορτώσει το έργο usb2uart.

Από το μενού στο πάνω μέρος της οθόνης επιλέγουμε Εργαλεία$\rightarrow$Εκτέλεση EEschema, για να εκτελεστεί η εφαρμογή EEschema με την οποία σχεδιάζουμε το σχηματικό κύκλωμα. 

\subsubsection{Προσθήκη εξαρτημάτων}

Πρέπει να βρίσκεστε στην αρχική οθόνη της εφαρμογής EEschema, και να είναι φορτωμένο το φύλλο του σχηματικού κυκλώματος του έργου usb2uart.

\begin{figure}
  \begin{center}
    %\includegraphics{img/kicad-main.png}
    %\caption{Αρχική οθόνη της εφαρμογής EEschema, με φορτωμένο το σχηματικό κύκλωμα του έργου usb2uart}
    \label{fig:kicad-main}
  \end{center}
\end{figure}

Αρχικά πρέπει να προσθέσουμε στο κύκλωμα τα εξαρτήματα CP2104 και bLED. 

Όπως κάναμε και σε προηγούμενα κεφάλαια, επιλέξτε από το μενού Τοποθέτηση$\rightarrow$Εξάρτημα, επιλέξτε το CP2104 από τη βιβλιοθήκη usb2uart, και τοποθετήστε το στο κύκλωμα. Στη συνέχεια κάντε το ίδιο για να τοποθετήσετε το εξάρτημα bLED (και τα δύο μέρη), από την ίδια βιβλιοθήκη. Το εξάρτημα bLED αποτελείται από δύο μέρη, οπότε πρέπει να προσθέσετε πρώτα το Μέρος Α και μετά το μέρος Β.

Να σημειωθεί ότι αντί να προσθέσουμε δύο φορές το bLED (μία το Μέρος Α, και μία το Μέρος Β), θα μπορούσαμε να το προσθέσουμε μία φορά, μετά να κάναμε αντιγραφή του (δεξί κλικ στο εξάρτημα$\rightarrow$Αντιγραφή Εξαρτήματος), και μετά να αλλάζαμε στις ιδιότητες του αντίγραφου το Μέρος. Το αποτέλεσμα θα ήταν το ίδιο.

Πλέον πρέπει να έχουμε φορτώσει στο φύλλο όλα τα εξαρτήματα που χρειαζόμαστε για το κύκλωμά μας.

\begin{figure}
  \begin{center}
    %\includegraphics{img/kicad-main.png}
    %\caption{Φύλλο σχηματικού κυκλώματος έργου usb2uart με όλα τα εξαρτήματα που χρειαζόμαστε}
    \label{fig:kicad-main}
  \end{center}
\end{figure}

Σε αυτή τη φάση έχουμε δημιουργήσει τα αρχεία kicad\_tut04.zip.

\subsubsection{Σχεδίαση κυκλώματος}

Πλέον αφού έχουμε όλα τα απαραίτητα εξαρτήματα στο φύλλο μας, πρέπει να τα τοποθετήσουμε στο φύλλο με τρόπο που να μας βοηθάει στην κατανόηση του κυκλώματος, και (κυρίως) να κάνουμε όλες τις συνδέσεις μεταξύ τους.

Κάποιες βασικές λειτουργίες για τη σχεδίαση του κυκλώματος είναι οι παρακάτω.

\paragraph{Μετακίνηση εξαρτήματος}
Μετατοπίζουμε το εξάρτημα όπου θέλουμε πάνω στο φύλλο του κυκλώματος. Η θέση αυτή δεν έχει κάποια ηλεκτρική σημασία, βοηθάει όμως στην καλύτερη οργάνωση του κυκλώματος. Για παράδειγμα το μοναδικό ολοκληρωμένο ενός κυκλώματος καλό θα ήταν να είναι στο μέσο του φύλλου ώστε να υπάρχει χώρος να τοποθετηθούν και άλλα εξαρτήματα γύρω του.

Για να μετακινήσουμε ένα εξάρτημα κάνουμε κλικ επάνω του, από το μενού που εμφανίζεται επιλέγουμε "Μετακίνηση Εξαρτήματος", και μετά το τοποθετούμε όπου θέλουμε στο φύλλο. 

Εναλλακτικά μπορούμε να πάμε τον δείκτη του ποντικιού επάνω από το εξάρτημα (χωρίς να κάνουμε κλικ) και να πατήσουμε το πλήκτρο m.

\paragraph{Περιστροφή εξαρτήματος}
Περιστρέφουμε το εξάρτημα πάνω στο φύλλο του κυκλώματος. Η θέση περιστροφής δεν έχει κάποια ηλεκτρική σημασία, βοηθάει όμως στην καλύτερη οργάνωση του κυκλώματος. Για παράδειγμα μία αντίσταση μπορεί να προτιμάμε να εμφανίζεται οριζόντια για να διαβάζεται πιο εύκολα το όνομά της.

Για να περιστρέψουμε ένα εξάρτημα κάνουμε δεξί κλικ επάνω του, και από το μενού που εμφανίζεται επιλέγουμε "Προσανατολισμός Εξαρτήματος"$\rightarrow$"Περιστροφή Αριστερόστροφα" (ή "Περιστροφή Δεξιόστροφα"). 

Εναλλακτικά μπορούμε να πάμε τον δείκτη του ποντικιού επάνω από το εξάρτημα (χωρίς να κάνουμε κλικ) και να πατήσουμε το πλήκτρο r.

\paragraph{Τοποθέτηση σύρματος}
Είναι από τις πιο σημαντικές λειτουργίες, καθώς με αυτή συνδέουμε αγώγιμα τα εξαρτήματα μεταξύ ενώνοντάς τα με σύρμα.

Για να ενώσουμε αγώγιμα δύο σημεία του κυκλώματος (πχ τα άκρα δύο αντιστάσεων) από το μενού στο πάνω μέρος της οθόνης επιλέγουμε "Τοποθέτηση"$\rightarrow$"Σύρμα", μετά κάνουμε αριστερό κλικ με το ποντίκι πάνω στο κύκλωμα μία φορά για να ξεκινήσει το σύρμα, μετά κάνουμε κλικ όσες φορές θέλουμε για να ορίσουμε τη διαδρομή, και όταν έχουμε τοποθετήσει όσο σύρμα θέλουμε κάνουμε δεξί κλικ και επιλέγουμε "Τέλος Σύρματος". 

Εναλλακτικά μπορούμε να επιλέξουμε το κουμπί "Τοποθέτηση σύρματος" %\includegraphics{img/graphic.png} 
στην αριστερή μπάρα, και μετά να πατήσουμε το πλήκτρο w για να αρχίσουμε να τοποθετούμε σύρμα και το πλήκτρο k για ορίσουμε το τέλος του σύρματος.

Αξίζει να σημειωθεί ότι όταν συνδέουμε ένα σύρμα με ένα άλλο σύρμα, το KiCad εμφανίζει ένα σύμβολο κόμβου %\includegraphics{img/graphic.png}
. Όταν θέλουμε να συνδέσουμε δύο σύρματα σε έναν ακροδέκτη, συνήθως συνδέουμε το ένα σύρμα στον ακροδέκτη και το δεύτερο σύρμα το συνδέουμε πάνω πάνω στο πρώτο σύρμα.

Συνηθίζεται τα σύρματα να έχουν μόνο κατακόρυφη και οριζόντια φορά (να μην τοποθετούνται διαγώνια δύρματα), διότι έτσι παράγονται πιο ευανάγνωστα κυκλώματα.

\begin{figure}
  \begin{center}
    %\includegraphics{img/kicad-main.png}
    %\caption{Δυο εξαρτήματα (αντιστάσεις) χωρίς σύρμα να ενώνει τα άκρα τους}
    \label{fig:kicad-main}
  \end{center}
\end{figure}

\begin{figure}
  \begin{center}
    %\includegraphics{img/kicad-main.png}
    %\caption{Δυο εξαρτήματα (αντιστάσεις) με σύρμα να ενώνει τα άκρα τους}
    \label{fig:kicad-main}
  \end{center}
\end{figure}
   
\begin{figure}
  \begin{center}
    %\includegraphics{img/kicad-main.png}
    %\caption{Δύο σύρματα συνδεδεμένα με τον ίδιο ακροδέκτη}
    \label{fig:kicad-main}
  \end{center}
\end{figure}

\paragraph{Τοποθέτηση ετικέτας}
Μία ετικέτα είναι ένα αντικείμενο του κυκλώματος που συμβολίζει έναν κόμβο, ένα σημείο σύνδεσης. Αν σχεδιάσουμε πχ ένα σύρμα από το εξάρτημα1 προς μία ετικέτα Χ και μετά ένα σύρμα από το εξάρτημα2 προς τη ίδια ετικέτα Χ, τότε τα δύο εξαρτήματα θα είναι συνδεδεμένα μεταξύ τους. Η συνδεσμολογία όπου δύο εξαρτήματα είναι συνδεδεμένα μεταξύ τους απευθείας είναι απόλυτα ισοδύναμη με τη συνδεσμολογία όπου δύο εξαρτήματα είναι συνδεδεμένα μεταξύ τους μέσω ετικέτας.

Για να τοποθετήσουμε μία ετικέτα, από το μενού στο πάνω μέρος της οθόνης επιλέγουμε "Τοποθέτηση"$\rightarrow$"Ετικέτα", μετά κάνουμε κλικ με το ποντίκι πάνω στο κύκλωμα, στο παράθυρο που εμφανίζεται δίνουμε ένα όνομα στην ετικέτα και πατάμε ΟΚ για να τοποθετήσουμε την ετικέτα πάνω στο κύκλωμα. Στη συνέχεια πρέπει να μετακινήσουμε την ετικέτα ώστε το κουτάκι που υπάρχει κάτω αριστερά από το όνομά της να συμπέσει με το κουτάκι που υπάρχει στο τέλος ενός ασύνδετου σύρματος. Εφόσον αυτά τα δύο κουτάκια είναι στο ίδιο σημείο (συμπίπτουν), πλέον το σύρμα έχει αυτή την ετικέτα, και θεωρείται συνδεδεμένο με όλα τα άλλα σύρματα που έχουν ετικέτα με το ίδιο όνομα. Λέμε ότι όλα αυτά τα σύρματα ανήκουν στο ίδιο δίκτυο.

\begin{figure}
  \begin{center}
    %\includegraphics{img/kicad-main.png}
    %\caption{Δύο εξαρτήματα, συνδεδεμένα μεταξύ τους απευθείας}
    \label{fig:kicad-main}
  \end{center}
\end{figure}

\begin{figure}
  \begin{center}
    %\includegraphics{img/kicad-main.png}
    %\caption{Δύο εξαρτήματα, συνδεδεμένα μεταξύ τους μέσω ετικέτας}
    \label{fig:kicad-main}
  \end{center}
\end{figure}

\paragraph{Διαγραφή αντικειμένου}
Για να διαγράψουμε ένα αντικείμενο από το φύλλο μας (πχ ένα κομμάτι σύρματος που αποφασίσαμε να τοποθετήσουμε διαφορετικά), κάνουμε δεξί κλικ επάνω του και από το μενού που εμφανίζεται επιλέγουμε "Διαγραφή".

Εναλλακτικά μπορούμε να πάμε τον δείκτη του ποντικιού επάνω από το εξάρτημα (χωρίς να κάνουμε κλικ) και να πατήσουμε το πλήκτο Delete στο πληκτρολόγιο.

Έχοντας κατανοήσει τις παραπάνω λειτουργίες, μπορούμε να σχεδιάσουμε το κύκλωμά μας.

Για ευκολία στη σχεδίαση συνδέσεων μετακινήστε το CP2104 ώστε να είναι στο κέντρο της σελίδας.

Στη συνέχεια, συνδέστε με σύρμα τους ακροδέκτες 7 και 8 του CP2104 με το εξάρτημα VCC. Για να το κάνετε, συνδέσετε με σύρμα τον ακροδέκτη 7 με το εξάρτημα VCC και συ συνέχεια τοποθετήστε ένα σύρμα από τον  ακροδέκτη 7 προς οποιοδήποτε σημείο του προηγούμενου σύρματος.

Στη συνέχεια, συνδέστε με σύρμα τους ακροδέκτες 5,6 και 9 του CP2104 με το 3.3V.

Στη συνέχεια, συνδέστε με σύρμα τους ακροδέκτες 2 και 25 του CP2104 με το GND.

\begin{figure}
  \begin{center}
    %\includegraphics{img/kicad-main.png}
    %\caption{Σύνδεση ακροδεκτών 7 και 8 του CP2104 με το εξάρτημα VCC}
    \label{fig:kicad-main}
  \end{center}
\end{figure}

Για να κάνετε τα παραπάνω, ίσως χρειαστεί να μετακινήσετε τα VCC, 3.3V και GND σε πιο βολικές θέσεις στο φύλλο. Για παράδειγμα, το VCC καλό θα ήταν να τοποθετηθεί πάνω αριστερά από το CP2104.

\begin{figure}
  \begin{center}
    %\includegraphics{img/kicad-main.png}
    %\caption{Το κύκλωμα με τις αρχικές συνδέσεις των VCC, 3.3V και GND}
    \label{fig:kicad-main}
  \end{center}
\end{figure}

Στη συνέχεια, αλλάξτε τις τιμές στους τέσσερις πυκνωτές ώστε οι τέσσερις πυκνωτές να έχουν τιμές 1u, 1u, 100n και 470n. Για να αλλάξετε τις τιμές δεξί κλικ πάνω στον πυκνωτή και επιλέξετε "Επεξεργασία Εξαρτήματος"$\rightarrow$"Τιμή".

Στη συνέχεια, συνδέστε τους πυκνωτές 1u και 100n ως bypass στο 3.3V, δηλαδή ανάμεσα στο 3.3v και στο GND. Για να το κάνετε αυτό και το κύκλωμα να παραμείνει ευανάγνωστο μπορείτε να προσθέσετε επιπλέον εξαρτήματα GND. Όλα τα εξαρτήματα GND που θα προσθέσετε συμβολίζουν το ίδιο σημείο, τον ίδιο κόμβο στο κύκλωμά σας, τον κόμβο της γείωσης. Οπότε, για τον πυκνωτή 1u προσθέστε ένα εξάρτημα GND στο φύλλο (επιλέγοντας στο μενού Τοποθέτηση$\rightarrow$Εξάρτημα) λίγο πιο δεξιά από το 3.3V, και μετά τοποθετήστε σύρμα που να συνδέει το 3.3V με το 1u και μετά σύρμα που να συνδέει το 1u με το (νέο) GND. Ομοίως θα κάνετε και για το 100n.

Στη συνέχεια, συνδέστε με τον ίδιο τρόπο όπως παραπάνω τον πυκνωτή 1u ως bypass στο VCC, δηλαδή συνδέστε τον ανάμεσα στο VCC και σε ένα GND.

Στη συνέχεια, συνδέστε τον πυκνωτή 470n ανάμεσα στους ακροδέκτες 16 και 2 του CP2104.

Στη συνέχεια, αλλάξτε τις τιμές στα δύο εξαρτήματα CONN\_01X04 ώστε το ένα να λέγεται "USB", και το άλλο "Header". Τοποθετήστε το USB αριστερά του CP2104 και το Header δεξιά του CP2104.

\begin{figure}
  \begin{center}
    %\includegraphics{img/kicad-main.png}
    %\caption{Το κύκλωμα με τις πλήρεις συνδέσεις των VCC, 3.3V και GND και τα USB, Header τοποθετημένα}
    \label{fig:kicad-main}
  \end{center}
\end{figure}

Στη συνέχεια, θα πρέπει να συνδέσουμε όλους τους ακροδέκτες των USB και Header. Αν συνδέαμε απευθείας μεταξύ τους όλους του ακροδέκτες το κύκλωμα δενθα ήταν ιδιαίτερα ευανάγνωστο και οργανωμένο διότι θα υπήρχαν πολλά διασταυρούμενα (αλλά μην συνδεδεμένα) σύρματα. Για να λύσουμε αυτό το πρόβλημα, θα χρησιμοποιήσουμε ετικέτες.

Με δεδομένο το εργαλείο της ετικέτας, θα τοπθετήσουμε σύρματα ώστε να συνδέσουμε όπως θέλουμε τους ακροδέκτες των USB και Header.

Τοποθετήστε σύρμα από τον ακροδέκτη 1 του USB έως το VCC.

Τοποθετήστε σύρμα από τον ακροδέκτη 4 του USB έως το GND.

Τοποθετήστε σύρμα από τον ακροδέκτη 1 του HEADER έως το 3.3V.

Τοποθετήστε σύρμα από τον ακροδέκτη 4 του HEADER έως το GND.

Τοποθετήστε σύρμα από τον ακροδέκτη 2 του USB έως το D- του CP2104, χρησιμοποιώντας ετικέτα με όνομα USBDP. 

Τοποθετήστε σύρμα από τον ακροδέκτη 3 του USB έως το D+ του CP2104, χρησιμοποιώντας ετικέτα με όνομα USBDM.

Τοποθετήστε σύρμα από τον ακροδέκτη 2 του HEADER έως το RXD του CP2104, χρησιμοποιώντας ετικέτα με όνομα RXD.

Τοποθετήστε σύρμα από τον ακροδέκτη 3 του HEADER έως το TXD του CP2104, χρησιμοποιώντας ετικέτα με όνομα TXD.

Πλέον ολοκληρώσαμε τις συνδέσεις των USB και Header.

Τώρα πρέπει να συνδέσουμε τα LED. 

Μετακινήστε τα LED (το ένα μονό και το ένα που αποτελείται απόδύο μέρη) και τις τρεις αντιστάσεις στην περιοχή πάνω από το CP2104.

Στη συνέχεια, αλλάξτε τις τιμές στις τρεις αντιστάσεις του κυκλώματος ώστε να έχουν τις τιμές 220R, 220R και 430R.

Στη συνέχεια, τοποθετήστε σύρμα που να συνδέει το VCC με τον ένα ακροδέκτη της 430R, και τον άλλο ακροδέκτη της 430R με την άνοδο του μονού led και μετά την κάθοδο του μονού led με το GND.

Στη συνέχεια, τοποθετήστε σύρμα που να συνδέει το 3.3V με τον ένα ακροδέκτη της 220R, και τον άλλο ακροδέκτη της 220R με την άνοδο του bled (Μέρος Α) και την κάθοδο του bled (Μέρος Α), μέσω ετικέτας LEDRX, με τον ακροδέκτη 13 του CP2104.

Στη συνέχεια, τοποθετήστε σύρμα που να συνδέει το 3.3V με τον ένα ακροδέκτη της δεύτερης 220R, και τον άλλο ακροδέκτη της δεύτερης 220R με την άνοδο του bled (Μέρος Β) και την κάθοδο του bled (Μέρος Β), μέσω ετικέτας LEDRX, με τον ακροδέκτη 14 του CP2104.

Στη συνέχεια, προσθέστε κείμενο δίπλα στα LED ώστε να είναι σαφές τι χρώμα είναι το κάθε ένα (μονό LED = μπλε, A = πράσινο, B = πορτοκαλί). Για να προσθέσουμε κείμενο, επιλέγουμε από τη δεξιά μπάρα το το εικονίδιο "Τοπoθέτηση κειμένου" %\includegraphics{img/graphic.png}
, κάνουμε κλικ στο φύλλο, και στο παράθυρο που εμφανίζεται γράφουμε ό,τι κείμενο θέλουμε, επιλέγουμε αν θα εμφανίζεται κάθετα, τι στοίχιση θα έχει, κοκ.

\begin{figure}
  \begin{center}
    %\includegraphics{img/kicad-main.png}
    %\caption{Συνδεσμολογία LED}
    \label{fig:kicad-main}
  \end{center}
\end{figure}

Τέλος, για λόγους καλύτερης οργάνωσης, πρέπει να προσθέσουμε ετικέτες και στα υπόλοιπα δίκτυα του κυκλώματός μας, και συγκεκριμένα στο VPP (ετικέτα VPP), και στα σύρματα ανάμεσα στις αντιστάσεις και τα LED (ετικέτες LEDONR, LEDRXR και LEDTXR).

Πλέον έχουμε σχεδιάσει πλήρως το κύκλωμα. Για λόγους καλύτερης οργάνωσης, και ηλεκτρολογικού ελέγχου του κυκλώματος, χρειαζόμαστε ακόμα δύο βήματα.

Καταρχάς πρέπει σε όλους τους ακροδέκτες που δεν έχουν κάτι συνδεδεμένο επάνω τους (όπως πχ οι ακροδέκτες 15 και 17 του CP2104) να τοποθετήσουμε ένα σύμβολο Μη Σύνδεσης. Αυτό θα βοηθήσει στο να δηλώσουμε στο KiCad ότι ο συγκεκριμένος ακροδέκτης θέλουμε όντως να μην συνδέεται πουθενά και δεν έχουμε απλά ξεχάσει να τον συνδέσουμε. Για να τοποθετήσουμε σύμβολο Μη Σύνδεσης, επιλέγουμε από το μενού στο πάνω μέρος της οθόνης "Τοποθέτηση"$\rightarrow$"Σήμανση Μη Σύνδεσης" και κάνουμε κλικ στο σημείο όπου θέλουμε να το τοποθετήσουμε. Πρέπει να τοποθετήσουμε μία Σήμανση Μη Σύνδεσης σε κάθε ακροδέκτη του CP2104 που δεν συνδέεται με κάτι άλλο.

Κατά δεύτερον, πρέπει στα δίκτυα VCC και GND να συνδέσουμε μέσω σύρματος από ένα εξάρτημα PWR\_FLAG (από τη βιβλιοθήκη power). Με αυτό τον τρόπο το KiCad, όταν κάνει ηλεκτρολογικό έλεγχο, θα καταλάβει ότι τα δύο δίκτυα αυτά είναι δίκτυα ισχύος, και ότι υπάρχει ισχύς στο κύκλωμά μας.

Πλέον έχουμε ολοκληρώσει πλήρως το κύκλωμα στο KiCad.

\begin{figure}
  \begin{center}
    \includegraphics{img/final-schem.png}
    \caption{Πλήρως σχεδιασμένο κύκλωμα}
    \label{fig:final-schem}
  \end{center}
\end{figure}

\subsubsection{Ολοκλήρωση σχεδιασμού σχηματικού}
Αφού έχουμε ολοκληρώσει τον σχεδιασμό του κυκλώματος, πρέπει να δώσουμε ονομασίες αναφοράς σε όλα τα εξαρτήματα, ώστε οι αντιστάσεις να έχουν ονομασία αναφοράς R1, R2, R3 κοκ. Να σημειωθεί ότι η ονομασία αναφοράς είναι διαφορετική έννοια από την τιμή (πχ 220R). 

Ονομασίες αναφοράς μπορούμε να δώσουμε μόνοι μας, κάνοντας επεξεργασία σε κάθε ένα εξάρτημα ή μπορούμε να πούμε στο KiCad να δώσει αυτόματα ονομασίες αναφοράς σε όλα τα εξαρτήματα. Για να το το κάνουμε αυτό, επιλέγουμε από το μενού στο πάνω μέρος της οθόνης "Εργαλεία"$\rightarrow$"Ονοματοδοσία σχηματικού", στο παράθυρο που εμφανίζεται δεν αλλάζουμε καμία ρύθμιση και κάνουμε κλικ στο "Ονοματοδοσία".

\begin{figure}
  \begin{center}
    %\includegraphics{img/kicad-main.png}
    %\caption{Παράθυρο Ονοματοδοσίας}
    \label{fig:kicad-main}
  \end{center}
\end{figure}

Όταν ολοκληρωθεί η διαδικασία, όλα τα εξαρτήματα θα έχουν συγκεκριμένες, αριθμημένες ονομασίες αναφοράς (και όχι ερωτηματικά).

Αφού ολοκληρωθεί η ονοματοδοσία, πρέπει να κάνουμε Έλεγχο Ηλεκτρικών Κανόνων στο κύκλωμα, ώστε να αναλύσει το κύκλωμα το KiCad και να μας ενημερώσει για πιθανά προβλήματα όπως πχ αν έχουμε ξεχάσει να συνδέσουμε ένα δίκτυο ή αν ένας ακροδέκτης εξαρτήματος έχει χαρακτηρισμό "Είσοδος" αλλά είναι συνδεδεμένος με έναν ακροδέκτη που έχει επίσης χαρακτηρισμό "Είσοδος".

Για να κάνουμε Έλεγχο Ηλεκτρικών Κανόνων, επιλέγουμε από το μενού στο πάνω μέρος της οθόνης "Εργαλεία"$\rightarrow$"Ελεγκτής Ηλεκτρικών Κανόνων" και στο παράθυρο που εμφανίζεται δεν αλλάζουμε καμία ρύθμιση και κάνουμε κλικ στο "Εκτέλεση". 

Αν έχετε χρησιμοποιήσει τα αρχεία του tutorial, ο Ελεγκτής Ηλεκτρικών Κανόνων πρέπει να σας βγάλει μία προειδοποίηση: ότι ο ακροδέκτης 9 (χαρακτηρισμένος ως αμφίδρομος στο εξάρτημα) του CP2104 είναι συνδεδεμένος με τον ακροδέκτη 6 (χαρακτηρισμένος ως έξοδος) του CP2104. Αυτό φαίνεται στο KiCad να είναι λάθος διότι ένας ακροδέκτης έξοδος πρέπει να είναι συνδεδεμένος με έναν ακροδέκτη είσοδο και όχι έναν αμφίδρομο. Στη συγκεκριμένη περίπτωση πρόκειται για απλό θέμα χαρακτηρισμού του ακροδέκτη 9 και μπορούμε να αγνοήσουμε την προειδοποίηση. Οπότε απλά κλείστε το παράθυρο του Ελεγκτή.
Εναλλακτικά, μπορούμε να επεξεργαστούμε το εξάρτημα CP2104 με το σχετικό εργαλείο (εξάλλου εμείς το δημιουργήσαμε το εξάρτημα, από το μηδέν) και να αλλάξουμε τον χαρακτηρισμό του ακροδέκτη 9 σε είσοδο.

Τέλος, αφού έχουμε ολοκληρώσει το κύκλωμά μας, καλό είναι να το κρατήσουμε και σε ένα αρχείο pdf, ώστε να είναι εύκολη η αναφορά σε αυτό χωρίς να ανοίγουμε το KiCad. Για να το κάνουμε αυτό, επιλέγουμε από το μενού στο πάνω μέρος της οθόνης Αρχείο$\rightarrow$Σχεδιογράφηση$\rightarrow$Σχεδιογράφηση και στο παράθυρο που εμφανίζεται επιλέγουμε Μορφή PDF, και πατάμε το κουμπί "Σχεδιογράφηση Τρέχουσας Σελίδας". 

\begin{figure}
  \begin{center}
    %\includegraphics{img/kicad-main.png}
    %\caption{Παράθυρο Σχεδιογράφησης}
    \label{fig:kicad-main}
  \end{center}
\end{figure}

Πλέον το κύκλωμα υπάρχει και σε μορφή pdf, στο φάκελο όπου έχουμε και όλα τα άλλα αρχεία του έργου μας.

Σε αυτή τη φάση έχουμε δημιουργήσει τα αρχεία kicad\_tut05.zip.

%* set custom fields (MFG, MFG code) [TAG JUMP to "tutorial05"]
%-= TAG "tutorial05" =-

\end{document}
