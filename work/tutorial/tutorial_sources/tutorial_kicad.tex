\documentclass[a4paper]{article}

\usepackage{fontspec}
%για επιλογή γραμματοσειρών (μεταξύ άλλων)
\usepackage{polyglossia}
%για συλλαβισμό λέξεων (μεταξύ άλλων)
\usepackage{geometry}
%για μορφή σελίδας a4 (μεταξύ άλλων)
\usepackage{longtable}
%για προχωρημένους πίνακες
\usepackage{tabu}
%για προχωρημένους πίνακες

\setdefaultlanguage{greek}
\setotherlanguage{english}

\setmainfont{DejaVu Sans}

\title{Tutorial - Οδηγός Εκμάθησης KiCad}
\author{Μιχάλης Μισιρλής}
\date{Οκτώβριος 2015}

\begin{document}

\maketitle

\section{Εισαγωγή}
Αυτός ο οδηγός εκμάθησης έχει ως στόχο την εξοικοίωση του Έλληνα χρήστη του KiCad μέσα από την ανάπτυξη ενός απλού αλλά πλήρους κυκλώματος (ηλεκτρονική συσκευή) ξεκινώντας από την βασική ιδέα και το σκοπό της συσκευής και καταλήγοντας στην παραγωγή αρχείων κατασκευής ώστε να μπορεί να παραχθεί η συσκευή από οποιοδήποτε εργοστάσιο ή εταιρεία παραγωγής πλακετών.

\subsection{Προαπαιτούμενα}
\label{sec:prereq}
Αυτός ο οδηγός έχει βασιστεί και απαιτεί το πρόγραμμα KiCad έκδοση 4.0.0 [TBD] με την Ελληνική μετάφραση [TBD]. Επίσης ο οδηγός θεωρεί ότι ο χρήστης είναι εξοικοιωμένος με τις βασικές έννοιες ηλεκτρονικού σχεδιασμού και μπορεί να σχεδιάσει πλήρως ένα κύκλωμα στο χαρτί ή σε κάποιο άλλο πρόγραμμα. Αυτός ο οδηγός δεν έχει ως στόχο τη διδασκαλία ηλεκτρονικού σχεδιασμού, αλλά τη διδασκαλία του πως γίνεται ο ηλεκτρονικος σχεδιασμός με το KiCad.
Τέλος ο οδηγός θεωρεί ότι ο χρήστης είναι εξοικοιωμένος τις βασικές έννοιες χρήσης ηλεκτρονικών υπολογιστών, όπως επιλογή μενού, έξοδος από ένα παράθυρο, επιλογή πολλαπλών στοιχείων κα

\subsection{Λεξικό όρων}
\begin{longtabu}{ c c }
    Ελληνικός όρος	&	Αγγλικός όρος	\\
    3Δ προβολή	&	3d viewer	\\
    αγκύρωση	&	anchor	\\
    ακροδέκτης	&	pin	\\
    ακτίνα	&	spoke	\\
    ακτίνα	&	spoke	\\
    αλφαριθμητικό	&	string	\\
    ανάλυση	&	parse	\\
    αναφορά, ονομασία αναφοράς	&	reference	\\
    αντεστραμμένο	&	inverted	\\
    αντικολλητικής μάσκας	&	solder resist mask	\\
    αντίσταση συγκόλλησης	&	solder resist	\\
    απόδοση	&	render	\\
    απόκλιση	&	skew	\\
    αποτύπωμα	&	footprint	\\
    γομα	&	eraser	\\
    γραμμή γραφικών	&	graphic line	\\
    διάκενο	&	clearance	\\
    διαμπερές	&	through (hole)	\\
    διάτρηση	&	drill	\\
    δίαυλος	&	bus	\\
    δίκτυα μίας έδρας	&	single pad nets	\\
    δρομολόγηση	&	route	\\
    δρόμος	&	pcb trace / track	\\
    έδρα	&	antipad clearance	\\
    έδρα	&	pad	\\
    εηκ	&	erc	\\
    εκς	&	drc	\\
    ελεγκτής ηλεκτρικών κανόνων	&	electrical rules checker	\\
    εξάρτημα	&	component	\\
    έξοδος	&	output	\\
    επάνω πλευρά	&	top side	\\
    επεξεργαστής σχεδίασης σελίδας	&	page layout editor	\\
    επιλογές	&	options / choices / selections	\\
    επίπεδα	&	layers	\\
    επίπεδα eco	&	eco layers	\\
    επίπεδο σχεδίων	&	drawings layer	\\
    επίπεδοα επικόλλησης	&	adhesive layers	\\
    επισήμανση	&	flag	\\
    ζώνες	&	zones	\\
    θαμμένο via	&	buried via	\\
    θύρα	&	port	\\
    θύρα ισχύος	&	power port	\\
    κανόνες σχεδιασμού	&	design rule	\\
    κατοπτρισμός, καθρέφτισμα	&	mirror	\\
    κάτω πλευρά	&	bottom side	\\
    κενό	&	gap	\\
    κοίλες γωνίες	&	fillet	\\
    κώδικας g	&	g code	\\
    λειτουργία γεμίσματος	&	filled mode	\\
    λειτουργία περιγράμματος	&	sketch mode	\\
    λειτουργία περιγράμματος	&	outline mode	\\
    λίστα δικτύων	&	netlist	\\
    λίστα υλικών	&	bill of materials	\\
    λοξότμηση	&	chamfer	\\
    μάσκα συγκόλλησης	&	solder mask	\\
    μέρη	&	unit	\\
    μεταξοτυπία	&	silk screen	\\
    Μετατόπιση \& σπρώξιμο	&	Push \& shove	\\
    μετατόπιση,θέση	&	offset	\\
    μηχανές ανύψωσης και τοποθέτησης	&	“pick and place”	\\
    μικρο-ρύθμιση, ρύθμιση	&	tune	\\
    μπροστινή όψη	&	front view	\\
    ονοματοδοσία	&	annotate	\\
    όρια πλακέτας	&	pcb/board edge	\\
    παλαιότερου τύπου	&	legacy	\\
    παραπομπή πλαισίου	&	frame reference	\\
    πάστα συγκόλλησης	&	solder paste	\\
    περιγράμμα πλακέτας	&	board outline	\\
    πίνακας	&	matrix	\\
    πίσω όψη	&	back view	\\
    πλαίσιο	&	box (για ui)	\\
    πλοηγός	&	navigator	\\
    πρόγραμμα προβολής pdf	&	pdf viewer	\\
    προκαθορισμένου	&	default	\\
    πρόσθετα	&	plugin	\\
    προτιμώμενο	&	favourite	\\
    ρυθμίσεις	&	settings	\\
    ρυθμίσεις	&	configurations	\\
    σκιαγράφηση	&	hatch	\\
    στελέχη	&	stubs	\\
    στοίβα έδρας	&	padstack	\\
    σύγκρουση, ασυμβατότητα	&	conflict	\\
    συντομεύσεις	&	hotkeys	\\
    συνώνυμο	&	alias	\\
    σύρμα	&	wire	\\
    σύρσιμο	&	drag	\\
    συστοιχία	&	array	\\
    σχεδίαση	&	design	\\
    σχεδίασης σελίδας	&	page layout	\\
    σχέδιο	&	drawing	\\
    σχεδιογραφώ	&	plot	\\
    σχηματικό διάγραμμα, σχηματικό	&	schematic	\\
    τόξο	&	arc	\\
    τρύπα	&	hole	\\
    φάκελος	&	folder / directory	\\
    φωλιά συνδέσεων	&	ratsnest	\\
    χαλκός	&	copper pour	\\
    χώρος αποτυπώματος	&	courtyard	\\
    χώρος σχεδίασης	&	canvas	\\
    ψευδώνυμο	&	nickname	\\
    excellon	&	excellon	\\
    excellon	&	excellon	\\
    gerber	&	gerber	\\
    NPTH	&	NPTH	\\
    PTH	&	PTH	\\
    SMD	&	SMD	\\
    VIA	&	VIA	\\
\end{longtabu}

\subsection{Περιγραφή συσκευής}
Η συσκευή ονομάζεται USB2UART και είναι ένας μετατροπέας USB σε 3.3V UART, βασισμένος στο ολοκληρωμένο CP2104 από την εταιρεία Silicon Labs. υποστηρίζεται από τα περισσότερα σύγχρονα λειτουργικά συστήματα ως εικονική σειριακή θύρα. Συνδέεται με έναν standard type B connector από τη μία πλευρά και ένα four-pin header από την άλλη. Περιλαμβάνει power status LED και RX/TX LED.

\subsection{Βασικές Έννοιες}
Το KiCad είναι μία ολοκληρωμένη σουίτα εφαρμογών σχεδίασης ηλεκτρονικών κυκλωμάτων EDA (Electronic Design Automation). Με το KiCad είναι εφικτή η σχεδίαση σχηματικών και τυπωμένων ηλεκτρονικών κυκλωμάτων, χρησιμοποιώντας διαφορετικές εφαρμογές για κάθε δουλειά.

Τα σχηματικά κυκλώματα σχεδιάζονται στο KiCad με την εφαρμογή EEschema. Αποτελούνται από εξαρτήματα (για παράδειγμα ένας πυκνωτής ή ένα ολοκληρωμένο) συνδέσεις μεταξύ τους, και άλλα στοιχεία. Τα σχηματικά κυκλώματα είναι οργανωμένα σε σχηματικά φύλλα. Κάθε σχηματικό φύλλο αποτελεί και ένα αρχείο στον υπολογιστή του χρήστη.

Τα εξαρτήματα των σχηματικών μπορεί να ανήκουν σε μία βιβλιοθήκη του KiCad ή μπορεί να τα έχει σχεδιάσει από το μηδέν ο χρήστης χρησιμοποιώντας το σχετικό εργαλείο του KiCad. Κάθε εξάρτημα αποτελεί και ένα αρχείο στον υπολογιστή του χρήστη.

Τα τυπωμένα κυκλώματα σχεδιάζονται στο KiCad με την εφαρμογή PCBnew. Αποτελούνται από πλακέτες που περιέχουν αποτυπώματα εξαρτημάτων, συνδέσεις μεταξύ τους μέσω δρόμων, και από τρύπες, via, κ.α.
Κάθε πλακέτα αποτελεί και ένα αρχείο στον υπολογιστή του χρήστη. 

Τα αποτυπώματα εξαρτημάτων μπορεί να ανήκουν σε μία βιβλιοθήκη του KiCad ή μπορεί να τα έχει σχεδιάσει από το μηδέν ο χρήστης χρησιμοποιώντας το σχετικό εργαλείο του KiCad. Κάθε αποτύπωμα αποτελεί και ένα αρχείο στον υπολογιστή του χρήστη. 

Τα αποτυπώματα μπορεί να έχουν τρισδιάστατη απεικόνιση. Οι τρισδιάστατες απεικονίσεις μπορεί να ανήκουν σε μία βιβλιοθήκη του KiCad ή μπορεί να τα έχει σχεδιάσει από το μηδέν ο χρήστης χρησιμοποιώντας κάποιο εργαλείο τρισδιάστατης σχεδίασης. Κάθε τρισδιάστατη απεικόνιση αποτελεί και ένα αρχείο στον υπολογιστή του χρήστη. 

Το σύνολο του κυκλώματος (σχηματικό, πλακέτα) οργανώνεται από το KiCad σε έργα. Κάθε έργο αποτελεί και ένα αρχείο στον υπολογιστή του χρήστη. 

\section{Δημιουργία Κυκλώματος}

\subsection{Δημιουργία έργου}
Εκτελέστε το πρόγραμμα KiCad. Δείτε το σχετικό κεφάλαιο ~\ref{sec:prereq} για την έκδοση του KiCad που πρέπει να εκτελέσετε. Θα εμφανιστεί η κεντρική οθόνη του προγράμματος KiCad. Από το μενού στο πάνω μέρος της οθόνης επιλέγουμε Αρχείο -> Νέο Έργο -> Νέο Έργο. Δίνουμε στο νέο έργο (και το σχετικό αρχείο) ένα όνομα, έστω usb2uart. Προτείνετε να δημιουργηθεί το αρχείο του έργου σε έναν φάκελο του υπολογιστή σας ο οποίος θα είναι αφιερωμένος στο συγκεκριμένο έργο.

\begin{figure}
  \begin{center}
    \includegraphics{img/kicad-main.png}
    \caption{Κεντρική οθόνη του προγράμματος KiCad}
    \label{fig:kicad-main}
  \end{center}
\end{figure}

\subsection{Δημιουργία σχηματικού}
Από το μενού στο πάνω μέρος της οθόνης επιλέγουμε Εργαλεία -> Εκτέλεση EEschema, για να εκτελεστεί η εφαρμογή EEschema με την οποία σχεδιάζουμε το σχηματικό κύκλωμα. Εάν εμφανιστεί κάποια προειδοποίηση, αποδεχτείτε την (TBD). Θα δημιουργηθούν δύο αρχεία στον υπολογιστή σας, τα usb2uart.sch και usb2uart-cache.lib. Εκτελέστε ξανά την εφαρμογή ώστε να μην εμφανιστεί κάποια προειδοποίηση (TBD).

Θα εμφανιστεί στην οθόνη σας κεντρική σελίδα της εφαρμογής EEschema. Αυτή αποτελείται από τα παρακάτω
\begin{itemize}
    \item ένα φύλλο σχηματικού κυκλώματος στο κέντρο της οθόνης
    \item γραμμή μενού στο πάνω μέρος της οθόνης
    \item μπάρα βασικών λειτουργιών στο πάνω μέρος της οθόνης
    \item μπάρα με γενικές λειτουργίες στο αριστερό μέρος της οθόνης
    \item μπάρα με συγκεκριμένες λειτουργίες σχεδίασης στο δεξί μέρος της οθόνης
\end{itemize}
Αξίζει να σημειωθεί πως όπως και στις περισσότερες εφαρμογές, έτσι και στο EEschema οι περισσότερες λειτουργίες μπορούν να γίνουν με δύο τρόπους: είτε επιλέγοντας κάτι στο μενού είτε κάνοντας κλικ στο σχετικό κουμπί μίας μπάρας.

\begin{figure}
  \begin{center}
    \includegraphics{img/kicad-main.png}
    \caption{Κεντρική σελίδα της εφαρμογής EEschema}
    \label{fig:kicad-main}
  \end{center}
\end{figure}

Το EEschema μπορεί να σχεδιάσει κυκλώματα που αποτελούνται από πολλά σχηματικά φύλλα. Σε αυτό το tutorial το κύκλωμά μας αποτελείται από ένα και μοναδικό φύλλο.

Από το μενού στο πάνω μέρος της οθόνης του EEschema επιλέξτε Αρχείο -> Ρυθμίσεις Σελίδας. Στο παράθυρο που εμφανίζεται, ορίστε τις ρυθμίσεις της σελίδας σας, όπως μέγεθος χαρτιού, ημερομηνία έκδοσης, τίτλος κυκλώματος, κα και πατήστε ΟΚ. Αυτές οι ρυθμίσεις δεν έχουν κάποια ηλεκτρική ή λειτουργική σημασία για το κύκλωμα, είναι όμως χρήσιμες πληροφορίες για την οργάνωση των κυκλωμάτων σας.

\begin{figure}
  \begin{center}
    \includegraphics{img/kicad-main.png}
    \caption{Ρυθμίσεις Σελίδας}
    \label{fig:kicad-main}
  \end{center}
\end{figure}

Από το μενού στο πάνω μέρος της οθόνης του EEschema επιλέξτε "Τοποθέτηση" -> "Γραφικό Πολυγραμμής".

Με αυτό το εργαλείο μπορείτε να σχεδιάσετε γραμμές στο φύλλο σας. Αυτές οι γραμμές γραφικών δεν έχουν κάποια ηλεκτρική ή λειτουργική σημασία για το κύκλωμα, είναι απλή "ζωγραφική", είναι όμως χρήσιμες για την οργάνωση των κυκλωμάτων σας.

Σχεδιάστε στο φύλλο μία κατακόρυφη γραμμή που χωρίζει το φύλλο σε δύο τμήματα. Το δεξί τμήμα πρέπει να είναι περίπου το ένα τρίτο του συνολικού φύλλου. Αυτό το δεξί τμήμα θα το χρησιμοποιήσουμε ως χώρο για να γράφουμε  βοηθητικές πληροφορίες, και για να εναποθέτουμε προχωρινά τα εξαρτήματα πριν τα τοποθετήσουμε στην οριστική τους θέση στο κύκλωμα που θα σχεδιάσουμε στο αριστερό τμήμα της σελίδας.

\begin{figure}
  \begin{center}
    \includegraphics{img/kicad-main.png}
    \caption{Σχηματικό φύλλο χωρισμένο στα δύο}
    \label{fig:kicad-main}
  \end{center}
\end{figure}

Η εφαρμογή EEschema περιλαμβάνει πολλά εξαρτήματα, τα οποία μπορείτε να χρησιμοποιήσετε στα κυκλώματά σας. Τα εξαρτήματα αυτά είναι οργανωμένα σε βιβλιοθήκες. Το EEschema περιλαμβάνει αρχικά περίπου 30 βιβλιοθήκες οι οποίες περιλαμβάνται εξαρχής σε κάθε νέο έργο που δημιουργείαι. Για λόγους απλότητας του έργου μας, εμείς θα αφαιρέσουμε τις βιβλιοθήκες εξαρτημάτων που δεν έχουν εξαρτήματα χρήσιμα για το συγκεκριμένο κύκλωμα που σχεδιάζουμε.

Από το μενού στο πάνω μέρος της οθόνης του EEschema επιλέξτε "Προτιμήσεις" -> "Βιβλιοθήκες Εξαρτημάτων". Θα εμφανιστεί ένα παράθυρο με τις βιβλιοθήκες του έργου. Σε αυτό το παράθυρο επιλέξτε όλες τις βιβλιοθήκες (μία προς μία ή όλες μαζί) εκτός από τις βιβλιοθήκες power, device και conn και κάντε κλικ στο "Αφαίρεση". Θα πρέπει στο παράθυρο να φαίνονται μόνο οι τρεις βιβλιοθήκες power, device και conn. Κάντε κλικ στο "ΟΚ". Θα επανέλθετε στην κεντρική σελίδα της εφαρμογής EEschema.

Από το μενού στο πάνω μέρος της οθόνης του EEschema επιλέξτε "Αρχείο" -> "Αποθήκευση Σχηματικού Έργου".

\begin{figure}
  \begin{center}
    \includegraphics{img/kicad-main.png}
    \caption{Παράθυρο με τις βιβλιοθήκες του έργου}
    \label{fig:kicad-main}
  \end{center}
\end{figure}

Σε αυτή τη φάση είμαστε στο commit tag tut00.

\subsection{Προσθήκη εξαρτημάτων}
Εφόσον έχουμε δημιουργήσει και ρυθμίσει το φύλλο μας, πρέπει να προσθέσουμε τα εξαρτήματα που θα αποτελέσουν το κύκλωμά μας. Πρέπει να βρισκόμαστε στην κεντρική σελίδα της εφαρμογής EEschema.

Από το μενού στο πάνω μέρος της οθόνης του EEschema επιλέξτε "Τοποθέτηση" -> "Εξάρτημα" και κάντε κλικ στο δεξί τμήμα του φύλλου σας. Θα εμφανιστεί ένα παράθυρο επιλογής εξαρτήματος.

\begin{figure}
  \begin{center}
    \includegraphics{img/kicad-main.png}
    \caption{Παράθυρο επιλογής εξαρτήματος}
    \label{fig:kicad-main}
  \end{center}
\end{figure}

Στο παράθυρο επιλογής εξαρτήματος επιλέξτε το εξάρτημα με όνομα "C" (ένας πυκνωτής) από τη βιβλιοθήκη "device", πατήστε ΟΚ (θα επανέλθετε το φύλλο σχηματικού) και κάντε κλικ στο δεξί τμήμα του φύλλου σας ώστε να τοποθετηθεί στο φύλλο το εξάρτημα που επιλέξατε. 

Αφού τοποθετήσετε το εξάρτημα με όνομα "C" στο φύλλο, ακολουθήστε την ίδια διαδικασία ώστε να τοποθετήσετε στο φύλλο σας (στο δεξί τμήμα) όλα τα παρακάτω εξαρτήματα. 

\begin{itemize}
    \item R, από τη βιβλιοθήκη device
    \item LED, από τη βιβλιοθήκη device
    \item VCC, από τη βιβλιοθήκη power
    \item +3.3V, από τη βιβλιοθήκη power
    \item GND, από τη βιβλιοθήκη power
    \item CONN\_01X04, από τη βιβλιοθήκη conn
\end{itemize}

Με την ολοκλήρωση αυτών των τοποθετήσεων, έχουμε στο φύλλο μας 7 εξαρτήματα, τοποθετημένα όλα τακτοποιημένα στο δεξί τμήμα του φύλλου.

\begin{figure}
  \begin{center}
    \includegraphics{img/kicad-main.png}
    \caption{Φύλλο σχηματικού με τοποθετημένα 7 εξαρτήματα}
    \label{fig:kicad-main}
  \end{center}
\end{figure}

Για λόγους ευκολίας ανάγνωση του κυκλώματος, θα χρειαστεί να αλλάξουμε το πρόθεμα της ονομασίας αναφοράς του CONN\_01X04 από P σε J. Για να το κάνουμε αυτό πρέπει να κάνουμε δεξί κλικ πάνω στο εξάρτημα CONN\_01X04, και από το μενού που εμφανίζεται να επιλέξουμε "Επεξεργασία Εξαρτήματος" -> "Επεξεργασία". Αυτό θα μας εμφανίσει το παράθυρο "Ιδιότητες Εξαρτήματος", όπου πρέπει να ορίσουμε την Τιμή Πεδίου της Ονομασίας Αναφοράς να είναι J? και όχι P?.

\begin{figure}
  \begin{center}
    \includegraphics{img/kicad-main.png}
    \caption{Παράθυρο "Ιδιότητες Εξαρτήματος"}
    \label{fig:kicad-main}
  \end{center}
\end{figure}


Στη συνέχεια θα χρειαστεί να τοποθετήσουμε στο κύκλωμά μας και άλλα εξαρτήματα C, R και CONN\_01X04. Για να το κάνουμε αυτό, αντί να κάνουμε πάλι τοποθέτηση και να τα επιλέγουμε από τις βιβλιοθήκες, μπορούμε να κάνουμε αντιγραφή των εξαρτημάτων που ήδη έχουμε στο φύλλο μας. 

Για να αντιγράψουμε (δηλαδή να φτιάξουμε ακόμα ένα αντίγραφο) ενός εξαρτήματος πρέπει να κάνουμε δεξί κλικ πάνω στο εξάρτημα, από το μενού που εμφανίζεται να επιλέξουμε "Αντιγραφή", και να κάνουμε κλικ πάνω στο φύλλο εκεί όπου θέλουμε να φιταχτεί το αντίγραφο του εξαρτήματος. 

\begin{figure}
  \begin{center}
    \includegraphics{img/kicad-main.png}
    \caption{Αντιγραφή Εξαρτήματος}
    \label{fig:kicad-main}
  \end{center}
\end{figure}

Με αυτό τον τρόπο πρέπει να φτιάξουμε τα παρακάτω

\begin{itemize}
    \item 3 αντίγραφα του C, ώστε να έχουμε συνολικά 4 C στο φύλλο
    \item 2 αντίγραφα του R, ώστε να έχουμε συνολικά 3 R στο φύλλο
    \item 1 αντίγραφο του CONN\_01X04, ώστε να έχουμε συνολικά 2 CONN\_01X04 στο φύλλο
\end{itemize}

Με την ολοκλήρωση αυτών των τοποθετήσεων, έχουμε στο φύλλο μας 13 εξαρτήματα, τοποθετημένα όλα τακτοποιημένα στο δεξί τμήμα του φύλλου.

\begin{figure}
  \begin{center}
    \includegraphics{img/kicad-main.png}
    \caption{Φύλλο σχηματικού με τοποθετημένα 13 εξαρτήματα}
    \label{fig:kicad-main}
  \end{center}
\end{figure}

Σε αυτή τη φάση είμαστε στο commit tag tut01.

\end{document}
